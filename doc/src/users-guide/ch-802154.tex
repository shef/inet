\ifdraft

\chapter{The 802.15.4 Model}
\label{cha:802154}

\section{Overview}

IEEE 802.15.4 is a technical standard which defines the operation of low-rate
wireless personal area networks (LR-WPANs). IEEE 802.15.4 was designed for data
rates of 250 kbit/s or lower, in order to achieve long battery life (months or
even years) and very low complexity. The standard specifies the physical layer
and media access control.

IEEE 802.15.4is the basis for the ZigBee, ISA100.11a, WirelessHART, MiWi, SNAP,
and the Thread specifications, each of which further extends the standard by
developing the upper layers which are not defined in IEEE 802.15.4.
Alternatively, it can be used with 6LoWPAN, the technology used to deliver IPv6
over WPANs, to define the upper layers. (Thread is also 6LoWPAN-based.)

% https://en.wikipedia.org/wiki/IEEE_802.15.4

\begin{note}

802.15.4 can use narrowband radio (on one of three possible unlicensed frequency
bands), using DSSS or alternatively, a combination of binary keying and amplitude 
shift keying.

In August 2007, IEEE 802.15.4a was released, adding two more PHYs: Direct
Sequence ultra-wideband (UWB), and another one using chirp spread spectrum
(CSS).

In April, 2009 IEEE 802.15.4c and IEEE 802.15.4d were released expanding the
available PHYs with several additional PHYs: one for 780 MHz band using O-QPSK
or MPSK, another for 950 MHz using GFSK or BPSK.


MAC: Important features include real-time suitability by reservation of
guaranteed time slots, collision avoidance through CSMA/CA and integrated
support for secure communications. Devices also include power management
functions such as link quality and energy detection.

\end{note}

The INET Framework contains a basic implementation of IEEE 802.15.4 protocol.

TODO which PHYs are supported; the exact parameterization (modes?) is missing

TODO which MAC features are supported:

MAC: \tbf{OK}: collision avoidance through CSMA/CA; \tbf{missing}: reservation
of guaranteed time slots, and integrated support for secure communications.
Devices also include power management functions such as link quality and energy
detection.

\section{Network Interfaces}

There are two network interfaces that differ in the type of radio:

\begin{itemize}
  \item \nedtype{Ieee802154NarrowbandInterface} -- TODO
  \item \nedtype{Ieee802154UwbIrInterface} -- TODO
\end{itemize}


To create a wireless node with a 802.15.4 interface, use a node type 
that has a wireless interface, and set the interface type to the 
appropriate type. For example, \nedtype{WirelessHost} is a node type 
which is preconfigured to have one wireless interface, \ttt{wlan[0]}.
\ttt{wlan[0]} is of parametric type, so if you build the network from
\nedtype{WirelessHost} nodes, you can configure all of them to use
802.15.4 with the following line in the ini file:

\begin{inifile}
**.wlan[0].typename = "Ieee802154NarrowbandInterface"
\end{inifile}

Corresponding mediums:

Ieee802154UwbIrRadioMedium
Ieee802154NarrowbandRadioMedium -- missing?

\section{MAC Protocol}

Ieee802154Mac: ``Generic CSMA protocol supporting Mac-ACKs as well as
constant, linear or exponential backoff times.''

Ieee802154Mac
Ieee802154NarrowbandMac
Ieee802154UwbIrMac -- missing?

\section{Physical Layer}

Ieee802154NarrowbandScalarRadio
Ieee802154NarrowbandDimensionalRadio
Ieee802154UwbIrRadio

\fi

%%% Local Variables:
%%% mode: latex
%%% TeX-master: "usman"
%%% End:

