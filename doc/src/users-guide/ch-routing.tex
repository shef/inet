\chapter{Internet Routing}
\label{cha:routing}

\section{Overview}

INET Framework has models for several internet routing protocols, including
RIP, OSPF and BGP.

The easiest way to add routing to a network is to use the \nedtype{Router}
NED type for routers. \nedtype{Router} contains a conditional instance
for each of the above protocols. These submodules can be enabled by
setting the \ttt{hasRIP}, \ttt{hasOSPF} and/or \ttt{hasBGP} parameters to
\ttt{true}.

Example:

\begin{verbatim}
**.hasRIP = true
\end{verbatim}

There are also NED types called \nedtype{RipRouter}, \nedtype{OspfRouter},
\nedtype{BgpRouter}, which are all \nedtype{Router}s with appropriate
routing protocol enabled.

\section{RIP}
\label{sec:rip}

RIP (Routing Information Protocol) is a distance-vector routing protocol
which employs the hop count as a routing metric. RIP prevents routing loops
by implementing a limit on the number of hops allowed in a path from source
to destination.

The \nedtype{Rip} module implements distance vector routing as
specified in RFC 2453 (RIPv2) and RFC 2080 (RIPng). Configuration
can be specified in an XML file that can be specified in the
\ttt{ripConfig} parameter.

The configuration file specifies the per interface parameters.
Each \ttt{<interface>} element configures one or more interfaces;
the \ttt{hosts}, \ttt{names}, \ttt{towards}, \ttt{among} attributes
select the configured interfaces (in a similar way as with
\nedtype{Ipv4NetworkConfigurator} \ref{cha:network-autoconfiguration}).

Additional attributes:
\begin{itemize}
  \item \ttt{metric}: metric assigned to the link, default value is 1.
        This value is added to the metric of a learned route,
        received on this interface. It must be an integer in
        the [1,15] interval.
  \item \ttt{mode}: mode of the interface.
\end{itemize}

The mode attribute can be one of the following:
\begin{itemize}
  \item \ttt{'NoRIP'}: no RIP messages are sent or received on this interface.
  \item \ttt{'NoSplitHorizon'}: no split horizon filtering; send all routes to
        neighbors.
  \item \ttt{'SplitHorizon'}: do not sent routes whose next hop is the neighbor.
  \item \ttt{'SplitHorizonPoisenedReverse'} (default): if the next hop is the neighbor, then
  set the metric of the route to infinity.
\end{itemize}

The following example sets the link metric between router
\ttt{R1} and \ttt{RB} to 2, while all other links will have a metric of 1.
\begin{verbatim}
<RIPConfig>
  <interface among="R1 RB" metric="2"/>
  <interface among="R? R?" metric="1"/>
</RIPConfig>
\end{verbatim}

The \nedtype{Rip} module has the following parameters:
\begin{itemize}
  \item \ttt{mode}: either "RIPv2" (RFC 2453) or "RIPng" (RFC 2080)
  \item \ttt{routingTableModule}: path to the routing table module
        e.g. \ttt{'\^{}.ipv4.routingTable'}
  \item \ttt{ripConfig}: an XML configuration file containing per-interface parameters
\end{itemize}

The following example configures a \nedtype{Router} module to use RIPv2:
\begin{verbatim}
    **.hasRIP = true
    **.mode = "RIPv2"
    **.ripConfig = xmldoc("RIPConfig.xml")
\end{verbatim}

\section{OSPF}
\label{sec:ospf}

OSPF (Open Shortest Path First) is a routing protocol for IP networks.
It uses a link state routing (LSR) algorithm and falls into the group
of interior gateway protocols (IGPs), operating within a single
autonomous system (AS).

The \nedtype{Ospf} module implements the OSPF Version 2. Areas and routers
can be configured using an XML file.

\nedtype{OspfRouter} is a \nedtype{Router} with the OSPF protocol enabled.


\section{BGP}
\label{sec:bgp}

BGP (Border Gateway Protocol) is a standardized exterior gateway protocol
designed to exchange routing and reachability information among
autonomous systems (AS) on the Internet.

The \nedtype{Bgp} module implements BGP Version 4. The model implements
RFC 4271, with some limitations. Autonomous Systems and rules can be
configured in an XML file.

\nedtype{BgpRouter} is a \nedtype{Router} with the BGP protocol enabled.

%%% Local Variables:
%%% mode: latex
%%% TeX-master: "usman"
%%% End:

