\chapter{The MPLS Models}
\label{cha:mpls}

\section{Overview}

Multi-Protocol Label Switching (MPLS) is a ``layer 2.5'' protocol for
high-performance telecommunications networks. MPLS directs data from one network
node to the next based on short path labels rather than long network addresses,
avoiding complex lookups in a routing table and allowing traffic engineering.
The labels identify virtual links (label-switched paths or LSPs, sometimes also
called MPLS tunnels) between distant nodes rather than endpoints. The routers
that make up a label-switched network are called label-switching routers (LSRs)
inside the network (``transit nodes''), and label edge routers (LER) on the
edges of the network (``ingress'' or ``egress'' nodes).

Labels are distributed among routers via signaling protocols. The two main
label distribution protocols are LDP and RSVP-TE.

INET provides basic support for building MPLS simulations. It provides models
for the MPLS, LDP and RSVP-TE protocols and their associated data structures,
and preassembled MPLS-capable router models. 

\section{Core Modules}

The core modules are:

\begin{itemize}
  \item \nedtype{Mpls} implements the MPLS protocol 
  \item \nedtype{LibTable} holds the LIB (Label Information Base)
  \item \nedtype{Ldp} implements the LDP control signaling for MPLS 
  \item \nedtype{RsvpTe} implements the RSVP-TE control signaling for MPLS 
  \item \nedtype{Ted} contains the Traffic Engineering Database 
  \item \nedtype{LinkStateRouting} is a simple link-state routing protocol
  \item \nedtype{SimpleClassifier} is a configurable ingress classifier for MPLS
\end{itemize}

\subsection{Mpls}

The \nedtype{Mpls} module implements the MPLS protocol. MPLS is situated between
layer 2 and 3, and its main function is to switch packets based on their labels.
For that, it relies on the data structure called LIB (Label Information Base).
LIB is fundamentally a table with the following columns: \textit{input-interface},
\textit{input-label}, \textit{output-interface}, \textit{label-operation(s)}.

Upon receiving a labelled packet from another LSR, MPLS first extracts the
incoming interface and incoming label pair, and then looks it up in local LIB. 
If matching entry is found, it applies the prescribed label operations, and 
forwards the packet to the output interface. 

Label operations can be the following:

\begin{itemize}
  \item \textit{Push} adds a new MPLS label to a packet. (A packet may 
     contain multiple labels, acting as a stack.) When a normal IP packet
     enters an LSP, the new label will be the first label on the packet.
  \item \textit{Pop} removes the topmost MPLS label from a packet. 
     This is typically done at either the penultimate or the egress router.
  \item \textit{Swap}: Replaces the topmost label with a new label.
\end{itemize}

Upon receiving an unlabelled (e.g. plain IPv4) packet, MPLS first determines the
forwarding equivalence class (FEC) for the packet using a classifier, and then
inserts one or more labels in the packet's newly created MPLS header. The packet
is then passed on to the next hop router for the LSP.

In INET, the local LIB is stored in a \nedtype{LibTable} module in the router.

The ingress classifier is also a separate module; it is selected depending 
on the choice of the signaling protocol.


\subsection{LibTable}

\nedtype{LibTable} stores the LIB (Label Information Base), as described
in the previous section. \nedtype{LibTable} is expected to have one instance
in the router. 

LIB is normally filled and maintained by label distribution protocols (RSVP-TE,
LDP), but in INET it is possible to preload it with initial contents.

The \nedtype{LibTable} module accepts an XML config file with the following
format. The root element in config file must be \ttt{<libtable>}.
\ttt{<libtable>} may contain \ttt{<libentry>} child elements, each one
describing a row in table. \ttt{<libentry>} contains \ttt{<inLabel>},
\ttt{<inInterface>}, \ttt{<outInterface>}, \ttt{<outLabel>} elements, and
optionally also \ttt{<color>}. Label operations are given in \ttt{<op>} elements
under \ttt{<outLabel>}.

An example configuration:

\begin{XML}
<?xml version="1.0"?>
<libtable>
    <libentry>
        <inLabel>203</inLabel>
        <inInterface>ppp1</inInterface>
        <outInterface>ppp2</outInterface>
        <outLabel>
            <op code="pop"/>
            <op code="push" value="200"/>
        </outLabel>
        <color>200</color>
    </libentry>
</libtable>
\end{XML}

\subsection{Ldp}

The \nedtype{Ldp} module implements the LDP protocol. The purpose of Label
Distribution Protocol (LDP) is to distribute labels in an MPLS environment.
LDP relies on the underlying routing information provided by an IGP routing protocol
in order to forward label packets. The router forwarding information base, or
FIB, is responsible for determining the hop-by-hop path through the network.
Unlike traffic engineered paths, which use constraints and explicit routes to
establish LSPs, LDP is used only for signaling best-effort LSPs.

When \nedtype{Ldp} is used as signaling protocol, it also serves as ingress
classifier for \nedtype{Mpls}.

In INET, the \nedtype{Ldp} module takes routing information from \nedtype{Ted}
module. The \nedtype{Ted} instance in the network is filled and maintained
by a \nedtype{LinkStateRouting} module. Unfortunately, it is currently not
possible to use other routing protocol implementations such as \nedtype{Bgp}
in conjuction with \nedtype{Ldp}. 

\subsection{Ted}

The \nedtype{Ted} module contains the Traffic Engineering Database (TED). 

\subsection{LinkStateRouting}

The \nedtype{LinkStateRouting} module provides a simple link state routing
protocol. It uses \nedtype{Ted} as its link state database. Unfortunately, the
\nedtype{LinkStateRouting} module cannot operate independently, it can only be
used inside an MPLS router.

 \subsection{RsvpTe}

The \nedtype{RsvpTe} module implements RSVP-TE (Resource Reservation Protocol --
Traffic Engineering), as signaling protocol for MPLS. RSVP-TE handles bandwidth
allocation and allows traffic engineering across an MPLS network. Like LDP, RSVP
uses discovery messages and advertisements to exchange LSP path information
between all hosts. However, whereas LDP is restricted to using the configured
IGP's shortest path as the transit path through the network, RSVP can take
taking into consideration network constraint parameters such as available
bandwidth and explicit hops. RSVP uses a combination of the Constrained Shortest
Path First (CSPF) algorithm and Explicit Route Objects (EROs) to determine how
traffic is routed through the network.

When \nedtype{RsvpTe} is used as signaling protocol, \nedtype{Mpls} needs a
separate ingress classifier module, which is usually a \nedtype{SimpleClassifier}.

The \nedtype{RsvpTe} module allows LSPs to be specified statically in an XML
config file. An example \ttt{traffic.xml} file:

\begin{XML}
<?xml version="1.0"?>
<sessions>
    <session>
        <endpoint>host3</endpoint>
        <tunnel_id>1</tunnel_id>
        <paths>
            <path>
                <lspid>100</lspid>
                <bandwidth>100000</bandwidth>
                <route>
                    <node>10.1.1.1</node>
                    <lnode>10.1.2.1</lnode>
                    <node>10.1.4.1</node>
                    <node>10.1.5.1</node>
                </route>
                <permanent>true</permanent>
                <color>100</color>
            </path>
        </paths>
    </session>
</sessions>
\end{XML}

In the route, \ttt{<node>} stands for strict hop, and \ttt{<lnode>} for loose hop.

Paths can also be set up and torn down dynamically with \nedtype{ScenarioManager} 
commands (see chapter \ref{cha:scenario-management}). 
\nedtype{RsvpTe} understands the \ttt{<add-session>} and \ttt{<del-session>}
\nedtype{ScenarioManager} commands. The contents of the \ttt{<add-session>}
element can be the same as the \ttt{<session>} element for the \ttt{traffic.xml}
above. The \ttt{<del-command>} element syntax is also similar, but only
\ttt{<endpoint>}, \ttt{<tunnel\_id>} and \ttt{<lspid>} need to be specified.

The following is an example \ttt{scenario.xml} file:

\begin{XML}
<?xml version="1.0"?>
<scenario>
    <at t="2">
        <add-session module="LSR1.rsvp">
            <endpoint>10.2.1.1</endpoint>
            <tunnel_id>1</tunnel_id>
            <paths>
                ...
            </paths>
        </add-session>
    </at>
    <at t="2.4">
        <del-session module="LSR1.rsvp">
            <endpoint>10.2.1.1</endpoint>
            <tunnel_id>1</tunnel_id>
            <paths>
                <path>
                    <lspid>100</lspid>
                </path>
            </paths>
        </del-session>
    </at>
</scenario>
\end{XML}

\section{Classifier}

The \nedtype{SimpleClassifier} module implements an ingress classifier for
\nedtype{Mpls}. The classifier can be configured with an XML config file.

\begin{inifile}
**.classifier.config = xmldoc("fectable.xml");
\end{inifile}

An example \ttt{fectable.xml} file:

\begin{XML}
<?xml version="1.0"?>
<fectable>
    <fecentry>
        <id>1</id>
        <destination>host5</destination>
		<source>host1</source>
        <tunnel_id>1</tunnel_id>
        <lspid>100</lspid>
    </fecentry>
</fectable>
\end{XML}

\section{MPLS-Enabled Router Models}

INET provides the following pre-assembled MPLS routers:

\begin{itemize}
  \item \nedtype{LdpMplsRouter} is an MPLS router with the LDP signaling protocol
  \item \nedtype{RsvpMplsRouter} is an MPLS router with the RSVP-T signaling protocol
\end{itemize}

% HINT: A good MPLS primer:
% "MPLS for Dummies", Richard A Steenbergen <ras@nlayer.net>, nLayer Communications, Inc.
% https:www.nanog.org/meetings/nanog49/presentations/Sunday/mpls-nanog49.pdf


%%% Local Variables:
%%% mode: latex
%%% TeX-master: "usman"
%%% End:
