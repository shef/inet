\chapter{Other Network Protocols}
\label{cha:other-network-protocols}

\section{Overview}

Network layer protocols in INET are not restricted to IPv4 and IPv6. INET nodes such as
\nedtype{Router} and \nedtype{StandardHost} can be configured to use an alternative
network layer protocols instead of, or in addition to, IPv4 and IPv6.

Node models contain three optional network layers that can be individually
turned on or off:

\begin{ned}
ipv4: <ipv4NetworkLayerType> like INetworkLayer if hasIpv4;
ipv6: <ipv6NetworkLayerType> like INetworkLayer if hasIpv6;
generic: <networkLayerType> like INetworkLayer if hasGn;
\end{ned}

In the default configuration, only IPv4 is turned on. If you want to use an
alternative network layer protocol instead of IPv4/IPv6, your configuration will
look something like this:

\begin{inifile}
**.hasIpv4 = false
**.hasIpv6 = false
**.hasGn = true  <================= unused in NetworkLayerNodeBase, but referenced in TransportLayerNodeBase
**.networkLayerType = "WiseRouteNetworkLayer"
\end{inifile}

The list of alternative network layers includes:

\begin{itemize}
  \item \nedtype{SimpleNetworkLayer} is a generic network layer where the
    concrete protocol type is a parameter
  \item \nedtype{GenericNetworkLayer} is a network layer specialized 
    for the ``Generic Network Protocol'', an abstract implementation of the 
    next-hop routing concept
  \item \nedtype{WiseRouteNetworkLayer} is specialized for the Wise Route protocol
\end{itemize}

The list of network layer protocols that can be plugged into 
\nedtype{SimpleNetworkLayer} includes:

\begin{itemize}
  \item \nedtype{Flood} implements flooding
  \item \nedtype{WiseRoute} implements the Wise Route protocol (TODO which is...?)
  \item \nedtype{ProbabilisticBroadcast} TODO
  \item \nedtype{AdaptiveProbabilisticBroadcast} TODO
\end{itemize}

In addition to the network layer protocol, \nedtype{SimpleNetworkLayer} 
also includes an instance of \nedtype{EchoProtocol}, a module type that
implements a simple \textit{ping}-like protocol.

TODO possible error in networklayer.ini:
**.networkConfiguratorType = "Ipv4NetworkConfigurator" --- ???

\section{Protocols}

\subsection{Flood}

\nedtype{Flood} is a simple flooding protocol for network-level broadcast.
It remembers already broadcasted messages, and does not rebroadcast 
them if it gets another copy of that message.

TODO like INetworkProtocol

\subsection{ProbabilisticBroadcast}

\nedtype{ProbabilisticBroadcast} is a multi-hop ad-hoc data dissemination 
protocol based on probabilistic broadcast.

This method reduces the number of packets sent on the channel (reducing the
broadcast storm problem) at the risk of some nodes not receiving the data.
It is particularly interesting for mobile networks.

The transmission probability for each attempt, the time between two transmission
attempts, the maximum number of broadcast transmissions of a packet, and
some other settings are parameters.

\nedtype{AdaptiveProbabilisticBroadcast}

\nedtype{AdaptiveProbabilisticBroadcast} is a variant of
\nedtype{ProbabilisticBroadcast} that automatically adjusts transmission
probabilities depending on the estimated number of neighbours.

\subsection{WiseRoute}

\nedtype{WiseRoute} implements Wise Route, a simple loop-free routing algorithm
that builds a routing tree from a central network point, designed for sensor
networks and convergecast traffic.

The sink (the device at the center of the network) broadcasts
a route building message. Each network node that receives it
selects the sink as parent in the routing tree, and rebroadcasts
the route building message. This procedure maximizes the probability
that all network nodes can join the network, and avoids loops.
Parameter sinkAddress gives the sink network address,
rssiThreshold is a threshold to avoid using bad links (with too low
RSSI values) for routing, and routeFloodsInterval should be set to
zero for all nodes except the sink. Each routeFloodsInterval, the
sink restarts the tree building procedure. Set it to a large value
if you do not want the tree to be rebuilt.

\subsection{GenericNetworkProtocol}

This module is a simplified generic network protocol that routes
generic datagrams using different kind of network addresses. 

TODO GenericNetworkLayer like INetworkLayer  is a compound module

routingTable: GenericRoutingTable,
gnp: GenericNetworkProtocol,
echo: EchoProtocol,
arp: GenericArp



\section{Addressing Schemes}

TODO how to choose addressing scheme?  every addressing scheme works with every protocol?
  
\section{Address Resolution}

WiseRouteNetworkLayer = WiseRoute + GenericArp

\nedtype{GenericArp}, \nedtype{Arp}, \nedtype{GlobalArp}

TODO

\section{/////////////////////////////////////////////////////////}


\section{InternetCloud}

This module is an IPv4 router that can delay or drop packets (while retaining
their order) based on which interface card the packet arrived on and 
on which interface It is leaving the cloud. The delayer module is replacable.

By default the delayer module is ~MatrixCloudDelayer which lets you configure
the delay, drop and datarate parameters in an XML file. Packet flows, as defined
by incoming and outgoing interface pairs, are independent of each other.

The ~InternetCloud module can be used only to model the delay between two hops, but
it is possible to build more complex networks using several ~InternetCloud modules.

\section{PIM}

Protocol Independent Multicast -- not a network protocol 

models: \nedtype{PimSm}, \nedtype{PimDm}; \nedtype{Pim} is a compound module



%%% Local Variables:
%%% mode: latex
%%% TeX-master: "usman"
%%% End:

