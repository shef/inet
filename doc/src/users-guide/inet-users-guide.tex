\documentclass{book}
\usepackage{a4wide}

%% possible fonts -- in order of preference
%%\usepackage{palatino}
\usepackage{bookman}
%%\usepackage{charter}
%%\usepackage{newcent}
%%\usepackage{times}
%%\usepackage{avant}
%%\usepackage{helvet}
%%\usepackage{sans}
%%\usepackage{chancery}

\usepackage[svgnames]{xcolor}	% for color text support
\usepackage[T1]{fontenc}
\usepackage[11pt]{moresize}
\usepackage{setspace}
\usepackage{ifpdf}
\usepackage{makeidx}
\usepackage{longtable}  %% page wrapping table environment
\usepackage{colortbl}   %% colors for tables
\usepackage{fancyvrb}   %% the "Verbatim" environment
\usepackage{fancyhdr}   %% custom headers and footers
\usepackage{multicol}
\usepackage{listings}   %% source code listings with syntax highlight (lstxxx commands)
\usepackage[tight]{shorttoc}   %% for generating a second table of contents, only containing chapter titles
\usepackage{bytefield}  %% for drawing protocol frames
\usepackage{paralist}   %% for compact lists
\usepackage[nottoc]{tocbibind}  %% makes Bibliography and Index show up in TOC
\settocbibname{References}

\setlength{\textwidth}{160mm}
%\setlength{\oddsidemargin}{12.5mm}
%\setlength{\evensidemargin}{12.5mm}
%\setlength{\topmargin}{0mm}
\setlength{\textheight}{220mm}
%\setlength{\parskip}{1ex}
%\setlength{\parindent}{5ex}

\renewcommand{\bottomfraction}{0.9}
\renewcommand{\topfraction}{0.9}
\renewcommand{\floatpagefraction}{0.9}

%% try to cure overfull hboxes
%% \tolerance=500

%% for navigation in dvi files, only needed by old teTeX versions
%%\usepackage{srcltx}

%% try this for spell checking: cat ess2002.tex | ispell -l -t -a | sort | uniq | more

%%
%% The following snippet changes the horizontal spacing between the number and
%% the title in the table of contents.
%%
%% http://tex.stackexchange.com/questions/33841/how-to-modify-the-space-between-the-numbers-and-text-of-sectioning-titles-in-the
%%
\makeatletter
 \renewcommand*\l@section{\@dottedtocline{1}{2em}{3em}}
 \renewcommand*\l@subsection{\@dottedtocline{2}{5em}{4em}}
\renewcommand*\l@chapter[2]{%
  \ifnum \c@tocdepth >\m@ne
    \addpenalty{-\@highpenalty}%
    \vskip 1.0em \@plus\p@
    \setlength\@tempdima{2em}%
    \begingroup
      \parindent \z@ \rightskip \@pnumwidth
      \parfillskip -\@pnumwidth
      \leavevmode \bfseries
      \advance\leftskip\@tempdima
      \hskip -\leftskip
      #1\nobreak\hfil \nobreak\hb@xt@\@pnumwidth{\hss #2}\par
      \penalty\@highpenalty
    \endgroup
  \fi}
\makeatother

%%
%% OMNeT++ logo, use as {\opp}
%%
\makeatletter
%%\DeclareRobustCommand{\omnetpp}{OM\-NeT\kern-.18em++\@}
\DeclareRobustCommand{\omnetpp}{OMNeT++\@}
\makeatother

\newcommand{\opp}{\omnetpp}

%%
%% PDF Header
%%
% note: \ifpdf now comes from the ifpdf package
%\newif\ifpdf
%\ifx\pdfoutput\undefined
%  \pdffalse
%\else
%  \pdfoutput=1
%  \pdftrue
%\fi
%% PDF-Info
\ifpdf
  \usepackage[pdftex]{graphicx}
  \usepackage[plainpages=false,linktocpage,bookmarksnumbered=true,pdftex]{hyperref}   %% automatic hyperlinking
  \pdfcompresslevel=9
  \pdfinfo{/Author (Andras Varga and others)
    /Title (INET Framework User's Guide)
    /Subject ()
    /Keywords (INET, INETMANET, OMNeT++, manual)}
\else
  \usepackage{graphicx}
  \usepackage[plainpages=false]{hyperref}   %% automatic hyperlinking
\fi

%%
%% Draft conditional to include unfinished parts
%%
\newif\ifdraft
%\draftfalse %% uncomment for final version
\drafttrue %% uncomment for draft version

%%
%% Generate Index
%%
\makeindex


%%
%% Link colors (hyperref package)
%%
\definecolor{MyDarkBlue}{rgb}{0.16,0.16,0.5}
%% XXX the next line apparently screws up all links except in TOC! they'll be colored nicely, but won't work.
%\hypersetup{
%    colorlinks=true,
%    linkcolor=MyDarkBlue,
%    anchorcolor=MyDarkBlue,
%    citecolor=MyDarkBlue,
%    filecolor=MyDarkBlue,
%    menucolor=MyDarkBlue,
%    runcolor=MyDarkBlue,
%    urlcolor=blue,
%}

%%
%% Heading and Footer
%%
\pagestyle{fancy}
\fancyhf{}
\renewcommand{\footrulewidth}{0.5pt}
\renewcommand{\chaptermark}[1]{\markboth{#1}{}}
\lhead{INET Framework User's Guide -- \leftmark}
\rfoot{\thepage}

%% this is used for chapter start pages
\fancypagestyle{plain}{
    \rfoot{\thepage}
}

%%
%% Use \begin{graybox}...\end{graybox} for notes
%%
\definecolor{MyGray}{rgb}{0.85,0.85,1.0}
\makeatletter\newenvironment{graybox}%
   {\begin{flushright}\begin{lrbox}{\@tempboxa}\begin{minipage}[r]{0.95\textwidth}}%
   {\end{minipage}\end{lrbox}\colorbox{MyGray}{\usebox{\@tempboxa}}\end{flushright}}%
\makeatother


\newenvironment{note}{\begin{graybox}\textbf{NOTE: }}{\end{graybox}}
\newenvironment{warning}{\begin{graybox}\textbf{WARNING: }}{\end{graybox}}
\newenvironment{caution}{\begin{graybox}\textbf{CAUTION: }}{\end{graybox}}
\newenvironment{rationale}{\begin{graybox}\textbf{Rationale: }}{\end{graybox}}
\newenvironment{important}{\begin{graybox}\textbf{IMPORTANT: }}{\end{graybox}}

%%
%% Set up listings package
%%
\lstloadlanguages{C++,make,perl,tcl,XML,R,Matlab}

%% See listings.pdf,pp20
\lstdefinelanguage{NED} {
    morekeywords={allowunconnected,bool,channel,channelinterface,connections,const,
                  default,double,extends,false,for,gates,if,import,index,inout,input,
                  int,like,module,moduleinterface,network,output,package,parameters,
                  property,simple,sizeof,string,submodules,this,true,types,volatile,
                  xml,xmldoc},
    sensitive=true,
    morecomment=[l]{//},
    morestring=[b]",
}
\lstdefinelanguage{MSG} {
    morekeywords={abstract,bool,char,class,cplusplus,double,enum,extends,false,
                  fields,int,long,message,namespace,noncobject,packet,properties,
                  readonly,short,string,struct,true,unsigned},
    sensitive=true,
    morecomment=[l]{//},
    morestring=[b]",
}
\lstdefinelanguage{inifile} {
    morekeywords={},
    sensitive=true,
    morecomment=[l]{\#},
    morestring=[b]",
}
\lstdefinelanguage{pseudocode} {
    morekeywords={if,then,else,otherwise,whenever,while},
    sensitive=true,
    morecomment=[l]{//},
    morestring=[b]",
    mathescape=true,
}

%% thick ruler on the left; also, designate backtick as LaTeX escape character
%% (e.g. \opp needs to be written as `\opp` inside listing blocks)
\lstset{
    escapechar=`,
    basicstyle=\ttfamily,
    identifierstyle=\color{Black},
    stringstyle=\color{DarkBlue},
    commentstyle=\color{SeaGreen},
    keywordstyle=\bfseries\color{Purple},
    showstringspaces=false,
    frame=leftline,
    framesep=10pt,
    framerule=3pt,
    xleftmargin=15pt
}

\definecolor{NEDRulerColor}{rgb}{0.5,1.0,0.5}  % pale green
\definecolor{MSGRulerColor}{rgb}{0.5,1.0,0.5}  % pale green
\definecolor{CPPRulerColor}{rgb}{0.8,0.5,0.2}  % pale orange
\definecolor{IniRulerColor}{rgb}{0.9,0.9,0.3}  % pale yellow
\definecolor{FileListingRulerColor}{rgb}{0.85,0.85,0.85}  % grey
%\definecolor{CommandLineRulerColor}{rgb}{0.9,0.9,0.2}
\definecolor{PseudoCodeRulerColor}{rgb}{0.0,1.0,1.0}  % cyan
\definecolor{XMLRulerColor}{rgb}{0.8,0.8,1.0}  % pale blue

%% See listings.pdf,pp39
\lstnewenvironment{ned}
    {\lstset{language=NED,rulecolor=\color{NEDRulerColor}}}
    {}
\lstnewenvironment{msg}
    {\lstset{language=MSG,rulecolor=\color{MSGRulerColor}}}
    {}
\lstnewenvironment{cpp}
    {\lstset{language=C++,rulecolor=\color{CPPRulerColor}}}
    {}
\lstnewenvironment{inifile}
    {\lstset{language=inifile,rulecolor=\color{IniRulerColor}}}
    {}
\lstnewenvironment{filelisting}
    {\lstset{language={},rulecolor=\color{FileListingRulerColor}}}
    {}
\lstnewenvironment{commandline}
    {\lstset{language={},framesep=11pt,framerule=1pt,xleftmargin=16pt}}
    {}
\lstnewenvironment{pseudocode}
    {\lstset{language=pseudocode,rulecolor=\color{PseudoCodeRulerColor}}}
    {}
\lstnewenvironment{XML}
    {\lstset{language=XML,rulecolor=\color{XMLRulerColor}}}
    {}

% add caption={#2} to display caption
\newcommand{\xmlsnippet}[2]{%
    \lstinputlisting[language=XML,rulecolor=\color{XMLRulerColor},linerange=<!\-\-#1\-\->-<!\-\-End\-\->,includerangemarker=false,firstnumber=0]{Snippets.xml}}
\newcommand{\cppsnippet}[2]{%
    \lstinputlisting[language=C++,rulecolor=\color{CPPRulerColor},linerange=//!#1-//!End,includerangemarker=false,firstnumber=0]{Snippets.cc}}
\newcommand{\msgsnippet}[2]{%
    \lstinputlisting[language=msg,rulecolor=\color{MSGRulerColor},linerange=//!#1-//!End,includerangemarker=false,firstnumber=0]{Snippets.msg}}
\newcommand{\nedsnippet}[2]{%
    \lstinputlisting[language=ned,rulecolor=\color{NEDRulerColor},linerange=//!#1-//!End,includerangemarker=false,firstnumber=0]{Snippets.ned}}
\newcommand{\inisnippet}[2]{%
    \lstinputlisting[language=inifile,rulecolor=\color{IniRulerColor},linerange=\#!#1-\#!End,includerangemarker=false,firstnumber=0]{Snippets.ini}}

%%
%% some customization
%%
\setlength{\parindent}{0pt}
\setlength{\parskip}{1ex}

%%
%% Shortcuts
%%
\newcommand{\appendixchapter}{\chapter} %% html converter needs to know which chapters are appendices

\newcommand{\tbf}{\textbf} %% bold faced text
\newcommand{\ttt}{\texttt} %% type writer font text

\newcommand{\tab}{\hspace*{5mm}} %% tabulator settings

\newcommand{\new}{$^{New!}$}
\newcommand{\changed}{$^{Changed!}$}

\newcommand{\program}{\textbf}

%% Colordefinition for table header rows (requires package colortbl)
\newcommand{\tabheadcol}{\rowcolor[gray]{0.8}}

%%
%% Module parameters list
%%
\newenvironment{params}{\begin{itemize}}{\end{itemize}}
\newcommand{\param}[2]{\item \fpar{#1}: #2}

%%
%% Function/Class/Macro/Variable/Program/Parameter/Define names
%%
%% Write the names in type writer font and do an index entry
%% Allows word wrap by automatic hyphenation
%%
%% Usage: \ffunc{take()}
%%    or: \ffunc[take()]{take(obj)}
%% the second form uses the bracketed word for the index entry
%%

\newcommand{\protocol}[1]{%
    {#1}}

%% NED type names
\newcommand{\nedtype}[2][\DefaultOpt]{\def\DefaultOpt{#2}%
  \index{#1}%
  \texttt{\hyphenchar\font=`\-\relax#2}}

%% MSG type names
\newcommand{\msgtype}[2][\DefaultOpt]{\def\DefaultOpt{#2}%
  \index{#1}%
  \texttt{\hyphenchar\font=`\-\relax#2}}

%% Function names
\newcommand{\ffunc}[2][\DefaultOpt]{\def\DefaultOpt{#2}%
  \index{#1}%
  \texttt{\hyphenchar\font=`\-\relax#2}}

%% Class names
\newcommand{\cppclass}[2][\DefaultOpt]{\def\DefaultOpt{#2}%
  \index{#1}%
  \texttt{\hyphenchar\font=`\-\relax#2}}

%% Macro names
\newcommand{\fmac}[2][\DefaultOpt]{\def\DefaultOpt{#2}%
  \index{#1}%
  \texttt{\hyphenchar\font=`\-\relax#2}}

%% Variable names
\newcommand{\fvar}[2][\DefaultOpt]{\def\DefaultOpt{#2}%
  \index{#1}%
  \texttt{\hyphenchar\font=`\-\relax#2}}

%% Program names
\newcommand{\fprog}[2][\DefaultOpt]{\def\DefaultOpt{#2}%
  \index{#1}%
  \texttt{\hyphenchar\font=`\-\relax#2}}

%% Parameter names
\newcommand{\fpar}[2][\DefaultOpt]{\def\DefaultOpt{#2}%
  \index{#1}%
  \texttt{\hyphenchar\font=`\-\relax#2}}

%% Defines
\newcommand{\fdef}[2][\DefaultOpt]{\def\DefaultOpt{#2}%
  \index{#1}%
  \texttt{\hyphenchar\font=`\-\relax#2}}

%% NED/MSG properties
\newcommand{\fprop}[2][\DefaultOpt]{\def\DefaultOpt{#2}%
  \index{#1}%
  \texttt{\hyphenchar\font=`\-\relax#2}}

%% Keywords (NED, MSG)
\newcommand{\fkeyword}[2][\DefaultOpt]{\def\DefaultOpt{#2}%
  \index{#1}%
  \textbf{\texttt{\hyphenchar\font=`\-\relax#2}}}

%% Configuration options
\newcommand{\fconfig}[2][\DefaultOpt]{\def\DefaultOpt{#2}%
  \index{#1}%
  \textbf{\texttt{\hyphenchar\font=`\-\relax#2}}}

%% File names
\newcommand{\ffilename}[2][\DefaultOpt]{\def\DefaultOpt{#2}%
  \index{#1}%
  \texttt{\hyphenchar\font=`\-\relax#2}}

%% Signals
\newcommand{\fsignal}[2][\DefaultOpt]{\def\DefaultOpt{#2}%
  \index{#1}%
  \texttt{\hyphenchar\font=`\-\relax#2}}

\newcommand{\fgate}[1]{\texttt{\hyphenchar\font=`\-\relax#1}}

%% do not number subsubsections
%\setcounter{secnumdepth}{4}

% limit the depth of TOC
\setcounter{tocdepth}{2}

%%
%% Start of document
%%
\begin{document}

%% set the image type preference
\DeclareGraphicsExtensions{.pdf,.png}

\pagestyle{empty}
\pagenumbering{roman}
\include{title}
\cleardoublepage

%%\setcounter{page}{1}
%\newpage
%%\pagenumbering{roman}

%% \shorttableofcontents{Chapters}{0}
%% \cleardoublepage

\tableofcontents
\cleardoublepage

\pagestyle{fancy}
\pagenumbering{arabic}

\include{ch-introduction}
\cleardoublepage

\include{ch-usage}
\cleardoublepage

\chapter{Networks}
\label{cha:networks}

%
% This chapter provides practical guidance on how to put together various 
% networks from the built-in node models and how to configure them,
% WITHOUT LOOKING AT THE INTERNALS OF THOSE NODES.
%

\section{Overview}

%TODO: wired, wireless, mixed wired/wireless, various topologies + generated, hierarchical, parametric
%TODO: ethernet networks, mpls networks, vpn, tunneling, PPP networks, sensor networks

INET heavily builds upon the modular architecture of OMNeT++. It provides 
numerous domain specific and highly parameterizable components which can be
combined in many ways. The primary means of building large custom network 
simulations in INET is the composition of existing models with custom models,
starting from small components and gradually forming ever larger ones up until
the composition of the network. Users are not required to have programming 
experience to create simulations unless they also want to implement 
their own protocols, for example.

Assembling an INET simulation starts with defining a module representing
the network. Networks are compound modules which contain network nodes,
automatic network configurators, and sometimes additionally transmission
medium, physical environment, various visualizer, and other infrastructure
related modules. Networks also contain connections between network nodes
representing cables. Large hierarchical networks may be further organized
into compound modules to directly express the hierarchy.

There are no predefined networks in INET, because it is very easy to create
one, and because of the vast possibilities. However, the OMNeT++ IDE provides
several topology generator wizards for advanced scenarios.

As INET is an OMNeT++-based framework, users mainly use NED to describe the
model topology, and ini files to provide configuration.\footnote{Some 
components require additional configuration to be provided as separate
files, e.g. in XML.} 

\section{Built-in Network Nodes and Other Top-Level Modules}

INET provides several pre-assembled network nodes with carefully selected
components. They support customization via parameters and parametric
submodule types, but they are not meant to be universal. Sometimes it may
be necessary to create special network node models for particular
simulation scenarios. In any case, the following list gives a taste of the
built-in network nodes.

\begin{itemize}
  \item \nedtype{StandardHost} contains the most common Internet protocols:
     \protocol{UCP}, \protocol{TDP}, \protocol{IPv4}, \protocol{IPv6},
     \protocol{Ethernet}, \protocol{IEEE 802.11}. It also supports an
     optional mobility model, optional energy models, and any number of
     applications which are entirely configurable from INI files.
  \item \nedtype{EtherSwitch} models an \protocol{Ethernet} switch containing
     a relay unit and one MAC unit per port.
  \item \nedtype{Router} provides the most common routing protocols:
     \protocol{OSPF}, \protocol{BGP}, \protocol{RIP}, \protocol{PIM}.
  \item \nedtype{AccessPoint} models a Wifi access point with multiple
     \protocol{IEEE 802.11} network interfaces and multiple \protocol{Ethernet}
     ports.
  \item \nedtype{WirelessHost} provides a network node with one (default)
     \protocol{IEEE 802.11} network interface in infrastructure mode,
     suitable for using with an \nedtype{AccessPoint}.
  \item \nedtype{AdhocHost} is a \nedtype{WirelessHost} with the network
     interface configured in ad-hoc mode and forwarding enabled.
  \item \nedtype{AodvRouter} is similar to an \nedtype{AdhocHost} with
     an additional \protocol{AODV} protocol.
\end{itemize}

Network nodes communicate at the network level by exchanging OMNeT++ messages 
which are the abstract representations of physical signals on the 
transmission medium.  Signals are either sent through OMNeT++ connections 
in the wired case, or sent directly to the gate of the receiving network node 
in the wireless case. Signals encapsulate INET-specific packets that represent 
the transmitted digital data. Packets are further divided into chunks that
provide alternative representations for smaller pieces of data (e.g. 
protocol headers, application data).

Additionally, there are components that occur on network level, but they
are not models of physical network nodes. They are necessary 
to model other aspects. Some of them are:

\begin{itemize}
  \item A \textit{radio medium} module such as \nedtype{Ieee80211RadioMedium},
     \nedtype{ApskScalarRadioMedium} and \nedtype{UnitDiskRadioMedium} 
     (there are a few of them) are a required component of wireless networks. 
  \item \nedtype{PhysicalEnvironment} models the effect of the physical 
     environment (i.e. obstacles) on radio signal propagation. It is an
     optional component.
  \item \textit{Configurators} such as \nedtype{Ipv4NetworkConfigurator},
     \nedtype{L2NetworkConfigurator} and \nedtype{GenericNetworkConfigurator}
     configure various aspects of the network. For example,  
     \nedtype{Ipv4\-Network\-Configurator} assigns IP addresses 
     to hosts and routers, and sets up static routing. It is used 
     when modeling dynamic IP address assignment (e.g. via DHCP) or 
     dynamic routing is not of importance. \nedtype{L2NetworkConfigurator} 
     allows one to configure 802.1 LANs and provide STP/RSTP-related 
     parameters such as link cost, port priority and the ``is-edge'' flag.
  \item \nedtype{ScenarioManager} allows scripted scenarios, such
     as timed failure and recovery of network nodes.
  \item \textit{Group coordinators} are needed for the operation of some 
     group mobility mdels. For example, \nedtype{MoBanCoordinator} is 
     the coordinator module for the MoBAN mobility model.
  \item \textit{Visualizers} like \nedtype{PacketDropOsgVisualizer} provide
     graphical rendering of some aspect of the simulation either in
     2D (canvas) or 3D (using OSG or osgEarth). The usual choice is 
     \nedtype{IntegratedVisualizer} which bundles together an instance 
     of each specific visualizer type in a compound module.
\end{itemize}

\section{Typical Networks}

\subsection{Wired Networks}

Wired network connections, for example \protocol{Ethernet} cables, are
represented with standard OMNeT++ connections using the
\nedtype{DatarateChannel} NED type. The channel's \nedtype{datarate} and
\nedtype{delay} parameters must be provided for all wired connections.

The following example shows how straightforward it is to create a model for
a simple wired network. This network contains a server connected to a router
using \protocol{PPP}, which in turn is connected to a switch using
\protocol{Ethernet}. The network also contains a parameterizable number of
clients, all connected to the switch forming a star topology. The utilized
network nodes are all predefined modules in INET. To avoid the manual
configuration of IP addresses and routing tables, an automatic network
configurator is also included.

\nedsnippet{WiredNetworkExample}{Wired network example}

In order to run a simulation using the above network, an OMNeT++ INI file must
be created. The INI file selects the network, sets its number of clients
parameter, and configures a simple \protocol{TCP} application for each
client. The server is configured to have a \protocol{TCP} application which
echos back all data received from the clients individually.

\inisnippet{WiredNetworkConfigurationExample}{Wired network configuration example}

When the above simulation is run, each client application connects to the
server using a \protocol{TCP} socket. Then each one of them sends 1MB of
data, which in turn is echoed back by the server, and the simulation
concludes. The default statistics are written to the \texttt{results}
folder of the simulation for later analysis.

\subsection{Wireless Networks}

TODO: AccessPoint, WirelessHost infrastructure mode

Wireless network connections are not modeled with OMNeT++ connections due the
dynamically changing nature of connectivity. For wireless networks, an
additional module, one that represents the transmission medium, is required to
maintain connectivity information.

TODO 

\nedsnippet{WirelessNetworkExample}{Wireless network example}

TODO adjust text: 

In the above network, positions in the display strings provide 
positions for the transmission medium during the computation of 
signal propagation and path loss. 

In addition, \ttt{host1} is configured to periodically send 
\protocol{UDP} packets to \ttt{host2} over the AP.

\inisnippet{WirelessNetworkConfigurationExample}{Wireless network configuration example}



\subsection{Mobile Ad hoc Networks}

TODO commentary

\nedsnippet{MobileAdhocNetworkExample}{Mobile ad hoc network example}

TODO

\inisnippet{MobileAdhocNetworkConfigurationExample}{Mobile ad hoc network configuration example}

TODO



\section{Frequent Tasks (How To...)}

Quick and somewhat superficial advice to many practical tasks.

\subsection{Automatic Wired Interfaces}

In many wired network simulations, the number of wired interfaces need not
be manually configured, because it can be automatically inferred from the
actual number of connections between network nodes.

\nedsnippet{AutomaticWiredInterfacesExample}{Automatic wired interfaces
example}

\subsection{Multiple Wireless Interfaces}

All built-in wireless network nodes support multiple wireless interfaces,
but only one is enabled by default.

\inisnippet{MultipleWirelessInterfacesExample}{Multiple wireless interfaces
example}

\subsection{Traffic Generation}

TODO scripted, synthetic, CBR, VBR, trace-based, .... 
TODO app[]; other ways 

\subsection{Specifying Addresses}

Nearly all application layer modules, but several other components as well,
have parameters that specify network addresses. They typically accept
addresses given with any of the following syntax variations:

\begin{itemize}
  \item literal IPv4 address: \ttt{"186.54.66.2"}
  \item literal IPv6 address: \ttt{"3011:7cd6:750b:5fd6:aba3:c231:e9f9:6a43"}
  \item module name: \ttt{"server"}, \ttt{"subnet.server[3]"}
  \item interface of a host or router: \ttt{"server/eth0"}, \ttt{"subnet.server[3]/eth0"}
  \item IPv4 or IPv6 address of a host or router: \ttt{"server(ipv4)"},
      \ttt{"subnet.server[3](ipv6)"}
  \item IPv4 or IPv6 address of an interface of a host or router:
      \ttt{"server/eth0(ipv4)"}, \ttt{"subnet.server[3]/eth0(ipv6)"}
\end{itemize}

TODO ini example


\subsection{Node Failure and Recovery}

\subsection{Enabling Dual IP Stack}

All built-in network nodes support dual Internet protocol stacks, that is
both \protocol{Ipv4} and \protocol{Ipv6} are available. They are also
supported by transport layer protocols, link layer protocols, and most
applications. Only \protocol{Ipv4} is enabled by default, so in order to
use \protocol{Ipv6}, it must be enabled first, and an application
supporting \protocol{Ipv6} (e.g., \nedtype{PingApp} must be used). The
following example shows how to configure two ping applications in a single
node where one is using an \protocol{Ipv4} and the other is using an
\protocol{Ipv6} destination address.

\inisnippet{DualStackExample}{Dual stack example}

\subsection{Enabling Packet Forwarding}

In general, network nodes don't forward packets by default, only
\nedtype{Router} and the like do. Nevertheless, it's possible to enable
packet forwarding as simply as flipping a switch.

\inisnippet{ForwardingExample}{Forwarding example}

\subsection{Adding Routing Protocols}

TODO internet routing and ad hoc routing

\subsection{Node Mobility}

TODO

\subsection{Topology Generation}

TODO: wizards, generated topologies

\subsection{Hierarchical Networks}

TODO: nested compound modules

%%% Local Variables:
%%% mode: latex
%%% TeX-master: "usman"
%%% End:



\cleardoublepage

\chapter{Network Nodes}
\label{cha:network-nodes}

\section{Overview}

Hosts, routers, switches, access points, mobile phones, and other network
nodes are represented in INET with compound modules. The previous chapter 
has introduced a few node types like \nedtype{StandardHost}, \nedtype{Router}, 
and showed how to put together networks from them. In this chapter,
we look at the internals of such node models, in order to provide a deeper
understanding of their customization possibilities and to give some guidance
on how custom nodes models can be assembled.

\section{Ingredients}

Node models are assembled from other modules which represent applications, 
communication protocols, network interfaces, routing tables, mobility models, 
energy models, and other functionality. These modules fall into the following
broad categories:

\begin{itemize}
  \item \emph{Applications} often model the user behavior as well as the
     application program (e.g., browser), and the application layer protocol
     (e.g., \protocol{HTTP}). Applications typically use transport layer
     protocols (e.g., \protocol{TCP} and/or \protocol{UDP}), but they may
     also directly use lower layer protocols (e.g., \protocol{IP} or
     \protocol{Ethernet}) via sockets.
  \item \emph{Routing protocols} are provided as separate modules:
     \protocol{OSPF}, \protocol{BGP}, or \protocol{AODV} for MANET routing.
     These modules use \protocol{TCP}, \protocol{UDP}, and \protocol{IPv4},
     and manipulate routes in the \nedtype{Ipv4\-RoutingTable} module.
  \item \emph{Transport layer protocols} are connected to applications and
     network layer protocols. They are most often represented by simple
     modules, currently \protocol{TCP}, \protocol{UDP}, and \protocol{SCTP}
     are supported. \protocol{TCP} has several implementations: \nedtype{Tcp}
     is the OMNeT++ native implementation; \nedtype{TcpLwip} module wraps the
     lwIP \protocol{TCP} stack; and \nedtype{TcpNsc} module wraps the
     Network Simulation Cradle library.
  \item \emph{Network layer protocols} are connected to transport layer
     protocols and network interfaces. They are usually modeled as compound
     modules: \nedtype{Ipv4NetworkLayer} for \protocol{IPv4}, and
     \nedtype{Ipv6NetworkLayer} for \protocol{IPv6}. The \nedtype{Ipv4NetworkLayer}
     module contains several protocol modules: \nedtype{Ipv4}, \nedtype{Arp},
     and \nedtype{Icmpv4}.
  \item \emph{Network interfaces} are represented by compound modules
     which are connected to the network layer protocols and other network
     interfaces in the wired case. They are often modeled as compound modules
     containing separate modules for queues, classifiers, MAC, and PHY protocols.
  \item \emph{Link layer protocols} are usually simple modules sitting
     in network interface modules. Some protocols, for example
     \protocol{IEEE 802.11 MAC}, are modeled as a compound module themselves
     due to the complexity of the protocol.
  \item \emph{Physical layer protocols} are compound modules also being part
     of network interface modules.
  \item \emph{Interface table} maintains the set of network interfaces
     (e.g. \texttt{eth0}, \texttt{wlan0}) in the network node. Interfaces
     are registered dynamically during initialization of network interfaces.
  \item \emph{Routing tables} maintain the list of routes for the corresponding
     network protocol (e.g., \nedtype{Ipv4RoutingTable} for \nedtype{Ipv4}).
     Routes are added by automatic network configurators or routing protocols.
     Network protocols use the routing tables to find out the best matching
     route for datagrams.
  \item \emph{Mobility modules} are responsible for moving around the network
     node in the simulated playground. The mobility model is mandatory for
     wireless simulations even if the network node is stationary. The mobility
     module stores the location of the network node which is needed to compute
     wireless propagation and path loss. Different mobility models are provided
     as different modules. Network nodes define their mobility submodule with
     a parametric type, so the mobility model can be changed in the configuration.
  \item \emph{Energy modules} model energy storage mechanisms, energy
     consumption of devices and software processes, energy generation of devices,
     and energy management processes which shutdown and startup network nodes.
  \item \emph{Status} (\nedtype{NodeStatus}) keeps track of the status of the
     network node (up, down, etc.)
  \item \emph{Other modules} with particular functionality such as
     \nedtype{PcapRecorder} are also available.
\end{itemize}

\section{Node Architecture}

Within network nodes, OMNeT++ connections are used to represent 
communication opportunities between protocols. Packets and 
messages sent on these connections represent software or hardware activity.

Although protocols may also be connected to each other directly, 
in most cases they are connected via \emph{dispatcher modules}.
Dispatchers (\nedtype{MessageDispatcher}) are small, low-overhead modules 
that allow protocol components to be connected in one-to-many and many-to-many 
fashion, and ensure that messages and packets sent from one component end up
being delivered to the correct component. Dispatchers need no manual 
configuration, as they use discovery and peek into packets.

In there pre-assembled node models, dispatchers allow arbitrary
protocol components to talk directly to each other, i.e. not only
to ones in neighboring layers.

TODO standalone modules like Ipv4RoutingTable: other modules find them by name, 
see "routingTableModule" parameter

TODO examples: direct connection between protocols, layered, star topology

\section{Customizing Nodes}

The built-in network nodes are written to be as versatile and customizable
as possible. This is achieved in several ways:

\subsection*{Submodule and Gate Vectors}

One way is the use of gate vectors and submodule vectors. The sizes 
of vectors may come from parameters or derived by the number of
external connections to the network node. For example, a host may
have an arbitrary number of wireless interfaces, and it will automatically
have as many \protocol{Ethernet} interfaces as the number of \protocol{Ethernet} 
devices connected to it.

For example, wireless interfaces for hosts are defined like this:

\begin{ned}
wlan[numWlanInterfaces]: <snip> // wlan interfaces in StandardHost etc al.
\end{ned}

Where \ttt{numWlanInterfaces} is a module parameter that defaults to 
either 0 or 1 (this is different for e.g. \nedtype{StandardHost} and
\nedtype{WirelessHost}.) To configure a host to have two interfaces, 
add the following line to the ini file: 

\begin{inifile}
**.hostA.numWlanInterfaces = 2
\end{inifile}

\subsection*{Conditional Submodules}

Submodules that are not vectors are often conditional. For example,
the \protocol{TCP} protocol module in hosts is conditional on
the \ttt{hasTcp} parameter. Thus, to disable \protocol{TCP} support
in a host (it is enabled by default), use the following ini file line:

\begin{inifile}
**.hostA.hasTcp = false
\end{inifile}

\subsection*{Parametric Types}

Another often used way of customization is parametric types, that is, the
type of a submodule (or a channel) may be specified as a string parameter.
Almost all submodules in the built-in node types have parametric types.
For example, the \protocol{TCP} protocol module is defined like this:

\begin{ned}
tcp: <tcpType> like ITcp if hasTcp;
\end{ned}

The \ttt{tcpType} parameter defaults to the default implementation, \nedtype{Tcp}.
To use another implementation instead, add the following line to the ini file:
 
\begin{inifile}
**.host*.tcpType = "TcpLwip"  # use lwIP's TCP implementation
\end{inifile}

Submodule vectors with parametric types are defined without the use of a
module parameter to allow elements have different types. An example
is how applications are defined in hosts: 

\begin{ned}
app[numApps]: <> like IApp;  // applications in StandardHost et al.
\end{ned}

And applications can be added in the following way:

\begin{inifile}
**.hostA.numApps = 2
**.hostA.apps[0].typename = "UdpBasicApp"
**.hostA.apps[1].typename = "PingApp"
\end{inifile}

\subsection*{Inheritance}

Inheritance can be use to derive new, specialized node types from existing ones.
A derived NED type may add new parameters, gates, submodules or connections,
and may set inherited unassigned parameters to specific values.

For example, \nedtype{WirelessHost} is derived from \nedtype{StandardHost}
in the following way:

\begin{ned}
module WirelessHost extends StandardHost 
{
    @display("i=device/wifilaptop");
    numWlanInterfaces = default(1);
}
\end{ned}

\section{Custom Network Nodes}

Despite the many pre-assembled network nodes and the several available
customization options, sometimes it is just easier to build a network node
from scratch. The following example shows how easy it is to build a simple
network node.

This network node already contains a configurable application and several
standard protocols. It also demonstrates how to use the packet dispatching
mechanism which is required to connect multiple protocols in a many-to-many
relationship.

\nedsnippet{NetworkNodeExample}{Network node example}



%%% Local Variables:
%%% mode: latex
%%% TeX-master: "usman"
%%% End:


\cleardoublepage

\chapter{Network Interfaces}
\label{cha:network-interfaces}

\section{Overview}

%TODO: MAC address, op mode, duplex mode, data rate, transmission power, queue limits, FCS mode

In INET simulations, network interface modules are the primary means of
communication between network nodes. They represent the required
combination of software and hardware elements from an operating system
point-of-view. 

Network interfaces are implemented with OMNeT++ compound modules that
conform to the \nedtype{INetworkInterface} module interface. 
Network interfaces can be further categorized as wired and wireless;
they conform to the \nedtype{IWiredInterface} and \nedtype{IWirelessInterface}
NED types, respectively, which are subtypes of \nedtype{INetworkInterface}.

\section{Built-in Network Interfaces}

INET provides pre-assembled network interfaces for several standard
protocols, protocol tunneling, hardware emulation, etc. The following list
gives the most commonly used network interfaces.

\begin{itemize}
    \item \nedtype{EthernetInterface} represents an \protocol{Ethernet} interface
    \item \nedtype{PppInterface} is for wired links using \protocol{PPP}
    \item \nedtype{Ieee80211Interface} represents a Wifi (\protocol{IEEE 802.11}) interface
    \item \nedtype{Ieee802154Interface} represents a \protocol{IEEE 802.15.4} interface
    \item \nedtype{BMacInterface}, \nedtype{LMacInterface}, \nedtype{XMacInterface} provide 
      low-power wireless sensor MAC protocols along with a simple hypothetical PHY protocol
    \item \nedtype{TunInterface} is a tunneling interface that can be directly used by applications
    \item \nedtype{LoopbackInterface} provides local loopback within the network node
    \item \nedtype{ExtInterface} represents a real-world interface, suitable for hardware-in-the-loop simulations
\end{itemize}

\section{Anatomy of Network Interfaces}

Network interfaces in the INET Framework are OMNeT++ compound modules that
contain many more components than just the corresponding layer 2 protocol
implementation. Most of these components are optional, i.e. absent by default,
and can be added via configuration.

Typical ingredients are:

\begin{itemize}
    \item \emph{Layer 2 protocol implementation}. For some interfaces such as
      \nedtype{PppInterface} this is a single module; for others like Ethernet
      and Wifi it consists of separate modules for MAC, LLC, and possibly 
      other subcomponents. 
    \item \emph{PHY model}. Some interfaces also contain separate
      module(s) that implement the physical layer. For example, 
      \nedtype{Ieee80211Interface} contains a radio module.
    \item \emph{Output queue}. This module is optional and absent by default, 
      because most MAC protocol implementations already contain an internal queue
      which is more efficient to work with. The possibility to plug in an 
      external queue module allows one to experiment with different queueing policies
      and implement QoS, RED, etc.
    \item \emph{Traffic conditioners} allow traffic shaping and policing elements
      to be added to the interface, for example to implement a Diffserv router.
    \item \emph{Hooks} allow extra modules to be inserted in the incoming
      and outgoing paths of packets. 
\end{itemize}


\subsection{Internal vs External Output Queue}

Network interfaces usually have the external queue module defined with a
parametric type like this: 

\begin{ned}
queue: <queueType> like IOutputQueue if queueType != "";
\end{ned} 

When \fpar{queueType} is empty (this is the default), the external queue 
module is absent, and the MAC (or equivalent L2) protocol will use its 
internal queue object. Conceptually, the internal queue is of inifinite size, 
but for better diagnostics one can often specify a hard limit for the queue
length in a module parameter -- if this is exceeded, the simulation 
stops with an error.

When \fpar{queueType} is not empty, it must name a NED type that 
implements the \nedtype{IOutputQueue} interface. The external 
queue module model allows modeling a finite buffer, or implement
various queueing policies for QoS and/or RED.

The most frequently used module type for external queue is 
\nedtype{DropTailQueue}, a finite-size FIFO that drops overflowing 
packets). Other queue types that implement queueing policies can be 
created by assembling compound modules from DiffServ components 
(see chapter \ref{cha:diffserv}). An example of such compound
modules is \nedtype{DiffservQueue}.

An example ini file fragment that installs drop-tail queues of size 10
on PPP interfaces:

\begin{inifile}
**.ppp[*].queueType = "DropTailQueue"
**.ppp[*].queue.frameCapacity = 10
\end{inifile}

\subsection{Traffic Conditioners}

Many network interfaces contain optional traffic conditioner submodules
defined with parametric types, like this: 

\begin{ned}
ingressTC: <ingressTCType> like ITrafficConditioner if ingressTCType != "";
egressTC: <egressTCType> like ITrafficConditioner if egressTCType != "";
\end{ned}

Traffic conditioners allow one to implement the policing and shaping actions
of a Diffserv router. They are added to the input or output packets paths  
in the network interface. (On the output path they are added before the queue 
module.) 

Traffic conditioners must implement the \nedtype{ITrafficConditioner} module
interface. Traffic conditioners can be assembled from DiffServ components 
(see chapter \ref{cha:diffserv}). There is no preassembled traffic conditioner
in INET, but you can find some in the example simulations.

An example configuration with fictituous types:

\begin{inifile}
**.ppp[*].ingressTCType = "CustomIngressTC"
**.ppp[*].egressTCType = "CustomEgressTC"
\end{inifile}


\subsection{Hooks}

Several network interfaces allow extra modules to be inserted in the incoming
and outgoing paths of packets at the top of the netwok interface. 
Hooks are added as a submodule vector with parametric type, like this: 

\begin{ned}
outputHook[numOutputHooks]: <default("Nop")> like IHook if numOutputHooks>0;
inputHook[numInputHooks]: <default("Nop")> like IHook if numInputHooks>0;
\end{ned}

This allows any number of hook modules to be added. The hook modules 
are chained in their numeric order.

Modules inserted as hooks may act as probes (for measuring or recording
traffic) or as means of modifying or perturbing the packet flow for 
experimentation. Module types implementing the \nedtype{IHook} NED interface
include \nedtype{ThruputMeter}, \nedtype{Delayer}, \nedtype{OrdinalBasedDropper},
and \nedtype{OrdinalBasedDuplicator}. 

The following ini file fragment inserts two hook modules into the output
paths of PPP interfaces, a delayer and a throughput meter:

\begin{inifile}
**.ppp[*].numOutputHooks = 2
**.ppp[*].outputHook[0].typename = "Delayer"
**.ppp[*].outputHook[1].typename = "ThruputMeter"
**.ppp[*].outputHook[0].delay = 3ms
\end{inifile}



\section{The Interface Table}

Network nodes normally contain an \nedtype{InterfaceTable} module.
The interface table is a sort of registry of all the network interfaces
in the host. It does not send or receive messages, other modules access it
via C++ function calls. Contents of the interface table can also
be inspected e.g. in Qtenv.

Network interfaces register themselves in the interface table at the
beginning of the simulation. Registration is usually the task of the
MAC (or equivalent) module. 


\section{Wired Network Interfaces}

Wired interfaces have a pair of special purpose OMNeT++ gates which represent
the capability of having an external physical connection to another network
node (e.g. Ethernet port). In order to make wired communication work,
these gates must be connected with special connections which represent the
physical cable between the physical ports. The connections must use special
OMNeT++ channels (e.g. \nedtype{DatarateChannel}) which determine datarate
and delay parameters.

Wired network interfaces are compound modules that implement the 
\nedtype{IWiredInterface} interface. INET has the following
wired network interfaces. 

\subsection{PPP}

Network interfaces for point-to-point links (\nedtype{PppInterface}) are 
described in chapter \ref{cha:ppp}. They are typically used in routers.

\subsection{Ethernet}

Ethernet interfaces (\nedtype{EthernetInterface}), alongside with models 
of Ethernet devices such as switches and hubs, are described in chapter
\ref{cha:ethernet}.

\section{Wireless Network Interfaces}

Wireless interfaces use direct sending\footnote{OMNeT++ \ttt{sendDirect()} calls} 
for communication instead of links, so their compound modules do not have
output gates at the physical layer, only an input gate dedicated to receiving. 
Another difference from the wired case is that wireless interfaces 
require (and collaborate with) a \textit{transmission medium} module 
at the network level. The medium module represents the shared transmission 
medium (electromagnetic field or acoustic medium), is responsible for 
modeling physical effects like signal attenuation, and maintains 
connectivity information. Also, while wired interfaces can do without
explicit modeling of the physical layer, a PHY module is an indispensable
part of a wireless interface.

Wireless network interfaces are compound modules that implement the 
\nedtype{IWirelessInterface} interface. In the following sections we 
give an overview of the wireless interfaces available in INET.

\subsection{Generic Wireless Interface}

The \nedtype{WirelessInterface} compound module is a generic implementation
of \nedtype{IWirelessInterface}. In this network interface, the types of the
MAC protocol and the PHY layer (the radio) are parameters:

\begin{ned}
mac: <macType> like IMacProtocol;
radio: <radioType> like IRadio if radioType != "";
\end{ned}

There are specialized versions of \nedtype{WirelessInterface} where 
the MAC and the radio modules are fixed to a particular value. 
One example is \nedtype{BMacInterface}, which contains a \nedtype{BMac}
and an \nedtype{ApskRadio}.

\subsection{IEEE 802.11}

IEEE 802.11 or Wifi network interfaces (\nedtype{Ieee80211Interface}),
alongside with models of devices acting as access points (AP),
are covered in chapter \ref{cha:80211}.

\subsection{IEEE 802.15.4}

\nedtype{Ieee802154Interface} is covered in a separate chapter, see \ref{cha:802154}.

\subsection{Wireless Sensor Networks}

MAC protocols for wireless sensor networks (WSNs) and the corresponding
network interfaces are covered in chapter \ref{cha:sensor-macs}.

\subsection{CSMA/CA} 

\nedtype{CsmaCaMac} implements an imaginary CSMA/CA-based MAC protocol with
optional acknowledgements and a retry mechanism. With the appropriate settings,
it can approximate basic 802.11b ad-hoc mode operation.

\nedtype{CsmaCaMac} provides a lot of room for experimentation: 
acknowledgements can be turned on/off, and operation parameters like
inter-frame gap sizes, backoff behaviour (slot time, minimum and maximum 
number of slots), maximum retry count, header and ACK frame sizes, bit rate,
etc. can be configured via NED parameters.

\nedtype{CsmaCaInterface} interface is a \nedtype{WirelessInterface} with
the MAC type set to \nedtype{CsmaCaMac}. 

\subsection{Acking MAC}

Not every simulation requires a detailed simulation of the lower layers.
\nedtype{AckingWirelessInterface} is a highly abstracted wireless interface 
that offers simplicity for scenarios where Layer 1 and 2 effects can be 
completely ignored, for example testing the basic functionality of a 
wireless ad-hoc routing protocol.

\nedtype{AckingWirelessInterface} is a \nedtype{WirelessInterface} 
parameterized to contain a unit disk radio (\nedtype{UnitDiskRadio})
and a trivial MAC protocol (\nedtype{AckingMac}). 

The most important parameter \nedtype{UnitDiskRadio} accepts is the 
transmission range. When a radio transmits a frame, all other radios 
within transmission range are able to receive the frame correctly, 
and radios that are out of range will not be affected at all. 
Interference modeling (collisions) is optional, and it is recommended
to turn off with \nedtype{AckingMac}.

\nedtype{AckingMac} implements a trivial MAC protocol that has packet
encapsulation and decapsulation, but no real medium access protocol. 
Frames are simply transmitted on the wireless channel as soon as the
transmitter becomes idle. (There is no carrier sense, collision avoidance, 
or collison detection.) \nedtype{AckingMac} also provides an optional 
out-of-band acknowledgement mechanism (using C++ function calls, 
not actual wirelessly sent frames), which is turned on by default.
There is no retransmission: if an acknowledgement does not arrive
after the first transmission of the packet, the MAC discards the
packet and counts it as ``given up''. 

\subsection{Shortcut}

\nedtype{ShortcutMac} implements error-free ``teleportation'' of packets 
to the peer MAC entity, with some delay computed from a transmission 
duration and a propagation delay. The physical layer is completely bypassed.
The corresponding network interface type, \nedtype{ShortcutInterface},
does not even have a radio model.

\nedtype{ShortcutInterface} is useful for modeling wireless networks
where full connectivity is assumed, and Layer 1 and Layer 2 effects
can be completely ignored. 

\section{Special-Purpose Network Interfaces}

 
\subsection{Tunnelling}

\nedtype{TunInterface} is a virtual network interface that can be used 
for creating tunnels, but it is more powerful than that.
It lets an application-layer module capture packets sent to 
the TUN interface and do whatever it pleases with it (including
sending it to a peer entity in an UDP or plain IPv4 packet.)

To set up a tunnel, add an instance of \nedtype{TunnelApp} to 
the node, and specify the protocol (IPv4 or UDP) and the remote
endpoint of the tunnel (address and port) in parameters. 

TODO example: see examples/inet/tunnel

\subsection{Local Loopback}

\nedtype{LoopbackInterface} provides local loopback within the network node.

\subsection{External Interface}

\nedtype{ExtInterface} represents a real-world interface, suitable for 
hardware-in-the-loop simulations. External interfaces are explained in 
chapter \ref{cha:emulation}.

\section{Custom Network Interfaces}

It's also possible to build custom network interfaces, the following
example shows how to build a custom wireless interface.

\nedsnippet{WirelessInterfaceExample}{Wireless interface example}

The above network interface contains very simple hypothetical MAC and PHY
protocols. The MAC protocol only provides acknowledgment without other
services (e.g., carrier sense, collision avoidance, collision detection),
the PHY protocol uses one of the predefined APSK modulations for the whole
signal (preamble, header, and data) without other services (e.g.,
scrambling, interleaving, forward error correction).


%%% Local Variables:
%%% mode: latex
%%% TeX-master: "usman"
%%% End:



\cleardoublepage

\chapter{Applications}
\label{cha:apps}


\section{Overview}

This chapter describes application models and traffic generators.

\section{TCP applications}

This sections describes the applications using the TCP protocol.
Each application must implement the \nedtype{ITCPApp} module interface
to ease configuring the \nedtype{StandardHost} module.

The applications described here are all contained by the
\nedtype{inet.applications.tcpapp} package. These applications use
\msgtype{GenericAppMsg} objects to represent the data sent between the client
and server. The client message contains the expected reply length, the
processing delay, and a flag indicating that the connection should be closed
after sending the reply. This way intelligence (behaviour specific to the
modelled application, e.g. HTTP, SMB, database protocol) needs only to be
present in the client, and the server model can be kept simple and dumb.


\subsection{TcpBasicClientApp}

Client for a generic request-response style protocol over TCP.
May be used as a rough model of HTTP or FTP users.

The model communicates with the server in sessions. During a session,
the client opens a single TCP connection to the server, sends several
requests (always waiting for the complete reply to arrive before
sending a new request), and closes the connection.

The server app should be \nedtype{TcpGenericServerApp}; the model sends
\msgtype{GenericAppMsg} messages.

Example settings:

FTP:

\begin{inifile}
numRequestsPerSession = exponential(3)
requestLength = truncnormal(20,5)
replyLength = exponential(1000000)
\end{inifile}

HTTP:

\begin{inifile}
numRequestsPerSession = 1 # HTTP 1.0
numRequestsPerSession = exponential(5)  # HTTP 1.1, with keepalive
requestLength = truncnormal(350,20)
replyLength = exponential(2000)
\end{inifile}

Note that since most web pages contain images and may contain frames,
applets etc, possibly from various servers, and browsers usually download
these items in parallel to the main HTML document, this module cannot
serve as a realistic web client.

Also, with HTTP 1.0 it is the server that closes the connection after
sending the response, while in this model it is the client.

\subsection{TcpSinkApp}

Accepts any number of incoming TCP connections, and discards whatever
arrives on them.

The module parameter \fpar{dataTransferMode} should be set the transfer mode in TCP layer.
Its possible values (``bytecount'', ``object'', ``bytestream'') are described in ...

\subsection{TcpGenericServerApp}

Generic server application for modelling TCP-based request-reply style
protocols or applications.

Requires message object preserving sendQueue/receiveQueue classes
to be used with \nedtype{Tcp} (that is, TCPMsgBasedSendQueue and TCPMsgBasedRcvQueue;
TCPVirtualBytesSendQueue/RcvQueue are not good).

The module accepts any number of incoming TCP connections, and expects
to receive messages of class \msgtype{GenericAppMsg} on them. A message should
contain how large the reply should be (number of bytes). \nedtype{TcpGenericServerApp}
will just change the length of the received message accordingly, and send
back the same message object. The reply can be delayed by a constant time
(replyDelay parameter).

\subsection{TcpEchoApp}

The \nedtype{TcpEchoApp} application accepts any number of incoming TCP
connections, and sends back the messages that arrive on them, The lengths of the
messages are multiplied by \fpar{echoFactor} before sending them back (echoFactor=1
will result in sending back the same message unmodified.) The reply can also be
delayed by a constant time (\fpar{echoDelay} parameter).

When \nedtype{TcpEchoApp} receives data packets from TCP (and such, when they can be
echoed) depends on the dataTransferMode setting.
With "bytecount" and "bytestream", TCP passes up data to us
as soon as a segment arrives, so it can be echoed immediately.
With "object" mode, our local TCP reproduces the same
messages that the sender app passed down to its TCP -- so if the sender
app sent a single 100 MB message, it will be echoed only when all
100 megabytes have arrived.

\subsection{TcpSessionApp}

Single-connection TCP application: it opens a connection, sends
the given number of bytes, and closes. Sending may be one-off,
or may be controlled by a "script" which is a series of
(time, number of bytes) pairs. May act either as client or as server,
and works with TCPVirtualBytesSendQueue/RcvQueue as sendQueue/receiveQueue
setting for ~TCP.
Compatible with both IPv4 (~IPv4) and ~IPv6.

\subsubsection*{Opening the connection}

Regarding the type of opening the connection, the application may
be either a client or a server. When active=false, the application
will listen on the given local localPort, and wait for an incoming connection.
When active=true, the application will bind to given local localAddress:localPort,
and connect to the connectAddress:connectPort. To use an ephemeral port
as local port, set the localPort parameter to -1.

Even when in server mode (active=false), the application will only
serve one incoming connection. Further connect attempts will be
refused by TCP (it will send RST) for lack of LISTENing connections.

The time of opening the connection is in the tOpen parameter.

\subsubsection*{Sending data}

Regardless of the type of OPEN, the application can be made to send
data. One way of specifying sending is via the tSend, sendBytes
parameters, the other way is sendScript. With the former, sendBytes
bytes will be sent at tSend. With sendScript, the format is
"<time> <numBytes>;<time> <numBytes>;..."

\subsubsection*{Closing the connection}

The application will issue a TCP CLOSE at time tClose. If tClose=-1, no
CLOSE will be issued.



\subsection{TelnetApp}

Models Telnet sessions with a specific user behaviour.
The server app should be \nedtype{TcpGenericServerApp}.

In this model the client repeats the following activity
between \fpar{startTime} and \fpar{stopTime}:

\begin{enumerate}
\item opens a telnet connection
\item sends \fpar{numCommands} commands. The command is \fpar{commandLength} bytes
      long. The command is transmitted as entered by the user character by character,
      there is \fpar{keyPressDelay} time between the characters. The server echoes
      each character. When the last character of the command is sent (new line),
      the server responds with a \fpar{commandOutputLength} bytes long message.
      The user waits \fpar{thinkTime} interval between the commands.
\item closes the connection and waits \fpar{idleInterval} seconds
\item if the connection is broken it is noticed after \fpar{reconnectInterval}
      and the connection is reopened
\end{enumerate}

Each parameter in the above description is ``volatile'', so you can
use distributions to emulate random behaviour.

Additional parameters:
addresses,ports
dataTransferMode

\begin{note}
This module emulates a very specific user behaviour, and as such,
it should be viewed as an example rather than a generic Telnet model.
If you want to model realistic Telnet traffic, you are encouraged
to gather statistics from packet traces on a real network, and
write your model accordingly.
\end{note}

\subsection{TcpServerHostApp}

This module hosts TCP-based server applications. It dynamically creates
and launches a new "thread" object for each incoming connection.

Server threads should be subclassed from the \cppclass{TcpServerThreadBase}
C++ class, registered in the C++ code using the Register\_Class() macro,
and the class name should be specified in the serverThreadClass
parameter of \nedtype{TcpServerHostApp}. The thread object will receive events
via a callback interface (methods like established(), dataArrived(),
peerClosed(), timerExpired()), and can send packets via TCPSocket's send()
method.

Example server thread class: \cppclass{TcpGenericServerThread}.

\begin{important}
Before you try to use this module, make sure you actually need it!
In most cases, \nedtype{TcpGenericServerApp} and \msgtype{GenericAppMsg} will be completely
enough, and they are a lot easier to handle. You'll want to subclass your
client from \cppclass{TCPGenericCliAppBase} then; check \nedtype{TelnetApp} and
\nedtype{TcpBasicClientApp} for examples.
\end{important}


\section{UDP applications}

All UDP applications should be derived from the \nedtype{IUDPApp} module interface,
so that the application of \nedtype{StandardHost} could be configured without changing its NED file.

The following applications are implemented in INET:
\begin{itemize}
\item \nedtype{UdpBasicApp} sends UDP packets to a given IP address at a given interval
\item \nedtype{UdpBasicBurst} sends UDP packets to the given IP address(es) in bursts, or acts as a packet sink.
\item \nedtype{UdpEchoApp} similar to \nedtype{UdpBasicApp}, but it sends back the packet after reception
\item \nedtype{UdpSink} consumes and prints packets received from the \nedtype{Udp} module
\item \nedtype{UdpVideoStreamClient},\nedtype{UdpVideoStreamServer} simulates UDP streaming
\end{itemize}

The next sections describe these applications in details.

\subsection{UdpBasicApp}

The \nedtype{UdpBasicApp} sends UDP packets to a the IP addresses given in the
\fpar{destAddresses} parameter. The application sends a message to one of the
targets in each \fpar{sendInterval} interval. The interval between message and
the message length can be given as a random variable. Before the packet is
sent, it is emitted in the \fsignal{sentPk} signal.

The application simply prints the received UDP datagrams. The \fsignal{rcvdPk}
signal can be used to detect the received packets.

The number of sent and received messages are saved as scalars at the end of the
simulation.

% could be a simple packet generator without the ability to receive packets?

\subsection{UdpSink}

This module binds an UDP socket to a given local port, and prints the
source and destination and the length of each received packet.

% TODO does not accept broadcast messages

\subsection{UdpEchoApp}

Similar to \nedtype{UdpBasicApp}, but it sends back the packet after reception.
It accepts only packets with \msgtype{UDPEchoAppMsg} type, i.e. packets that
are generated by another \nedtype{UdpEchoApp}.

When an echo response received, it emits an \fsignal{roundTripTime} signal.

\subsection{UdpVideoStreamClient}

This module is a video streaming client. It send one ``video streaming request'' to
the server at time \fpar{startTime} and receives stream from \nedtype{UdpVideoStreamServer}.

The received packets are emitted by the \fsignal{rcvdPk} signal.

\subsection{UdpVideoStreamServer}

This is the video stream server to be used with \nedtype{UdpVideoStreamClient}.

The server will wait for incoming "video streaming requests".
When a request arrives, it draws a random video stream size
using the \fpar{videoSize} parameter, and starts streaming to the client.
During streaming, it will send UDP packets of size \fpar{packetLen} at every
\fpar{sendInterval}, until \fpar{videoSize} is reached. The parameters \fpar{packetLen}
and \fpar{sendInterval} can be set to constant values to create CBR traffic,
or to random values (e.g. sendInterval=uniform(1e-6, 1.01e-6)) to
accomodate jitter.

The server can serve several clients, and several streams per client.

% FIXME why streamVector? VideoStreamData could be deleted immediately after last byte sent
% TODO this is video-on-demand, support multicast/broadcast video streaming too

\subsection{UdpBasicBurst}

Sends UDP packets to the given IP address(es) in bursts, or acts as a
packet sink. Compatible with both IPv4 and IPv6.

\subsubsection*{Addressing}

The \fpar{destAddresses} parameter can contain zero, one or more destination
addresses, separated by spaces. If there is no destination address given,
the module will act as packet sink. If there are more than one addresses,
one of them is randomly chosen, either for the whole simulation run,
or for each burst, or for each packet, depending on the value of the
\fpar{chooseDestAddrMode} parameter. The \fpar{destAddrRNG} parameter controls which
(local) RNG is used for randomized address selection.
The own addresses will be ignored.

An address may be given in the dotted decimal notation, or with the module
name. (The \cppclass{L3AddressResolver} class is used to resolve the address.)
You can use the "Broadcast" string as address for sending broadcast messages.

INET also defines several NED functions that can be useful:
\begin{itemize}
\item[-] moduleListByPath("pattern",...): \\
         Returns a space-separated list of the modulenames.
         All modules whole getFullPath() matches one of the pattern parameters will get included.
         The patterns may contain wilcards in the same syntax as in ini files.
         See cTopology::extractByModulePath() function
         example: destaddresses = moduleListByPath("**.host[*]", "**.fixhost[*]")
\item[-] moduleListByNedType("fully.qualified.ned.type",...): \\
         Returns a space-separated list of the modulenames with the given NED type(s).
         All modules whose getNedTypeName() is listed in the given parameters will get included.
         The NED type name is fully qualified.
         See cTopology::extractByNedTypeName() function
         example: destaddresses = moduleListByNedType("inet.nodes.inet.StandardHost")
\end{itemize}

The peer can be UDPSink or another UDPBasicBurst.

\subsubsection*{Bursts}

The first burst starts at \fpar{startTime}. Bursts start by immediately sending
a packet; subsequent packets are sent at \fpar{sendInterval} intervals. The
sendInterval parameter can be a random value, e.g. exponential(10ms).
A constant interval with jitter can be specified as 1s+uniform(-0.01s,0.01s)
or uniform(0.99s,1.01s). The length of the burst is controlled by the
\fpar{burstDuration} parameter. (Note that if \fpar{sendInterval} is greater than
\fpar{burstDuration}, the burst will consist of one packet only.) The time between
burst is the \fpar{sleepDuration} parameter; this can be zero (zero is not
allowed for \fpar{sendInterval}.) The zero \fpar{burstDuration} is interpreted as infinity.

\subsubsection*{Packets}

Packet length is controlled by the \fpar{messageLength} parameter.

The module adds two parameters to packets before sending:
\begin{itemize}
\item[-] sourceID: source module ID
\item[-] msgId: incremented by 1 after send any packet.
\end{itemize}
When received packet has this parameters, the module checks the order of received packets.

\subsubsection*{Operation as sink}

When \fpar{destAddresses} parameter is empty, the module receives packets and makes statistics only.

\subsubsection*{Statistics}

Statistics are collected on outgoing packets:
\begin{itemize}
\item[-] sentPk: packet object
\end{itemize}

Statistics are collected on incoming packets:
\begin{itemize}
\item[-] outOfOrderPk: statistics of out of order packets.
       The packet is out of order, when has msgId and sourceId parameters and module
       received bigger msgId from same sourceID.
\item[-] dropPk: statistics of dropped packets.
       The packet is dropped when not out-of-order packet and delay time is larger than
       delayLimit parameter. The delayLimit=0 is infinity.
\item[-] rcvdPk: statistics of not dropped, not out-of-order packets.
\item[-] endToEndDelay: end to end delay statistics of not dropped, not out-of-order packets.
\end{itemize}


\section{SCTP applications}

TODO


\section{IPv4/IPv6 traffic generators}

The applications described in this section use the services of the network
layer only, they do not need transport layer protocols.
They can be used with both IPv4 and IPv6.

\nedtype{IIPvXTraffixGenerator} (prototype) sends IP or IPv6 datagrams to the
given address at the given \fpar{sendInterval}.
The \fpar{sendInterval} parameter can be a constant or a random value (e.g. exponential(1)).
If the \fpar{destAddresses} parameter contains more than one address, one
of them is randomly for each packet. An address may be given in the
dotted decimal notation (or, for IPv6, in the usual notation with colons),
or with the module name. (The \cppclass{L3AddressResolver} class is used to resolve
the address.) To disable the model, set destAddresses to "".

The \nedtype{IpvxTrafGen} sends messages with length \fpar{packetLength}.
The sent packet is emitted in the \fsignal{sentPk} signal.
The length of the sent packets can be recorded as scalars and vectors.

The \nedtype{IpvxTrafSink} can be used as a receiver of the packets
generated by the traffic generator. This module emits the packet
in the \fsignal{rcvdPacket} signal and drops it. The \ttt{rcvdPkBytes}
and \ttt{endToEndDelay} statistics are generated from this signal.

The \nedtype{IpvxTrafGen} can also be the peer of the traffic generators;
it handles the received packets exactly like \nedtype{IpvxTrafSink}.

\section{The PingApp application}

The \nedtype{PingApp} application
generates ping requests and calculates the packet loss and round trip
parameters of the replies.

Start/stop time, sendInterval etc. can be specified via parameters. An address
may be given in the dotted decimal notation (or, for IPv6, in the usual
notation with colons), or with the module name.
(The \cppclass{L3AddressResolver} class is used to resolve the address.)
To disable send, specify empty destAddr.

Every ping request is sent out with a sequence number, and replies are
expected to arrive in the same order. Whenever there's a jump in the
in the received ping responses' sequence number (e.g. 1, 2, 3, 5), then
the missing pings (number 4 in this example) is counted as lost.
Then if it still arrives later (that is, a reply with a sequence number
smaller than the largest one received so far) it will be counted as
out-of-sequence arrival, and at the same time the number of losses is
decremented. (It is assumed that the packet arrived was counted earlier as a loss,
which is true if there are no duplicate packets.)

Uses \msgtype{PingPayload} as payload for the ICMP(v6) Echo Request/Reply packets.

\subsection*{Parameters}

\begin{itemize}
  \item \fpar{destAddr}: destination address
  \item \fpar{srcAddr}: source address (useful with multi-homing)
  \item \fpar{packetSize}: of ping payload, in bytes (default is 56)
  \item \fpar{sendInterval}: time to wait between pings (can be random, default is 1s)
  \item \fpar{hopLimit}: TTL or hopLimit for IP packets (default is 32)
  \item \fpar{count}: stop after \fpar{count} ping request, 0 means continuously
  \item \fpar{startTime}: send first ping request at \fpar{startTime}
  \item \fpar{stopTime}: time of finish sending, 0 means forever
  \item \fpar{printPing}: dump on stdout (default is \fkeyword{false})
\end{itemize}

\subsection*{Signals and Statistics}

\begin{itemize}
  \item \fsignal{rtt} value of the round trip time
  \item \fsignal{numLost} number of lost packets
  \item \fsignal{outOfOrderArrivals} number of packets arrived out-of-order
  \item \fsignal{pingTxSeq} sequence number of the sent ping request
  \item \fsignal{pingRxSeq} sequence number of the received ping response
\end{itemize}

% FIXME seqNo should be part of ICMPMessage


\section{Ethernet applications}

The \nedtype{inet.applications.ethernet} package contains modules
for a simple client-server application. The \nedtype{EtherAppClient} is a simple
traffic generator that peridically sends \msgtype{EtherAppReq} messages
whose length can be configured. destAddress, startTime,waitType, reqLength, respLength

The server component of the model (\nedtype{EtherAppServer}) responds with a
\msgtype{EtherAppResp} message of the requested length. If the response does
not fit into one ethernet frame, the client receives the data in multiple
chunks.

% FIXME reqLength>1500 causes an error in the LLC module
% FIXME numFrames field of EtherAppRes is not used
% FIXME server always sends 1497 byte chunks, it should depend on the framing (1497 is for LLC)
% FIXME if registerSAP is false (default), the and EtherLLC used, then the client won't receive messages (auto config?)
% FIXME Ieee802Nic -> EthernetInterface in the NED comment

Both applications have a \fpar{registerSAP} boolean parameter.
This parameter should be set to \ttt{true} if the application is connected
to the \nedtype{EtherLlc} module which requires registration of the SAP
before sending frames.

Both applications collects the following statistics: sentPkBytes, rcvdPkBytes,
endToEndDelay.

The client and server application works with any model that accepts
Ieee802Ctrl control info on the packets (e.g. the 802.11 model).
The applications should be connected directly to the \nedtype{EtherLlc}
or an EthernetInterface NIC module.

The model also contains a host component that groups the applications
and the LLC and MAC components together (\nedtype{EtherHost}). This node does
not contain higher layer protocols, it generates Ethernet traffic directly.
By default it is configured to use half duplex MAC (CSMA/CD).



%%% Local Variables:
%%% mode: latex
%%% TeX-master: "usman"
%%% End:


\cleardoublepage

\chapter{Transport Protocols}
\label{cha:transport-protocols}

\section{TCP}
\label{sec:tcp}

\subsection{Overview}
\label{sec:tcp_overview}

The \nedtype{Tcp} simple module is the primary implementation 
of the TCP protocol in the INET framework.

INET contains two other implementation of the TCP protocol:
\nedtype{TCP\_lwIP} and \nedtype{TCP\_NSC}, which are based
on external codebases.

All TCP modules implements the \nedtype{ITcp} interface and
communicate with the application and the IP layer through the
same interface, so they can be interchanged and can also be 
mixed in the same network.

\subsection{Protocol Overview}
\label{sec:tcp_prot}

TCP protocol is the most widely used protocol of the Internet. It provides
reliable, ordered delivery of stream of bytes from one application on one computer
to another application on another computer. It is used by such applications as
World Wide Web, email, file transfer amongst others.

The baseline TCP protocol is described in RFC793, but other tens of RFCs
contains modifications and extensions to the TCP. These proposals
enhance the efficiency and safety of the TCP protocol and they are widely
implemented in the real TCP modules. As a result, TCP is a complex protocol
and sometimes it is hard to see how the different requirements interacts
with each other.

The TCP modules of the INET framework implements the following RFCs:

\begin{tabular}{ll}
RFC 793 & Transmission Control Protocol \\
RFC 896 & Congestion Control in IP/TCP Internetworks \\
RFC 1122 & Requirements for Internet Hosts -- Communication Layers \\
RFC 1323 & TCP Extensions for High Performance \\
RFC 2018 & TCP Selective Acknowledgment Options \\
RFC 2581 & TCP Congestion Control \\
RFC 2883 & An Extension to the Selective Acknowledgement (SACK) Option for TCP \\
RFC 3042 & Enhancing TCP's Loss Recovery Using Limited Transmit \\
RFC 3390 & Increasing TCP's Initial Window \\
RFC 3517 & A Conservative Selective Acknowledgment (SACK)-based Loss Recovery \newline
                 Algorithm for TCP \\
RFC 3782 & The NewReno Modification to TCP's Fast Recovery Algorithm \\
\end{tabular}

In this section we describe the features of the TCP protocol specified by these RFCs,
the following sections deal with the implementation of the TCP in the INET framework.

\subsubsection{TCP segments}

The TCP module transmits a stream of the data over the unreliable, datagram service
that the IP layer provides. When the application writes a chunk of data into the socket,
the TCP module breaks it down to packets and hands it over the IP. On the receiver side,
it collects the recieved packets, order them, and acknowledges the reception. The packets
that are not acknowledged in time are retransmitted by the sender.

The TCP procotol can address each byte of the data stream by \emph{sequence numbers}.
The sequence number is a 32-bit unsigned integer, if the end of its range is reached,
it is wrapped around.

The layout of the TCP segments is described in RFC793:

\begin{center}
\begin{bytefield}{32}
\bitheader{0,3,4,7,8,15,16,31} \\
\bitbox{16}{Source Port} &
\bitbox{16}{Destination Port} \\
\bitbox{32}{Sequence Number} \\
\bitbox{32}{Acknowledgment Number} \\
\bitbox{4}{\small Data Offset} &
\bitbox{6}{Reserved} &
\bitbox{6}{Flags} &
\bitbox{16}{Window} \\
\bitbox{16}{Checksum} &
\bitbox{16}{Urgent Pointer} \\
\bitbox{24}{Options} &
\bitbox{8}{Padding} \\
\wordbox{3}{Data}
\end{bytefield}
\end{center}

Here
\begin{itemize}
  \item the Source and Destination Ports, together with the Source and Destination
  addresses of the IP header identifies the communication endpoints.
  \item the Sequence Number identifier of the first data byte transmitted in the sequence,
  Sequence Number + 1 identifies the second byte, so on. If the SYN flag is set it consumes
  one sequence number before the data bytes.
  \item the Acknowlegment Number refers to the next byte (if the ACK flag is set) expected
  by the receiver using its sequence number
  \item the Data Offset is the length of the TCP header in 32-bit words (needed because the
  Options field has variable length)
  \item the Reserved bits are unused
  \item the Flags field composed of 6 bits:
  \begin{itemize}
    \item URG: Urgent Pointer field is significant
    \item ACK: Acknowledgment field is significant
    \item PSH: Push Function
    \item RST: Reset the connection
    \item SYN: Synchronize sequence number
    \item FIN: No more data from sender
  \end{itemize}
  \item the Window is the number of bytes the receiver TCP can accept (because of its
  limited buffer)
  \item the Checksum is the 1-complement sum of the 16-bit words of the IP/TCP header and
  data bytes
  \item the Urgent Pointer is the offset of the urgent data (if URG flag is set)
  \item the Options field is variable length, it can occupy 0-40 bytes in the header and is
  always padded to a multiple of 4 bytes.
\end{itemize}

\subsubsection{TCP connections}

When two applications are communicating via TCP, one of the applications is the client,
the other is the server. The server usually starts a socket with a well known local port
and waits until a request comes from clients. The client applications are issue connection
requests to the port and address of the service they want to use.

After the connection is established both the client and the server can send and receive data.
When no more data is to be sent, the application closes the socket. The application can still
receive data from the other direction. The connection is closed when both communication partner
closed its socket.

...

When opening the connection an initial sequence number is choosen and communicated to the
other TCP in the SYN segment. This sequence number can not be a constant value (e.g. 0),
because then data segments from a previous incarnation of the connection (i.e. a connection
with same addresses and ports) could be erronously accepted in this connection. Therefore
most TCP implementation choose the initial sequence number according to the system clock.


\begin{figure}
\includegraphics[width=\textwidth]{figures/tcpstate}
\caption{TCP state diagram}
\label{fig:tcp_states}
\end{figure}

\subsubsection{Flow control}
\label{subsec:flow_control}

The TCP module of the receiver buffers the data of incoming segments.
This buffer has a limited capacity, so it is desirable to notify the sender
about how much data the client can accept. The sender stops the transmission
if this space exhausted.

In TCP every ACK segment holds a Window field; this is the available space
in the receiver buffer. When the sender reads the Window, it can send at most
Window unacknowledged bytes.

\subsubsection*{Window Scale option}

% RFC1323
The TCP segment contains a 16-bit field for the Window, thus allowing at most
65535 byte windows. If the network bandwidth and latency is large, it is surely
too small. The sender should be able to send bandwitdh*latency bytes without
receiving ACKs.

For this purpose the Window Scale (WS) option had been introduced in RFC1323.
This option specifies a scale factor used to interpret the value of the Window field.
The format is the option is:

\begin{center}
\begin{bytefield}{24}
\bitbox{8}{Kind=3} &
\bitbox{8}{Length=3} &
\bitbox{8}{shift.cnt}
\end{bytefield}
\end{center}

If the TCP want to enable window sizes greater than 65535, it should send
a WS option in the SYN segment or SYN/ACK segment (if received a SYN with WS
option). Both sides must send the option in the SYN segment to enable window scaling,
but the scale in one direction might differ from the scale in the other direction.
The $shift.cnt$ field is the 2-base logarithm of the window scale of the sender.
Valid values of $shift.cnt$ are in the $[0,14]$ range.

\subsubsection*{Persistence timer}

When the reciever buffer is full, it sends a 0 length window in the ACK segment
to stop the sender. Later if the application reads the data,
it will repeat the last ACK with an updated window to resume data sending.
If this ACK segment is lost, then the sender is not notified, so a deadlock
happens.

To avoid this situation the sender starts a Persistence Timer when it received
a 0 size window. If the timer expires before the window is increased it send
a probe segment with 1 byte of data. It will receive the current window of the
receiver in the response to this segment.

\subsubsection*{Keepalive timer}

TCP keepalive timer is used to detect dead connections.

\subsubsection{Transmission policies}
\label{subsec:trans_policies}

\subsubsection*{Retransmissions}

% source: RFC1222 4.3.2.1 and Tannenbaum 6.5.10

When the sender TCP sends a TCP segment it starts a
retransmission timer.
If the ACK arrives before the timer expires it is stopped,
otherwise it triggers a retransmission of the segment.

If the retransmission timeout (RTO) is too high, then lost segments
causes high delays, if it is too low, then the receiver gets
too many useless duplicated segments. For optimal behaviour, the
timeout must be dynamically determined.

Jacobson suggested to measure the RTT mean and deviation
and apply the timeout:

$$ RTO = RTT + 4 * D $$

Here RTT and D are the measured smoothed roundtrip time and its
smoothed mean deviation. They are initialized to 0 and updated each time an
ACK segment received according to the following formulas:

$$ RTT = \alpha*RTT + (1-\alpha) * M $$

$$ D = \alpha*D + (1-\alpha)*|RTT-M| $$

where $M$ is the time between the segments send and the acknowledgment
arrival. Here the $\alpha$ smoothing factor is typically $7/8$.

One problem may occur when computing the round trip: if the
retransmission timer timed out and the segment is sent again,
then it is unclear that the received ACK is a response to the
first transmission or to the second one. To avoid confusing the
RTT calculation, the segments that have been retransmitted
do not update the RTT. This is known as Karn's modification.
He also suggested to double the $RTO$ on each failure until the
segments gets through (``exponential backoff'').

\subsubsection*{Delayed ACK algorithm}

% RFC1122 4.2.3.2

A host that is receiving a stream of TCP data segments can
increase efficiency in both the Internet and the hosts by
sending fewer than one ACK (acknowledgment) segment per data
segment received; this is known as a "delayed ACK" [TCP:5].

Delay is max. 500ms.

A delayed ACK gives the application an opportunity to
update the window and perhaps to send an immediate
response.  In particular, in the case of character-mode
remote login, a delayed ACK can reduce the number of
segments sent by the server by a factor of 3 (ACK,
window update, and echo character all combined in one
segment).

In addition, on some large multi-user hosts, a delayed
ACK can substantially reduce protocol processing
overhead by reducing the total number of packets to be
processed [TCP:5].  However, excessive delays on ACK's
can disturb the round-trip timing and packet "clocking"
algorithms [TCP:7].

% RFC2581 3.2

a TCP receiver SHOULD send an immediate ACK
when the incoming segment fills in all or part of a gap in the
sequence space.

\subsubsection*{Nagle's algorithm}

RFC896 describes the ``small packet problem": when the application
sends single-byte messages to the TCP, and it transmitted immediatly
in a 41 byte TCP/IP packet (20 bytes IP header, 20 bytes TCP header,
1 byte payload), the result is a 4000\% overhead that can cause
congestion in the network.

The solution to this problem is to delay the transmission until
enough data received from the application and send all collected
data in one packet. Nagle proposed that
when a TCP connection has outstanding data that has not
yet been acknowledged, small segments should not be sent
until the outstanding data is acknowledged.

\subsubsection*{Silly window avoidance}

The Silly Window Syndrome (SWS) is described in RFC813. It occurs when
a TCP receiver advertises a small window and the TCP sender immediately
sends data to fill the window. Let's take the example when the sender
process writes a file into the TCP stream in big chunks, while the
receiver process reads the bytes one by one. The first few bytes
are transmitted as whole segments until the receiver buffer
becomes full. Then the application reads one
byte, and a window size 1 is offered to the sender. The sender sends
a segment with 1 byte payload immediately, the receiver buffer becomes
full, and after reading 1 byte, the offered window is 1 byte again.
Thus almost the whole file is transmitted in very small segments.

In order to avoid SWS, both sender and receiver must try to avoid this
situation. The receiver must not advertise small windows and the sender
must not send small segments when only a small window is advertised.

In RFC813 it is offered that
\begin{enumerate}
  \item the receiver should not advertise windows that is smaller than the maximum
        segment size of the connection
  \item the sender should wait until the window is large enough for a maximum sized
        segment.
\end{enumerate}

\subsubsection*{Timestamp option}

Efficient retransmissions depends on precious RTT measurements.
Packet losses can reduce the precision of these measurements radically.
When a segment lost, the ACKs received in that window can not be used;
thus reducing the sample rate to one RTT data per window. This is
unacceptable if the window is large.

The proposed solution to the problem is to use a separate timestamp
field to connect the request and the response on the sender side.
The timestamp is transmitted as a TCP option. The option contains two
32-bit timestamps:

\begin{center}
\begin{bytefield}{80}
\bitbox{8}{Kind=5} &
\bitbox{8}{Length=10} &
\bitbox{32}{TS Value} &
\bitbox{32}{TS Echo Reply} &
\end{bytefield}
\end{center}

Here the TS Value (TSVal) field is the current value of the timestamp
clock of the TCP sending the option, TS Echo Reply (TSecr) field is
0 or echoes the timestamp value of that was sent by the remote TCP.
The TSscr field is valid only in ACK segments that acknowledges new
data. Both parties should send the TS option in their SYN segment
in order to allow the TS option in data segments.

The timestamp option can also be used for PAWS (protection against wrapped
sequence numbers).


\subsubsection{Congestion control}

Flow control allows the sender to slow down the transmission when the
receiver can not accept them because of memory limitations. However
there are other situations when a slow down is desirable. If the sender
transmits a lot of data into the network it can overload the processing
capacities of the network nodes, so packets are lost in the network
layer.

For this purpose another window is maintained at the sender side, the
congestion window (CWND). The congestion window is a sender-side limit
on the amount of data the sender can transmit into the network before
receiving ACK. More precisely, the sender can send at most max(CWND, WND)
bytes above SND.UNA, therefore $ SND.NXT < SND.UNA + max(CWND, WND) $ is
guaranteed.

The size of the congestion window is dinamically determined by monitoring
the state of the network.

% RFC2581
%
% Definitions:
% SMSS: sender maximum segment size
% RMSS: receiver maximum segment size (default 536)
% rwnd: most recently advertised receiver window
% IW: initial sender's congestion window
% LW: loss window, size of congestion window after a TCP sender detects loss
% RW: restart window, size of congestion window after a TCP restarts transmission after an idle period
% fligth size: amount of data has been sent but not yet acknowledged
% cwnd: congestion window, sender-size limit on the amount of data the sender
%       can transmit into the network before receiving an ACK
% rwnd: receiver advertised window, receiver-side limit on the amount of outstanding data
% sstresh: whether slow start or congestion avoidance used
%
% IW <= 2*MSS


\subsubsection*{Slow Start and Congestion Avoidance}

There are two algorithm that updates the congestion window, ``Slow Start''
and ``Congestion Avoidance''. They are specified in RFC2581.

\begin{pseudocode}
$cwnd \gets 2*SMSS$
$ssthresh \gets $ upper bound of the window (e.g. $65536$)
whenever ACK received
  if $cwnd < ssthresh$
    $cwnd \gets cwnd + SMSS$
  otherwise
    $cwnd \gets cwnd + SMSS*SMSS/cwnd$
whenever packet loss detected
  $cwnd \gets SMSS$
  $ssthresh \gets max(FlightSize/2, 2*SMSS)$
\end{pseudocode}

Slow Start means that when the connection opened the sender initially
sends the data with a low rate. This means that the initial
window (IW) is at most 2 MSS, but no more than 2 segments. If there was no packet loss,
then the congestion window is increased rapidly, it is doubled in each flight.
When a packet loss is detected, the congestion window is reset to 1 MSS (loss window, LW)
and the ``Slow Start'' is applied again.

\begin{note}
RFC3390 increased the IW to roughly 4K bytes: $min(4*MSS, max(2*MSS, 4380))$.
\end{note}

When the congestion window reaches a certain limit (slow start threshold),
the ``Congestion Avoidance'' algorithm is applied. During ``Congestion Avoidance''
the window is incremented by 1 MSS per round-trip-time (RTT). This is usually
implemented by updating the window according to the $ cwnd += SMSS*SMSS/cwnd $
formula on every non-duplicate ACK.

The Slow Start Threshold is updated when a packet loss is detected.
It is set to $max(FlightSize/2, 2*SMSS)$.

How the sender estimates the flight size? The data sent, but not yet acknowledged.

How the sender detect packet loss? Retransmission timer expired.


\subsubsection*{Fast Retransmit and Fast Recovery}

RFC2581 specifies two additional methods to increase the efficiency
of congestion control: ``Fast Retransmit'' and ``Fast Recovery''.

``Fast Retransmit'' requires that the receiver signal the event,
when an out-of-order segment arrives. It is achieved by sending
an immediate duplicate ACK. The receiver also sends an immediate
ACK when the incoming segment fills in a gap or part of a gap.

When the sender receives the duplicated ACK it knows that some
segment after that sequence number is received out-of-order or
that the network duplicated the ACK. If 3 duplicated ACK received
then it is more likely that a segment was dropped or delayed.
In this case the sender starts to retransmit the segments
immediately.

``Fast Recovery'' means that ``Slow Start'' is not applied
when the loss is detected as 3 duplicate ACKs. The arrival
of the duplicate ACKs indicates that the network is not fully
congested, segments after the lost segment arrived, as well
the ACKs.

% Details?
%
%    1.  When the third duplicate ACK is received, set ssthresh to no more
%        than the value given in equation 3.
%
%    2.  Retransmit the lost segment and set cwnd to ssthresh plus 3*SMSS.
%        This artificially "inflates" the congestion window by the number
%        of segments (three) that have left the network and which the
%        receiver has buffered.
%
%    3.  For each additional duplicate ACK received, increment cwnd by
%        SMSS.  This artificially inflates the congestion window in order
%        to reflect the additional segment that has left the network.
%
%    4.  Transmit a segment, if allowed by the new value of cwnd and the
%        receiver's advertised window.
%
%    5.  When the next ACK arrives that acknowledges new data, set cwnd to
%        ssthresh (the value set in step 1).  This is termed "deflating"
%        the window.
%
%        This ACK should be the acknowledgment elicited by the
%        retransmission from step 1, one RTT after the retransmission
%        (though it may arrive sooner in the presence of significant out-
%        of-order delivery of data segments at the receiver).
%        Additionally, this ACK should acknowledge all the intermediate
%        segments sent between the lost segment and the receipt of the
%        third duplicate ACK, if none of these were lost.

\subsubsection*{Loss Recovery Using Limited Transmit}

If there is not enough data to be send after a lost segment,
then the Fast Retransmit algorithm is not activated, but the
costly retranmission timeout used.

RFC3042 suggests that the sender TCP should send a new data segment
in response to each of the first two duplicate acknowledgement. Transmitting
these segments increases the probability that TCP can recover from a single
lost segment using the fast retransmit algorithm, rather than using a costly
retransmission timeout.

\subsubsection*{Selective Acknowledgments}

% RFC2018

With selective
acknowledgments (SACK), the data receiver can inform the sender about all
segments that have arrived successfully, so the sender need
retransmit only the segments that have actually been lost.

With the help of this information the sender can detect
\begin{itemize}
  \item replication by the network
  \item false retransmit due to reordering
  \item retransmit timeout due to ACK loss
  \item early retransmit timeout
\end{itemize}


In the congestion control algorithms described so far
the sender has only rudimentary information about which
segments arrived at the receiver. On the other hand
the algorithms are implemented completely on the sender side,
they only require that the client sends immediate ACKs on
duplicate segments. Therefore they can work in a heterogenous
environment, e.g. a client with Tahoe TCP can communicate with
a NewReno server. On the other hand SACK must be supported by
both endpoint of the connection to be used.

If a TCP supports SACK it includes the \emph{SACK-Permitted} option
in the SYN/SYN-ACK segment when initiating the connection.
The SACK extension enabled for the connection if the \emph{SACK-Permitted}
option was sent and received by both ends. The option occupies
2 octets in the TCP header:

\begin{center}
\begin{bytefield}{16}
\bitbox{8}{Kind=4} &
\bitbox{8}{Length=2}
\end{bytefield}
\end{center}

If the SACK is enabled then the data receiver adds SACK option
to the ACK segments. The SACK option informs the sender about
non-contiguous blocks of data that have been received and queued.
The meaning of the \emph{Acknowledgement Number} is unchanged,
it is still the cumulative sequence number. Octets received
before the \emph{Acknowledgement Number} are kept by the receiver,
and can be deleted from the sender's buffer. However the receiver
is allowed to drop the segments that was only reported in the SACK
option.

The \emph{SACK} option contains the following fields:

\begin{center}
\begin{bytefield}{32}
\bitbox[]{16}{} &
\bitbox{8}{Kind=5} &
\bitbox{8}{Length} \\
\bitbox{32}{Left Edge of 1st Block} \\
\bitbox{32}{Right Edge of 1st Block} \\
\wordbox[]{1}{$\vdots$ \\[1ex]} \\
\bitbox{32}{Left Edge of nth Block} \\
\bitbox{32}{Right Edge of nth Block}
\end{bytefield}
\end{center}

Each block represents received bytes of data that are
contiguous and isolated with one exception: if a segment
received that was already ACKed (i.e. below $RCV.NXT$),
it is included as the first block of the \emph{SACK} option.
The purpose is to inform the sender about a spurious retransmission.

Each block in the option occupies 8 octets. The TCP header
allows 40 bytes for options, so at most 4 blocks can be
reported in the \emph{SACK} option (or 3 if TS option is also used).
The first block is used for reporting the most recently received
data, the following blocks repeats the most recently reported
SACK blocks. This way each segment is reported at least 3 times,
so the sender receives the information even if some ACK segment is
lost.


\textbf{SACK based loss recovery}

% RFC3517: loss recovery based on SACK

Now lets see how the sender can use the information in the
\emph{SACK} option. First notice that it can give a better
estimation of the amount of data outstanding in the network
(called $pipe$ in RFC3517).
If $highACK$ is the highest ACKed sequence number, and
$highData$ of the highest sequence number transmitted,
then the bytes between $highACK$ and $highData$ can be
in the network. However $ pipe \neq highData - highACK $
if there are lost and retransmitted segments:

$$ pipe = highData - highACK - lostBytes + retransmittedBytes $$

A segment is supposed to be lost if it was not received
but 3 segments recevied that comes after this segment in the sequence
number space.
This condition is detected by the sender by receiving
either 3 discontiguous SACKed blocks, or at least
$3*SMSS$ SACKed bytes above the sequence numbers of the
lost segment.

The transmission of data starts with a \emph{Slow Start} phase.
If the loss is detected by 3 duplicate ACK, the sender
goes into the recovery state: it sets
$cwnd$ and $ssthresh$ to $FlightSize / 2$.
It also remembers the $highData$ variable, because
the recovery state is left when this sequence number
is acknowledged.

In the recovery state it sends data
until there is space in the congestion window (i.e. $cwnd-pipe >= 1 SMSS$)
The data of the segment is choosen by the following rules (first rule that applies):

\begin{enumerate}
  \item send segments that is lost and not yet retransmitted
  \item send segments that is not yet transmitted
  \item send segments that is not yet retransmitted and possibly fills a gap
        (there is SACKed data above it)
\end{enumerate}

If there is no data to send, then the sender waits for the next ACK, updates
its variables based on the data of the received ACK, and then try to transmit
according to the above rules.

If an RTO occurs, the sender drops the collected SACK information and
initiates a Slow Start. This is to avoid a deadlock when the receiver
dropped a previously SACKed segment.

% highACK: highest ACKed sequence number
%
% highData: highest sequence number transmitted
%
% highRxt: highest sequence number which has been retransmitted
%
%
% Normal phase: before the first loss, until 3 duplicate ACK
%
% Loss recovery phase: until ACK for RecoveryPoint received
%
% On the transition to loss recovery phase
% \begin{enumerate}
%   \item RecoveryPoint=HighData
%   \item ssthresh=cwnd=FlightSize/2
%   \item compute \emph{pipe}
% \end{enumerate}
%
% In the loss recovery phase, for each incoming ACK:
%
% \begin{enumerate}
%   %\Alph
%   \item if cumulative ACK above RecoveryPoint, leave loss recovery phase
%   \item update SACK info and compute pipe
%   \item if $cwnd-pipe >= 1 SMSS$ send one or more segments (if there is data to send)
%   \item update HighRxt, HighData according to the sent bytes
%   \item increment $pipe$ by the amount of data sent
%   \item if $cwnd-pipe >= 1 SMSS$, continue sending
% \end{enumerate}
%
% Which bytes to be send are determined as follows:
%
% \begin{enumerate}
%   \item if there is a byte which is lost and not yet retransmitted, send that in 1 segment
%   \item otherwise if there is unsent data, send that in 1 segment
%   \item otherwise if there is not yet retransmitted data, and above that there is SACKed data, send that
%   \item otherwise there is no data to send
% \end{enumerate}


\subsection{The TCP module}
\label{sec:tcp_module}

The \nedtype{Tcp} simple module is the main implementation of the TCP protocol in the INET framework.
The \nedtype{Tcp} module as other transport protocols work above the network layer and below the application
layer, therefore it has gates to be connected with the IPv4 or IPv6 network (ipIn/ipOut or ipv6In/ipv6Out),
and with the applications (appIn[k], appOut[k]).
One \nedtype{Tcp} module can serve several application modules, and several
connections per application. The $k$th application connects to \nedtype{Tcp}'s
\ttt{appIn[k]} and \ttt{appOut[k]} ports.

The TCP module usually specified by its module interface
(\nedtype{ITcp}) in the NED definition of hosts, so it can be replaced with any implementation
that communicates through the same gates.

\subsubsection{Statistics}

The \nedtype{Tcp} module collects the following vectors:

\begin{tabular}{l p{10cm}}
  \ttt{send window} & $SND.WND$ \\
  \ttt{receive window} & $RCV.WND$, after SWS avoidance applied \\
  \ttt{advertised window} & $RCV.NXT + RCV.WND$ \\
  \ttt{sent seq} & \emph{Sequence Number} of the sent segment \\
  \ttt{sent ack} & \emph{Acknowledgement Number} of the sent segment \\
  \ttt{rcvd seq} & \emph{Sequence Number} of the received segment \\
  \ttt{rcvd ack} & \emph{Acknowledgement Number} of the received segment \\
  \ttt{unacked bytes} & number of sent and unacknowledged bytes ($max of SND.NXT - SND.UNA$) \\
  \ttt{rcvd dupAcks} & number of duplicate acknowledgements, reset to 0 when $SND.UNA$ advances \\
  \ttt{pipe} & the value of the SACK $pipe$ variable
               (estimated number of bytes outstanding in the network) \\
  \ttt{sent sacks} & number of SACK blocks sent \\
  \ttt{rcvd sacks} & number of SACK blocks received \\
  \ttt{rcvd oooseg} & number of received out-of-order segments \\
  \ttt{rcvd naseg} & number of received unacceptable segments (outside the receive window) \\
  \ttt{rcvd sackedBytes} & total amount of SACKed bytes in the buffer of the sender \\
  \ttt{tcpRcvQueueBytes} & number of bytes in the receiver's buffer \\
  \ttt{tcpRcvQueueDrops} & number of bytes dropped by the receiver (not enough buffer) \\
  \ttt{cwnd} & congestion window \\
  \ttt{ssthresh} & slow start threshold \\
  \ttt{measured RTT} & measured round trip time \\
  \ttt{smoothed RTT} & smoothed round trip time \\
  \ttt{RTTVAR} & measured smoothed variance of round trip time \\
  \ttt{RTO} & retransmission timeout \\
  \ttt{numRTOs} & number of retransmission timeouts occured \\
\end{tabular}

If the \fpar{recordStats} parameter is set to \fkeyword{false}, then none
of these output vectors are generated.

% \subsection{Animation effects}
%
% TCP module text: number of connections sorted by state
%
% log, log verbose

\subsection{TCP lwIP}
\label{sec:tcp_lwip}

lwIP is a light-weight implementation of the TCP/IP protocol suite
that was originally written by Adam Dunkels of the Swedish Institute of
Computer Science. The current development homepage is
\url{http://savannah.nongnu.org/projects/lwip/}.

The implementation targets embedded devices: it has
very limited resource usage (it works ``with tens of kilobytes of RAM and
around 40 kilobytes of ROM'') and does not require an underlying OS.

The \nedtype{TCP\_lwIP} simple module is based on the 1.3.2 version of
the lwIP sources.

Features:

\begin{compactitem}
\item delayed ACK
\item Nagle's algorithm
\item round trip time estimation
\item adaptive retransmission timeout
\item SWS avoidance
\item slow start threshold
\item fast retransmit
\item fast recovery
\item persist timer
\item keep-alive timer
\end{compactitem}

\subsubsection*{Limitations}

\begin{itemize}
  \item only MSS and TS TCP options are supported. The TS option is turned off
        by default, but can be enabled by defining LWIP\_TCP\_TIMESTAMPS to 1
        in \ffilename{lwipopts.h}.
  \item \fvar{fork} must be \fkeyword{true} in the passive open command
  \item The status request command (TCP\_C\_STATUS) only reports the
          local and remote addresses/ports of the connection and
          the MSS, SND.NXT, SND.WND, SND.WL1, SND.WL2, RCV.NXT, RCV.WND variables.
\end{itemize}

% lwIP license file missing from INET source
% FIXME TCP_lwIP uses only connId to identify the connection instead of (connId,appGateIndex)
% FIXME status command returns message kind TCP_C_STATUS instead of TCP_I_STATUS

\subsubsection*{Statistics}

The \nedtype{TCP\_lwIP} module generates the following vector files:

\begin{itemize}
  \item \ttt{send window}: value of the $SND.WND$ variable
  \item \ttt{sent seq}: \emph{Sequence Number} of the sent segment
  \item \ttt{sent ack}: \emph{Acknowledgment Number } of the sent segment
  \item \ttt{receive window}: value of the $RCV.WND$ variable
  \item \ttt{rcvd seq}: \emph{Sequence Number} of the received segment
  \item \ttt{rcvd acq}: \emph{Acknowledgment Number} of the received segment
\end{itemize}

% FIXME the following vectors are created, but not used (copy paste?):
%       'sent sacks', 'advertised window', 'rcvd sacks', 'unacked bytes',
%       'rcvd dupAcks', 'pipe', 'rcvd oooseg', 'rcvd sackedBytes',
%       'tcpRcvQueueBytes', 'tcpRcvQueueDrops'
The creation of these vectors can be disabled by setting the \fpar{recordStats}
parameter to \fkeyword{false}.


\subsection{TCP NSC}
\label{sec:tcp_nsc}

Network Simulation Cradle (NSC) is a tool that allow real-world TCP/IP network stacks
to be used in simulated networks. The NSC project is created by Sam Jansen
and available on \url{http://research.wand.net.nz/software/nsc.php}. NSC currently
contains Linux, FreeBSD, OpenBSD and lwIP network stacks, although on 64-bit
systems only Linux implementations can be built.

To use the \nedtype{TCP\_NSC} module you should download the
\ffilename{nsc-0.5.2.tar.bz2} package and follow the instructions
in the \ffilename{<inet\_root>/3rdparty/README} file to build it.

\begin{warning}
Before generating the INET module, check that the \emph{opp\_makemake} call
in the make file (\ffilename{<inet\_root>/Makefile}) includes the
\emph{-DWITH\_TCP\_NSC} argument. Without this option the \nedtype{TCP\_NSC}
module is not built. If you build the INET library from the IDE, it is enough
to enable the \emph{TCP (NSC)} project feature.
\end{warning}

\subsubsection*{Parameters}

The \nedtype{TCP\_NSC} module has the following parameters:

\begin{itemize}
  \item \fpar{stackName}: the name of the TCP implementation to be used.
  Possible values are: \ttt{liblinux2.6.10.so}, \ttt{liblinux2.6.18.so},
  \ttt{liblinux2.6.26.so}, \ttt{libopenbsd3.5.so}, \ttt{libfreebsd5.3.so} and
  \ttt{liblwip.so}. (On the 64 bit systems, the \ttt{liblinux2.6.26.so} and
  \ttt{liblinux2.6.16.so} are available only).

  \item \fpar{stackBufferSize}: the size of the receive and send buffer of
  one connection for selected TCP implementation.
  The NSC sets the \fvar{wmem\_max}, \fvar{rmem\_max}, \fvar{tcp\_rmem}, \fvar{tcp\_wmem}
  parameters to this value on linux TCP implementations. For details, you can see
  the NSC documentation.
\end{itemize}

\subsubsection*{Statistics}

The \nedtype{TCP\_NSC} module collects the following vectors:

\begin{itemize}
  \item \ttt{sent seq} \emph{Sequence Number} of the sent TCP segment
  \item \ttt{sent ack} \emph{Acknowledgment Number} of the sent TCP segment
  \item \ttt{rcvd seq} \emph{Sequence Number} of the received TCP segment
  \item \ttt{rcvd ack} \emph{Acknowledgement Number} of the received TCP segment
\end{itemize}

\subsubsection*{Limitations}

\begin{itemize}
\item Because the kernel code is not reentrant, NSC creates a record containing
the global variables of the stack implementation. By default there is room
for 50 instance in this table, so you can not create more then 50 instance
of \nedtype{TCP\_NSC}. You can increase the \fvar{NUM\_STACKS} constant
in \ffilename{num\_stacks.h} and recompile NSC to overcome this limitation.

\item The \nedtype{TCP\_NSC} module does not supprt TCP\_TRANSFER\_OBJECT
data transfer mode.

\item The MTU of the network stack fixed to 1500, therefore MSS is 1460.

\item TCP\_C\_STATUS command reports only local/remote addresses/ports and
      current window of the connection.

\end{itemize}

% local address: 1.0.0.253, gateway address: 1.0.0.254, remote addresses: 2.0.0.1, 2.0.0.2, ...

% FIXME connections are identified by connId, not by (appGateIndex,connId) as in TCP module.
% FIXME TCP_NSC_Connection::getSocket() and TCP_NSC_Connection::do_checkedclose() are declared but not implemented




\section{UDP}
\label{sec:udp}

\subsection{Overview}

The UDP protocol is a very simple datagram transport protocol, which
basically makes the services of the network layer available to the applications.
It performs packet multiplexing and demultiplexing to ports and some basic
error detection only.

The frame format as described in RFC768:

\begin{center}
\begin{bytefield}{32}
\bitheader{0,7,8,15,16,23,24,31} \\
\bitbox{16}{Source Port} &
\bitbox{16}{Destination Port} \\
\bitbox{16}{Length} &
\bitbox{16}{Checksum} \\
\wordbox{3}{Data}
\end{bytefield}
\end{center}

The ports represents the communication end points that are allocated by the
applications that want to send or receive the datagrams. The ``Data'' field
is the encapsulated application data, the ``Length'' and ``Checksum'' fields
are computed from the data.

The INET framework contains an \nedtype{Udp} module that performs the encapsulation/decapsulation
of user packets, an \nedtype{UdpSocket} class that provides the application the usual
socket interface, and several sample applications.

These components implement the following statndards:
\begin{itemize}
\item RFC768: User Datagram Protocol
\item RFC1122: Requirements for Internet Hosts -- Communication Layers
\end{itemize}

\subsection{The UDP module}

The UDP protocol is implemented by the \nedtype{Udp} simple module.
There is a module interface (\nedtype{IUdp}) that defines the gates of the
\nedtype{Udp} component. In the \nedtype{StandardHost} node, the UDP component
can be any module implementing that interface.

Each UDP module has gates to connect to the IPv4 and IPv6 network layer
(ipIn/ipOut and ipv6In/ipv6Out), and a gate array to connect to the applications
(appIn/appOut).

The UDP module can be connected to several applications, and each application
can use several sockets to send and receive UDP datagrams.



\section{SCTP}
\label{sec:sctp}

TODO

\section{RTP}
\label{sec:rtp}

TODO


%%% Local Variables:
%%% mode: latex
%%% TeX-master: "usman"
%%% End:


\cleardoublepage

\include{ch-ipv4}
\cleardoublepage

\include{ch-ipv6}
\cleardoublepage

\chapter{Other Network Protocols}
\label{cha:other-network-protocols}

TODO how to choose addressing scheme?  every addressing scheme works with every protocol?  

Ipv4NetworkLayer
Ipv6NetworkLayer
WiseRouteNetworkLayer = WiseRoute + GenericArp

SimpleNetworkLayer like INetworkLayer

contains:
  np: <networkProtocolType> like INetworkProtocol,
  echo: EchoProtocol

generic: <networkLayerType> like INetworkLayer if networkLayerType != ""

set:

**.hasIpv4 = false
**.hasIpv6 = false
**.hasGn = true  <================= unused in NetworkLayerNodeBase, but referenced in TransportLayerNodeBase
**.networkLayerType = ``WiseRouteNetworkLayer''

see networklayer.ini

**.networkConfiguratorType = "Ipv4NetworkConfigurator"   ???

\section{GenericArp, GlobalArp}

TODO

\section{Flood}

\nedtype{Flood} is a simple flooding protocol for network-level broadcast.
It remembers already broadcasted messages, and does not rebroadcast 
them if it gets another copy of that message.

TODO like INetworkProtocol

\section{ProbabilisticBroadcast}

\nedtype{ProbabilisticBroadcast} is a multi-hop ad-hoc data dissemination 
protocol based on probabilistic broadcast.

This method reduces the number of packets sent on the channel (reducing the
broadcast storm problem) at the risk of some nodes not receiving the data.
It is particularly interesting for mobile networks.

The transmission probability for each attempt, the time between two transmission
attempts, the maximum number of broadcast transmissions of a packet, and
some other settings are parameters.

TODO like INetworkProtocol

\nedtype{AdaptiveProbabilisticBroadcast} is a version that automatically 
adapts transmission probabilities depending on the estimated number of 
neighbours.

\section{WiseRoute}

\nedtype{Wiseroute} is a simple loop-free routing algorithm that
builds a routing tree from a central network point, designed
for sensor networks and convergecast traffic.

The sink (the device at the center of the network) broadcasts
a route building message. Each network node that receives it
selects the sink as parent in the routing tree, and rebroadcasts
the route building message. This procedure maximizes the probability
that all network nodes can join the network, and avoids loops.
Parameter sinkAddress gives the sink network address,
rssiThreshold is a threshold to avoid using bad links (with too low
RSSI values) for routing, and routeFloodsInterval should be set to
zero for all nodes except the sink. Each routeFloodsInterval, the
sink restarts the tree building procedure. Set it to a large value
if you do not want the tree to be rebuilt.

TODO like INetworkProtocol

WiseRouteNetworkLayer

\section{GenericNetworkProtocol}

This module is a simplified generic network protocol that routes
generic datagrams using different kind of network addresses. 

TODO GenericNetworkLayer like INetworkLayer  is a compound module

routingTable: GenericRoutingTable,
gnp: GenericNetworkProtocol,
echo: EchoProtocol,
arp: GenericArp


\section{/////////////////////////////////////////////////////////}


\section{InternetCloud}

This module is an IPv4 router that can delay or drop packets (while retaining
their order) based on which interface card the packet arrived on and 
on which interface It is leaving the cloud. The delayer module is replacable.

By default the delayer module is ~MatrixCloudDelayer which lets you configure
the delay, drop and datarate parameters in an XML file. Packet flows, as defined
by incoming and outgoing interface pairs, are independent of each other.

The ~InternetCloud module can be used only to model the delay between two hops, but
it is possible to build more complex networks using several ~InternetCloud modules.

\section{PIM}

Protocol Independent Multicast -- not a network protocol 

models: \nedtype{PimSm}, \nedtype{PimDm}; \nedtype{Pim} is a compound module



%%% Local Variables:
%%% mode: latex
%%% TeX-master: "usman"
%%% End:


\cleardoublepage

\chapter{Internet Routing}
\label{cha:routing}

\section{Overview}

INET Framework has models for several internet routing protocols, including
RIP, OSPF and BGP.

The easiest way to add routing to a network is to use the \nedtype{Router}
NED type for routers. \nedtype{Router} contains a conditional instance
for each of the above protocols. These submodules can be enabled by
setting the \ttt{hasRIP}, \ttt{hasOSPF} and/or \ttt{hasBGP} parameters to
\ttt{true}.

Example:

\begin{verbatim}
**.hasRIP = true
\end{verbatim}

There are also NED types called \nedtype{RipRouter}, \nedtype{OspfRouter},
\nedtype{BgpRouter}, which are all \nedtype{Router}s with appropriate
routing protocol enabled.

\section{RIP}
\label{sec:rip}

RIP (Routing Information Protocol) is a distance-vector routing protocol
which employs the hop count as a routing metric. RIP prevents routing loops
by implementing a limit on the number of hops allowed in a path from source
to destination.

The \nedtype{Rip} module implements distance vector routing as
specified in RFC 2453 (RIPv2) and RFC 2080 (RIPng). Configuration
can be specified in an XML file that can be specified in the
\ttt{ripConfig} parameter.

The configuration file specifies the per interface parameters.
Each \ttt{<interface>} element configures one or more interfaces;
the \ttt{hosts}, \ttt{names}, \ttt{towards}, \ttt{among} attributes
select the configured interfaces (in a similar way as with
\nedtype{Ipv4NetworkConfigurator} \ref{cha:network-autoconfiguration}).

Additional attributes:
\begin{itemize}
  \item \ttt{metric}: metric assigned to the link, default value is 1.
        This value is added to the metric of a learned route,
        received on this interface. It must be an integer in
        the [1,15] interval.
  \item \ttt{mode}: mode of the interface.
\end{itemize}

The mode attribute can be one of the following:
\begin{itemize}
  \item \ttt{'NoRIP'}: no RIP messages are sent or received on this interface.
  \item \ttt{'NoSplitHorizon'}: no split horizon filtering; send all routes to
        neighbors.
  \item \ttt{'SplitHorizon'}: do not sent routes whose next hop is the neighbor.
  \item \ttt{'SplitHorizonPoisenedReverse'} (default): if the next hop is the neighbor, then
  set the metric of the route to infinity.
\end{itemize}

The following example sets the link metric between router
\ttt{R1} and \ttt{RB} to 2, while all other links will have a metric of 1.
\begin{verbatim}
<RIPConfig>
  <interface among="R1 RB" metric="2"/>
  <interface among="R? R?" metric="1"/>
</RIPConfig>
\end{verbatim}

The \nedtype{Rip} module has the following parameters:
\begin{itemize}
  \item \ttt{mode}: either "RIPv2" (RFC 2453) or "RIPng" (RFC 2080)
  \item \ttt{routingTableModule}: path to the routing table module
        e.g. \ttt{'\^{}.ipv4.routingTable'}
  \item \ttt{ripConfig}: an XML configuration file containing per-interface parameters
\end{itemize}

The following example configures a \nedtype{Router} module to use RIPv2:
\begin{verbatim}
    **.hasRIP = true
    **.mode = "RIPv2"
    **.ripConfig = xmldoc("RIPConfig.xml")
\end{verbatim}

\section{OSPF}
\label{sec:ospf}

OSPF (Open Shortest Path First) is a routing protocol for IP networks.
It uses a link state routing (LSR) algorithm and falls into the group
of interior gateway protocols (IGPs), operating within a single
autonomous system (AS).

The \nedtype{Ospf} module implements the OSPF Version 2. Areas and routers
can be configured using an XML file.

\nedtype{OspfRouter} is a \nedtype{Router} with the OSPF protocol enabled.


\section{BGP}
\label{sec:bgp}

BGP (Border Gateway Protocol) is a standardized exterior gateway protocol
designed to exchange routing and reachability information among
autonomous systems (AS) on the Internet.

The \nedtype{Bgp} module implements BGP Version 4. The model implements
RFC 4271, with some limitations. Autonomous Systems and rules can be
configured in an XML file.

\nedtype{BgpRouter} is a \nedtype{Router} with the BGP protocol enabled.

%%% Local Variables:
%%% mode: latex
%%% TeX-master: "usman"
%%% End:


\cleardoublepage

\include{ch-adhoc-routing}
\cleardoublepage

\include{ch-diffserv}
\cleardoublepage

\chapter{The MPLS Models}
\label{cha:mpls}

% HINT: A good MPLS primer (a 64-slide presentation):
% "MPLS for Dummies", Richard A Steenbergen <ras@nlayer.net>, nLayer Communications, Inc.
% https://www.nanog.org/meetings/nanog49/presentations/Sunday/mpls-nanog49.pdf

TODO rename Rsvp to RsvpTe
TODO rename LdpRouter to LdpMplsRouter (or LdpLsr?), and RsvpRouter to RsvpMplsRouter

\section{Overview}

TODO what is MPLS

INET provides the following components for building MPLS simulations.

The core modules are:

\begin{itemize}
  \item \nedtype{Mpls}, 
  \item \nedtype{LibTable}, 
  \item \nedtype{Ldp}, 
  \item \nedtype{Rsvp}, 
  \item \nedtype{Ted}, 
  \item \nedtype{LinkStateRouting} 
\end{itemize}

MPLS-enabled routers are:

\begin{itemize}
  \item \nedtype{LdpRouter} is an MPLS router with LDP as control protocol
  \item \nedtype{RsvpRouter} is an MPLS router with RSVP-TE as control protocol 
\end{itemize}


\section{The MPLS Module}

% collaborations with other modules, like LibTable.  libTableModule parameter

LibTable stores the LIB (Label Information Base), accessed by ~Mpls and its
associated control protocols (~Rsvp, ~Ldp) via direct C++ method calls.

xml config

TODO

\section{The LDP Module}

% collaborations with other modules

TODO

\section{The RSVP Module}

% collaborations with other modules

TODO

\section{LIB Table File Format}

The format of a LIB table file is:

The beginning of the file should begin with comments. Lines that begin with \# are treated
as comments. An empty line is required after the comments. The "LIB TABLE"
syntax must come next with an empty line. The column headers follow. This header
must be strictly "In-lbl In-intf Out-lbl Out-intf". Column
values are after that with space or tab for field separation.
The following is a sample of lib table file.

\begin{verbatim}
#lib table for MPLS network simulation test
#lib1.table for LSR1 - this is an edge router
#no incoming label for traffic from in-intf 0 &1 - LSR1 is ingress router for those traffic
#no outgoing label for traffic from in_intf 2 &3 - LSR 1 is egress router for those traffic

LIB TABLE:

In-lbl  In-intf         Out-lbl     Out-intf
1       193.233.7.90    1           193.231.7.21
2       193.243.2.1     0           193.243.2.3
\end{verbatim}


\section{The traffic.xml file}

The traffic.xml file is read by the RSVP-TE module (RSVP).
The file must be in the same folder as the executable
network simulation file.

The XML elements used in the "traffic.xml" file:

\begin{itemize}
  \item \ttt{<Traffic></Traffic>} is the root element. It may contain one or more \ttt{<Conn>} elements.
  \item \ttt{<Conn></Conn>} specifies an RSVP session. It may contain the following elements:
  \begin{itemize}
    \item \ttt{<src></src>} specifies sender IP address
    \item \ttt{<dest></dest>} specifies receiver IP address
    \item \ttt{<setupPri></setupPri>} specifies LSP setup priority
    \item \ttt{<holdingPri></holdingPri>} specifies LSP holding priority
    \item \ttt{<bandwidth></bandwidth>} specifies the requested BW.
    \item \ttt{<delay></delay>} specifies the requested delay.
    \item \ttt{<route></route>} specifies the explicit route. This is a comma-separated
      list of IP-address, hop-type pairs (also separated by comma).
      A hop type has a value of 1 if the hop is a loose hop and 0 otherwise.
  \end{itemize}
\end{itemize}

The following presents an example file:

\begin{verbatim}
<?xml version="1.0"?>
<!-- Example of traffic control file -->
<traffic>
   <conn>
       <src>10.0.0.1</src>
       <dest>10.0.1.2</dest>
       <setupPri>7</setupPri>
       <holdingPri>7</holdingPri>
       <bandwidth>400</bandwidth>
       <delay>5</delay>
   </conn>
   <conn>
       <src>11.0.0.1</src>
       <dest>11.0.1.2</dest>
       <setupPri>7</setupPri>
       <holdingPri>7</holdingPri>
       <bandwidth>100</bandwidth>
       <delay>5</delay>
   </conn>
</traffic>
\end{verbatim}

An example of using RSVP-TE as signaling protocol can be found in
ExplicitRouting folder distributed with the simulation. In this
example, a network similar to the network in LDP-MPLS example is
setup. Instead of using LDP, "signaling" parameter is set to 2 (value
of RSVP-TE handler). The following xml file is used for traffic
control. Note the explicit routes specified in the second connection.
It indicates that the route is a strict one since the values of every
hop types are 0. The route defined is 10.0.0.1 -> 1.0.0.1 ->
10.0.0.3 -> 1.0.0.4 -> 10.0.0.5 -> 10.0.1.2.

\begin{verbatim}
<?xml version="1.0"?>
<!-- Example of traffic control file -->
<traffic>
    <conn>
        <src>10.0.0.1</src>
        <dest>10.0.1.2</dest>
        <setupPri>7</setupPri>
        <holdingPri>7</holdingPri>
        <bandwidth>0</bandwidth>
        <delay>0</delay>
        <ER>false</ER>
    </conn>
    <conn>
        <src>11.0.0.1</src>
        <dest>11.0.1.2</dest>
        <setupPri>7</setupPri>
        <holdingPri>7</holdingPri>
        <bandwidth>0</bandwidth>
        <delay>0</delay>
        <ER>true</ER>
        <route>1.0.0.1,0,1.0.0.3,0,1.0.0.4,0,1.0.0.5,0,10.0.1.2,0</route>
    </conn>
</traffic>
\end{verbatim}

\section{MPLS-enabled Router Models}

MPLS-enabled routers are \nedtype{LdpRouter}, \nedtype{RsvpRouter} 

TODO 

\section{Example Network}

TODO

\section{Related Standards}

\begin{itemize}
  \item RFC 2702: Requirements for Traffic Engineering Over MPLS
  \item RFC 3031: Multiprotocol Label Switching Architecture
  \item RFC 3036: LDP Specification
  \item RFC 2205: Resource ReSerVation Protocol
  \item RFC 3209: RSVP-TE Extension to RSVP for LSP tunnels
  \item RFC 2205: RSVP Version 1 - Functional Specification
  \item RFC 2209: RSVP Message processing Version 1
\end{itemize}

%%% Local Variables:
%%% mode: latex
%%% TeX-master: "usman"
%%% End:

\cleardoublepage

\include{ch-ppp}
\cleardoublepage

% last synchronized to 'dbc28949bf4332ac86d84b95705fbea9af4f84f7'
\chapter{The Ethernet Model}
\label{cha:ethernet}

TODO: 802.1d (STP, RSTP), 802.2 (LLC), ...

% TODO: comment numWirelessPorts in MacRelayUnitPP
% TODO: comment origByteLength in EtherFrame
% FIXME: wrong header length in EtherFrame.msg

\section{Overview}

Variations: 10Mb/s ethernet, fast ethernet, Gigabit Ethernet, Fast Gigabit Ethernet, full duplex

The Ethernet model contains a MAC model (\nedtype{EtherMac}), LLC model (\nedtype{EtherLlc}) as well
as a bus (\nedtype{EtherBus}, for modelling coaxial cable) and a hub (\nedtype{EtherHub}) model.
A switch model (\nedtype{EtherSwitch}) is also provided.

\begin{itemize}
  \item \nedtype{EtherHost} is a sample node with an Ethernet NIC;
  \item \nedtype{EtherSwitch}, \nedtype{EtherBus}, \nedtype{EtherHub} model switching hub, repeating hub and
        the old coxial cable;
  \item basic components of the model: \nedtype{EtherMac}, \nedtype{EtherLlc}/\nedtype{EtherEncap} module types,
        \nedtype{MacRelayUnit} (\nedtype{MACRelayUnitNP} and \nedtype{MACRelayUnitPP}), \nedtype{EtherFrame} message type,
        \cppclass{MacAddress} class
\end{itemize}


\section{Physical layer}

Stations on an Ethernet networks are connected by coaxial,
twisted pair or fibre cables. (Coaxial only has historical importance,
but is supported by INET anyway.) There are several cable types specified
in the standard.

In the INET framework, the cables are represented by connections.
The connections used in Ethernet LANs must be derived from
\nedtype{DatarateConnection} and should have their \fpar{delay} and
\fpar{datarate} parameters set.
The delay parameter can be used to model the distance between the
nodes. The datarate parameter can have four values:

\begin{itemize}
  \item \ttt{10Mbps} classic Ethernet
  \item \ttt{100Mbps} Fast Ethernet
  \item \ttt{1Gbps} Gigabit Ethernet
  \item \ttt{10Gbps} Fast Gigabit Ethernet
\end{itemize}


\subsection{EtherHub}

Ethernet hubs are a simple broadcast devices. Messages arriving on a port
are regenerated and broadcast to every other port.

The connections connected to the hub must have the same data rate.
Cable lengths should be reflected in the delays of the connections.

Messages are not interpreted by the \nedtype{EtherType} hub model in any way,
thus the hub model is not specific to Ethernet. Messages may
represent anything, from the beginning of a frame transmission to
end (or abortion) of transmission.

% TODO: model delay in hubs: class I device 140 bit time, class II device 92 bit time (for fast ethernet)

\subsection{EtherBus}

The \nedtype{EtherBus} component can model a common coaxial cable
found in early Ethernet LANs. The nodes are attached via taps at specific
positions on the cable. When a node sends a signal, it will propagate
along the cable in both directions at the given propagation speed.

The gates of the \nedtype{EtherBus} represent taps. The positions
of the taps are given by the \fpar{positions} parameter as a
space separated list of distances in metres. If there are more
gates then positions given, the last distance is repeated.
The bus component send the incoming message in one direction and
a copy of the message to the other direction (except at the ends).
The propagation delays are computed from the distances of the taps
and the \fpar{propagationSpeed} parameter.

Messages are not interpreted by the bus model in any way, thus the bus
model is not specific to Ethernet. Messages may represent anything, 
from the beginning of a frame transmission to end (or abortion) of transmission.

% FIXME #356 NED comment is wrong: data rate must not be zero!
% FIXME #354 default propagation speed is wrong (should be 2e8mps)
%            btw there is a hard coded propagation speed in EtherMACBase.cc


\section{Ethernet Interfaces}

\subsection{EthernetInterface}

The \nedtype{EthernetInterface} compound module implements the \nedtype{IWiredInterface}
interface. Complements \nedtype{EtherMac} and \nedtype{EtherEncap} with an output queue
for QoS and RED support. It also has configurable input/output filters as \nedtype{IHook}
components similarly to the \nedtype{PppInterface} module.

% TODO there is no IWiredNic with EtherLLC


\subsection{Ethernet MAC Layer}

The Ethernet MAC (Media Access Control) layer transmits the Ethernet frames on
the physical media. This is a sublayer within the data link layer. Because
encapsulation/decapsulation is not always needed (e.g. switches does not do
encapsulation/decapsulation), it is implemented in a separate modules
(\nedtype{EtherEncap} and \nedtype{EtherLlc}) that are part of the LLC layer.

\subsection{Implemented Standards}

The Ethernet model operates according to the following standards:

\begin{itemize}
  \item Ethernet: IEEE 802.3-1998
  \item Fast Ethernet: IEEE 802.3u-1995
  \item Full-Duplex Ethernet with Flow Control: IEEE 802.3x-1997
  \item Gigabit Ethernet: IEEE 802.3z-1998
\end{itemize}

Nowadays almost all Ethernet networks operate using full-duplex
point-to-point connections between hosts and switches. This means
that there are no collisions, and the behaviour of the MAC component
is much simpler than in classic Ethernet that used coaxial cables and
hubs. The INET framework contains two MAC modules for Ethernet:
the \nedtype{EtherMacFullDuplex} is simpler to understand and easier to extend,
because it supports only full-duplex connections. The \nedtype{EtherMac}
module implements the full MAC functionality including CSMA/CD, it
can operate both half-duplex and full-duplex mode.

\subsection*{Packets and frames}

The environment of the MAC modules is described by the \nedtype{IEtherMac}
module interface. Each MAC modules has gates to connect to the physical
layer (\ttt{phys\$i} and \ttt{phys\$o}) and to connect to the upper layer
(LLC module is hosts, relay units in switches): \ttt{upperLayerIn} and
\ttt{upperLayerOut}.

When a frame is received from the higher layers, it must be an
\msgtype{EtherFrame}, and with all protocol fields filled out
(including the destination MAC address). The source address, if left empty,
will be filled in with the configured \fpar{address} of the MAC.
% TODO document auto MAC address


Packets received from the network are \msgtype{EtherTraffic} objects.
They are messages representing inter-frame gaps (\msgtype{EtherPadding}),
jam signals (\msgtype{EtherJam}), control frames (\msgtype{EtherPauseFrame})
or data frames (all derived from \msgtype{EtherFrame}). Data frames
are passed up to the higher layers without modification.
In \fpar{promiscuous} mode, the MAC passes up all received frames;
otherwise, only the frames with matching MAC addresses and
the broadcast frames are passed up.

Also, the module properly responds to PAUSE frames, but never sends them
by itself -- however, it transmits PAUSE frames received from upper layers.
See section~\ref{subsec:pause_handling} for more info.

\subsection*{Queueing}

When the transmission line is busy, messages received from the upper layer
needs to be queued.

In routers, MAC relies on an external queue module (see \nedtype{OutputQueue}),
and requests packets from this external queue one-by-one. The name of the
external queue must be given as the \fpar{queueModule} parameer.
There are implementations of \nedtype{OutputQueue} to model finite buffer,
QoS and/or RED.

In hosts, no such queue is used, so MAC contains an internal
queue named \fvar{txQueue} to queue up packets waiting for transmission.
Conceptually, \fvar{txQueue} is of infinite size, but for better diagnostics
one can specify a hard limit in the \fpar{txQueueLimit} parameter -- if this is
exceeded, the simulation stops with an error.

\subsection*{PAUSE handling}
\label{subsec:pause_handling}

The 802.3x standard supports PAUSE frames as a means of flow
control. The frame contains a timer value, expressed as a multiple
of 512 bit-times, that specifies how long the transmitter should
remain quiet. If the receiver becomes uncongested before the
transmitter's pause timer expires, the receiver may elect to send
another PAUSE frame to the transmitter with a timer value of zero,
allowing the transmitter to resume immediately.

\nedtype{EtherMac} will properly respond to PAUSE frames it receives
(\msgtype{EtherPauseFrame} class),
however it will never send a PAUSE frame by itself. (For one thing,
it doesn't have an input buffer that can overflow.)

\nedtype{EtherMac}, however, transmits PAUSE frames received by higher layers,
and \nedtype{EtherLlc} can be instructed by a command to send a PAUSE frame to MAC.

% FIXME PAUSE frames should only be sent on full-duplex ethernet.
%       If a switch uses half-duplex mode to connect to hosts, it can ask sending hosts
%       to slow down their sending rates:
%       - force collisions with incoming frames
%       - make it appear as if the channel is busy
% FIXME PAUSE frames should have 0x8808 in the etherType field

\subsection*{Error handling}

If the MAC is not connected to the network ("cable unplugged"), it will
start up in "disabled" mode. A disabled MAC simply discards any messages
it receives. It is currently not supported to dynamically connect/disconnect
a MAC.

CRC checks are modeled by the \fvar{bitError} flag of the packets. Erronous
packets are dropped by the MAC.


%\subsection*{Auto-Negotiation}
% Ethernet Auto-Negotiation not supported



\subsection{EtherMacFullDuplex}

From the two MAC implementation \nedtype{EtherMacFullDuplex} is the simpler one,
it operates only in full-duplex mode (its \fpar{duplexEnabled} parameter fixed to
\ttt{true} in its NED definition). This module does not need to implement
CSMA/CD, so there is no collision detection, retransmission with exponential backoff,
carrier extension and frame bursting. Flow control works as described in section
\ref{subsec:pause_handling}.

% FIXME remove frameBursting from NED def, or fix it to false
%       currently setting it to 'true' has no effect

In the \nedtype{EtherMacFullDuplex} module,
packets arrived at the \ttt{phys\$i} gate are handled when their last bit received.

Outgoing packets are transmitted according to the following state diagram:

\begin{center}
\includegraphics{figures/EtherMACFullDuplex_txstates}
\end{center}

The \nedtype{EtherMacFullDuplex} module records two scalars in addition to the
ones mentioned earlier:
\begin{itemize}
\item \ttt{rx channel idle (\%)}: reception channel idle time
        as a percentage of the total simulation time
\item \ttt{rx channel utilization (\%)}: total reception
        time as a percentage of the total simulation time
\end{itemize}

\subsection{EtherMac}

Ethernet MAC layer implementing CSMA/CD. It supports both half-duplex and full-duplex operations;
in full-duplex mode it behaves as \nedtype{EtherMacFullDuplex}. In half-duplex  mode
it detects collisions, sends jam messages and retransmit frames upon collisions using
the exponential backoff algorithm. In Gigabit Ethernet networks it supports carrier
extension and frame bursting. Carrier extension can be turned off by setting the
\fpar{carrierExtension} parameter to \ttt{false}.

Unlike \nedtype{EtherMacFullDuplex}, this MAC module processes the incoming packets when their
first bit is received. The end of the reception is calculated by the MAC and
detected by scheduling a self message.

When frames collide the transmission is aborted -- in this case the transmitting
station transmits a jam signal. Jam signals are represented
by a \msgtype{EtherJam} message. The jam message contains the tree identifier
of the frame whose transmission is aborted. When the \nedtype{EtherMac} receives a jam
signal, it knows that the corresponding transmission ended in jamming and have
been aborted. Thus when it receives as many jams as collided frames, it can
be sure that the channel is free again. (Receiving a jam message marks the
beginning of the jam signal, so actually has to wait for the duration of the jamming.)

The operation of the MAC module can be schematized by the following state chart:

\begin{center}
\includegraphics{figures/EtherMAC_txstates}
\end{center}

The module generates these extra signals:
\begin{itemize}
\item \fsignal{collision} when collision starts (received a frame,
         while transmitting or receiving another one; or start to transmit while receiving a frame),
         the constant value 1
\item \fsignal{backoff} when jamming period ended and before waiting according to the
         exponential backoff algorith, the constant value 1
\end{itemize}

These scalar statistics are generated about the state of the line:
\begin{itemize}
  \item \ttt{rx channel idle (\%)} reception channel idle time (full duplex) or channel
         idle time (half-duplex), as a percentage of the total simulation time
  \item \ttt{rx channel utilization (\%)} total successful reception time (full-duplex) or total
         successful reception/transmission time (half duplex), as a percentage
         of the total simulation time
  \item \ttt{rx channel collision (\%)} total unsuccessful reception time, as a percentage
         of the total simulation time
  \item \ttt{collisions} total number collisions (same as count of \fsignal{collisionSignal})
  \item \ttt{backoffs} total number of backoffs (same as count of \fsignal{backoffSignal})
\end{itemize}

\subsection{EtherEncap}

The \nedtype{EtherEncap} module generates \msgtype{EthernetIIFrame} messages.

EtherFrameII

\subsection{EtherLlc}

TODO what it does


% document error conditions (causing error() calls in the code)

% FIXME handleRestransmission() comment is not true: // no beginSendFrames(), because end of jam signal sending will trigger it automatically
%       in case of inner queue, the queued msg is not transmitted
% FIXME should not enter PAUSE state when !duplexMode


\section{Switches}

Ethernet switches play an important role in modern Ethernet LANs. Unlike
passive hubs and repeaters, that work in the physical layer, the switches
operate in the data link layer and routes data frames between the connected
subnets.

While a hub repeats the data frames on each connected line, possibly causing
collisions, switches help to segment the network to small collision domains.
In modern Gigabit LANs each node is connected to the switch direclty
by full-duplex lines, so no collisions are possible. In this case the
CSMA/CD is not needed and the channel utilization can be high.

\subsection{MAC relay units}

INET framework ethernet switches are built from \nedtype{IMacRelayUnit}
components. Each relay unit has N input and output gates for sending/receiving
Ethernet frames. They should be connected to \nedtype{IEtherMac} modules.

Internally the relay unit holds a table for the destination address -> output
port mapping. When it receives a data frame it updates the table with the
source address->input port. The table can also be pre-loaded from a text file
while initializing the relay unit. The file name given as the \fpar{addressTableFile}
parameter. Each line of the file contains a hexadecimal MAC address and a decimal port
number separated by tabs. Comment lines beginning with '\#' are also allowed:

\begin{verbatim}
01 ff ff ff ff    0
00-ff-ff-ee-d1    1
0A:AA:BC:DE:FF    2
\end{verbatim}

% FIXME #352 addressTableSize is not checked in readAddressTable -> if overflown
%            then later check updateTableWithAddress has no effect
% FIXME format is wrong in the comment of readAddressTable()

The size of the lookup table is restricted by the \fpar{addressTableSize} parameter.
When the table is full, the oldest address is deleted. Entries are also deleted
if their age exceeds the duration given as the \fpar{agingTime} parameter.

If the destination address is not found in the table, the frame is broadcasted.
The frame is not sent to the same subnet it was received from, because the
target already received the original frame. The only exception if the frame
arrived through a radio channel, in this case the target can be out of range.
The port range 0..\fpar{numWirelessPorts}-1 are reserved for wireless connections.

The \nedtype{IMacRelayUnit} module is not a concrete implementation,
it just defines gates and parameters an \nedtype{IMacRelayUnit} should have.
Concrete inplementations add
capacity and performance aspects to the model (number of frames processed
per second, amount of memory available in the switch, etc.)
C++ implementations can subclass from the class \cppclass{MACRelayUnitBase}.

There are two versions of \nedtype{IMacRelayUnit}:

\begin{description}
  \item[\nedtype{MACRelayUnitNP}] models one or more CPUs with shared memory,
    working from a single shared queue.
  \item[\nedtype{MACRelayUnitPP}] models one CPU assigned to each incoming port,
    working with shared memory but separate queues.
\end{description}

In both models input messages are queued. CPUs poll messages from the queue
and process them in \fpar{processingTime}. If the memory usage exceeds
\fpar{bufferSize}, the frame will be dropped.

A simple scheme for sending PAUSE frames is built in (although
users will probably change it). When the buffer level goes
above a high watermark, PAUSE frames are sent on all ports.
The watermark and the pause time is configurable; use zero
values to disable the PAUSE feature.

% FIXME valid values for pauseTime: 0..0xFFFF
% FIXME ETHER_PAUSE_COMMAND_BYTES should be 4 in Ethernet.h (2bytes opcode + 2bytes pauseTime)
% FIXME PAUSE frame should not be sent on all ports probably
% TODO add lowWatermark, send PauseFrame(pauseUnits=0) to resume sending

The relay units collects the following statistics:

\begin{description}
\item[usedBufferBytes] memory usage as function of time
\item[processedBytes] count and length of processed frames
\item[droppedBytes] count and length of frames dropped caused by out of memory
\end{description}

% FIXME MACRelayUnitNP: no signals are generated, how does @statistic work in the ned file?

\subsection{EtherSwitch}

Model of an Ethernet switch containing a relay unit and multiple MAC units.

The duplexChannel attributes of the MACs must be set according to the
medium connected to the port; if collisions are possible (it's a bus or hub)
it must be set to false, otherwise it can be set to true.
By default it uses half duples MAC with CSMA/CD.

TODO STP, RSTP



%%% Local Variables:
%%% mode: latex
%%% TeX-master: "usman"
%%% End:

\cleardoublepage

\include{ch-80211}
\cleardoublepage

\ifdraft

\chapter{The 802.15.4 Model}
\label{cha:802154}

\section{Overview}

IEEE 802.15.4 is a technical standard which defines the operation of low-rate
wireless personal area networks (LR-WPANs). IEEE 802.15.4 was designed for data
rates of 250 kbit/s or lower, in order to achieve long battery life (months or
even years) and very low complexity. The standard specifies the physical layer
and media access control.

IEEE 802.15.4is the basis for the ZigBee, ISA100.11a, WirelessHART, MiWi, SNAP,
and the Thread specifications, each of which further extends the standard by
developing the upper layers which are not defined in IEEE 802.15.4.
Alternatively, it can be used with 6LoWPAN, the technology used to deliver IPv6
over WPANs, to define the upper layers. (Thread is also 6LoWPAN-based.)

% https://en.wikipedia.org/wiki/IEEE_802.15.4

\begin{note}

802.15.4 can use narrowband radio (on one of three possible unlicensed frequency
bands), using DSSS or alternatively, a combination of binary keying and amplitude 
shift keying.

In August 2007, IEEE 802.15.4a was released, adding two more PHYs: Direct
Sequence ultra-wideband (UWB), and another one using chirp spread spectrum
(CSS).

In April, 2009 IEEE 802.15.4c and IEEE 802.15.4d were released expanding the
available PHYs with several additional PHYs: one for 780 MHz band using O-QPSK
or MPSK, another for 950 MHz using GFSK or BPSK.


MAC: Important features include real-time suitability by reservation of
guaranteed time slots, collision avoidance through CSMA/CA and integrated
support for secure communications. Devices also include power management
functions such as link quality and energy detection.

\end{note}

The INET Framework contains a basic implementation of IEEE 802.15.4 protocol.

TODO which PHYs are supported; the exact parameterization (modes?) is missing

TODO which MAC features are supported:

MAC: \tbf{OK}: collision avoidance through CSMA/CA; \tbf{missing}: reservation
of guaranteed time slots, and integrated support for secure communications.
Devices also include power management functions such as link quality and energy
detection.

\section{Network Interfaces}

There are two network interfaces that differ in the type of radio:

\begin{itemize}
  \item \nedtype{Ieee802154NarrowbandInterface} -- TODO
  \item \nedtype{Ieee802154UwbIrInterface} -- TODO
\end{itemize}


To create a wireless node with a 802.15.4 interface, use a node type 
that has a wireless interface, and set the interface type to the 
appropriate type. For example, \nedtype{WirelessHost} is a node type 
which is preconfigured to have one wireless interface, \ttt{wlan[0]}.
\ttt{wlan[0]} is of parametric type, so if you build the network from
\nedtype{WirelessHost} nodes, you can configure all of them to use
802.15.4 with the following line in the ini file:

\begin{inifile}
**.wlan[0].typename = "Ieee802154NarrowbandInterface"
\end{inifile}

Corresponding mediums:

Ieee802154UwbIrRadioMedium
Ieee802154NarrowbandRadioMedium -- missing?

\section{MAC Protocol}

Ieee802154Mac: ``Generic CSMA protocol supporting Mac-ACKs as well as
constant, linear or exponential backoff times.''

Ieee802154Mac
Ieee802154NarrowbandMac
Ieee802154UwbIrMac -- missing?

\section{Physical Layer}

Ieee802154NarrowbandScalarRadio
Ieee802154NarrowbandDimensionalRadio
Ieee802154UwbIrRadio

\fi

%%% Local Variables:
%%% mode: latex
%%% TeX-master: "usman"
%%% End:


\cleardoublepage

\include{ch-sensor-macs}
\cleardoublepage

\chapter{The Physical Layer (Transceiver Modeling)}
\label{cha:physicallayer}

\section{Overview}

Wireless network interfaces contain a radio model component, which is
responsible for modeling the physical layer (PHY).\footnote{Wired network interfaces
could similarly contain an explicit PHY model. The reason they do not is that
wired links normally have very low error rates and simple observable behavior,
and there is usually not much to be gained from modeling the physical layer in detail.}
The radio model describes the physical device that is capable of transmitting
and receiving signals on the medium. 

Conceptually, a radio model relies on several sub-models:

\begin{itemize}
  \item antenna model
  \item transmitter model 
  \item receiver model 
  \item error model (as part of the receiver model)
  \item energy consumption model 
\end{itemize}

The antenna model is shared between the transmitter model and the receiver model.
The separation of the transmitter model and the receiver model allows 
asymmetric configurations. The energy consumer model is optional, and 
it is only used when the simulation of energy consumption is necessary.

TODO multiple implementations are provided for each model. For different
level of detail (abstract/fast versus detailed), different modeling strategy, etc.

TODO explain scalar, dimensional, and ``layered''

TODO different signal representations for models of different detail levels, etc.

\section{Generic Radio}

In INET, radio models implement the \nedtype{IRadio} module interface. 
A generic, often used implementation of \nedtype{IRadio} is the 
\nedtype{Radio} NED type. \nedtype{Radio} is an active compound module, 
that is, it has an associated C++ class that encapsulates the computations.

\nedtype{Radio} contains its antenna, transmitter, receiver and energy
consumer models as submodules with parametric types:

\begin{ned}
antenna: <antennaType> like IAntenna;
transmitter: <transmitterType> like ITransmitter;
receiver: <receiverType> like IReceiver;
energyConsumer: <energyConsumerType> like IEnergyConsumer 
    if energyConsumerType != "";
\end{ned}

The following sections describe the parts of the radio model.

\section{Components of a Radio}

\subsection{Antenna Models}

The antenna model describes the effects of the physical device which converts
electric signals into radio waves, and vice versa. This model captures the
antenna characteristics that heavily affect the quality of the communication
channel. For example, various antenna shapes, antenna size and geometry, antenna
arrays, and antenna orientation causes different directional or frequency
selectivity.

The antenna model provides a position and an orientation using a mobility model
that defaults to the mobility of the node. The main purpose of this model is to
compute the antenna gain based on the specific antenna characteristics and the
direction of the signal. The signal direction is computed by the medium from the
position and the orientation of the transmitter and the receiver. The following
list provides some examples:

\begin{itemize}
  \item \nedtype{IsotropicAntenna}: antenna gain is exactly 1 in any direction
  \item \nedtype{ConstantGainAntenna}: antenna gain is a constant determined by
    a parameter
  \item \nedtype{DipoleAntenna}: antenna gain depends on the direction according
    to the dipole antenna characteristics
  \item \nedtype{InterpolatingAntenna}: antenna gain is computed by linear
    interpolation according to a table indexed by the direction angles
\end{itemize}

\subsection{Transmitter Models}

The transmitter model describes the physical process which converts packets into
electric signals. In other words, this model converts an L2 frame into a signal
that is transmitted on the medium. The conversion process and the representation
of the signal depends on the level of detail and the physical characteristics
of the implemented protocol.

There are two main levels of detail (or modeling depths):
 
\begin{itemize}
\item In the \textit{flat model}, the transmitter model skips the symbol domain 
and the sample domain representations, and it directly creates the analog domain 
representation. The bit domain representation is reduced to the bit length of 
the packet, and the actual bits are ignored.

\item In the \textit{layered model}, the conversion process involves various 
processing steps such as packet serialization, forward error correction encoding, 
scrambling, interleaving, and modulation. This transmitter model requires 
significantly more computation, but it produces accurate bit domain, 
symbol domain, and sample domain representations.
\end{itemize}

Some of the transmitter types available in INET:

\begin{itemize}
  \item \nedtype{UnitDiskTransmitter}
  \item \nedtype{ApskScalarTransmitter}
  \item \nedtype{ApskDimensionalTransmitter}
  \item \nedtype{ApskLayeredTransmitter}
  \item \nedtype{Ieee80211ScalarTransmitter}
  \item \nedtype{Ieee80211DimensionalTransmitter}
\end{itemize}

TODO scalar radio parameters:
    power
    frquency
    bandwidth
    ...

TODO dimensional parameterization: 

 ApskDimensionalTransmitter:
        string dimensions = default("time");                // dimensions of power: time and/or frequency
        string timeGains = default("0% 0dB 100% 0dB");      // sequence of time and gain pairs; time is in [%] or [s], negative time measures from the end; gain is in [dB] or [0..1]; default value is a flat signal
        string frequencyGains = default("0% 0dB 100% 0dB"); // sequence of frequency and gain pairs; frequency is in [%] or [Hz], negative frequency measures from the end; gain is in [dB] or [0..1]; default value is a flat signal
        string interpolationMode @enum("linear", "sample-hold") = default("sample-hold");


\subsection{Receiver Models}

The receiver model describes the physical process which converts electric
signals into packets. In other words, this model converts a reception, along
with an interference computed by the medium model, into a MAC packet and a
reception indication.

For a packet to be received successfully, reception must be \textit{possible}
(based on reception power, bandwidth, modulation scheme and other characteristics),
it must be \textit{attempted} (i.e. the receiver must synchronize itself on
the preamble and start receiving), and it must be \textit{successful} 
(as determined by the error model and the simulated part of the signal decoding).

In the \textit{flat model}, the receiver model skips the sample domain, the symbol domain,
and the bit domain representations, and it directly creates the packet domain
representation by copying the packet from the transmission. It uses the error
model to decide whether the reception is successful.

In the \textit{layered model}, the conversion process involves various processing steps
such as demodulation, descrambling, deinterleaving, forward error correction
decoding, and deserialization. This reception model requires much more
computation than the flat model, but it produces accurate sample domain, 
symbol domain, and bit domain representations.

Some of the receiver types available in INET:

\begin{itemize}
  \item \nedtype{UnitDiskReceiver}
  \item \nedtype{ApskScalarReceiver}
  \item \nedtype{ApskDimensionalReceiver}
  \item \nedtype{ApskLayeredReceiver}
  \item \nedtype{Ieee80211ScalarReceiver}
  \item \nedtype{Ieee80211DimensionalReceiver}
\end{itemize}


\subsection{Error Models}

Determining reception errors is a crucial part of the reception process.
There are often several different statistical error models in the literature
even for a particular physical layer. In order to support this diversity, the
error model is a separate replaceable component of the receiver. 

The error model describes how the signal to noise ratio affects the amount of
errors at the receiver. The main purpose of this model is to determine whether
the received packet has errors or not. It also computes various physical
layer indications for higher layers such as packet error rate, bit error rate,
and symbol error rate. For the layered reception model it needs to compute the
erroneous bits, symbols, or samples depending on the lowest simulated physical
domain where the real decoding starts. The error model is optional (if omitted,
all receptions are considered successful.)

The following list provides some examples:

\begin{itemize}
  \item \nedtype{StochasticErrorModel}: simplistic error model with constant
    symbol/bit/packet error rates as parameters; suitable for testing. 
  \item \nedtype{ApskErrorModel} 
  \item \nedtype{Ieee80211NistErrorModel}, \nedtype{Ieee80211YansErrorModel}, 
    \nedtype{Ieee80211BerTable\-Error\-Model}: various error models for IEEE 802.11
    network interfaces.
\end{itemize}

\subsection{Power Consumption Models}

A substantial part of the energy consumption of communication devices comes from
transmitting and receiving signals. The energy consumer model describes how the
radio consumes energy depending on its activity. This model is optional (if
omitted, energy consumption is ignored.) 

The following list provides some examples:

\begin{itemize}
  \item \nedtype{StateBasedEpEnergyConsumer}: power consumption is
    determined by the radio state (a combination of radio mode, 
    transmitter state and receiver state), and specified in 
    parameters like \fpar{receiverIdlePowerConsumption} and 
    \fpar{receiverReceivingDataPowerConsumption}, in watts.
  \item \nedtype{StateBasedCcEnergyConsumer}: similar to the previous
    one, but consumption is given in amp\`eres.
\end{itemize}

\section{Layered Radio Models}

In layered radio models, the transmitter and receiver models are split
to several stages to allow more fine-grained modeling. 

For transmission, processing steps such as packet serialization, 
forward error correction (FEC) encoding, scrambling, interleaving, and 
modulation are explicitly modeled. Reception involves the inverse
operations: demodulation, descrambling, deinterleaving, 
FEC decoding, and deserialization.

In layered radio models, these processing steps are encapsulated
in four stages, represented as four submodules in both the 
transmitter and receiver model: 

\begin{enumerate}
  \item \textit{Encoding and Decoding} describe how the packet domain 
    signal representation is converted into the bit domain, and vice versa.
  \item \textit{Modulation and Demodulation} describe how the bit domain
    signal representation is converted into the symbol domain, and vice versa.
  \item \textit{Pulse Shaping and Pulse Filtering} describe how the 
    symbol domain signal representation is converted into the sample domain, 
    and vice versa.
  \item \textit{Digital Analog and Analog Digital Conversion} describe 
    how the sample domain signal representation is converted into the 
    analog domain, and vice versa.
\end{enumerate}

In layered radio transmitters and receivers such as \nedtype{ApskLayeredTransmitter}
and \nedtype{ApskLayeredReceiver}, these submodules have parametric
types to make them replaceable. This provides immense freedom for 
experimentation.

\section{Notable Radio Models}

The \nedtype{Radio} module has several specialized versions derived
from it, where certain submodule types and parameters are set to fixed values.
This section describes some of the frequently used ones.

The radio can be replaced in wireless network interfaces by setting the
\fpar{radioType} parameter, like in the following ini file fragment.
  
\begin{inifile}
**.wlan[*].radioType = "UnitDiskRadio"
\end{inifile}

However, be aware that not all MAC protocols can be used with all radio models,
and that some radio models require a matching transmission medium module. 

\subsection{UnitDiskRadio}

\nedtype{UnitDiskRadio} provides a very simple but fast and predictable 
physical layer model. It is the implementation (with some extensions)
of the \textit{Unit Disk Graph} model, which is widely used 
for the study of wireless ad-hoc networks.
\nedtype{UnitDiskRadio} is applicable if network nodes need 
to have a finite communication range, but physical effects 
of signal propagation are to be ignored.

\nedtype{UnitDiskRadio} allows three radii to be given as parameters,
instead of the usual one: communication range, interference range, and
detection range. One can also turn off interference modeling 
(meaning that signals colliding at a receiver will all be received 
correctly), which is sometimes a useful abstraction.

\nedtype{UnitDiskRadio} needs to be used together with a special physical
medium model, \nedtype{UnitDiskRadioMedium}.

The following ini file fragment shows an example configuration.

TODO wtf about those 0 meters???

\begin{inifile}
*.radioMediumType = "UnitDiskRadioMedium"
*.host[*].wlan[*].radioType = "UnitDiskRadio"
*.host[*].wlan[*].radio.transmitter.bitrate = 2Mbps
*.host[*].wlan[*].radio.transmitter.preambleDuration = 0s
*.host[*].wlan[*].radio.transmitter.headerLength = 100b
*.host[*].wlan[*].radio.transmitter.communicationRange = 100m
*.host[*].wlan[*].radio.transmitter.interferenceRange = 0m    
*.host[*].wlan[*].radio.transmitter.detectionRange = 0m
*.host[*].wlan[*].radio.receiver.ignoreInterference = true
\end{inifile}

As a side note, if modeling full connectivity and ignoring 
interference is required, then \nedtype{ShortcutInterface} 
provides an even simpler and faster alternative.

\subsection{APSK Radio}

APSK radio models provide a hypothetical radio that simulates 
one of the well-known ASP, PSK and QAM modulations. 
(APSK stands for Amplitude and Phase-Shift Keying.)

APSK radio has scalar/dimensional, and flat/layered variants.
The flat variants, \nedtype{ApskScalarRadio} and \nedtype{ApskDimensionalRadio}
model frame transmissons in the selected modulation scheme
but without utilizing other techniques such as forward error 
correction (FEC), interleaving, spreading, etc. These radios
require matching medium models, \nedtype{ApskScalarRadioMedium}
and \nedtype{ApskDimensionalRadioMedium}.

The layered versions, \nedtype{ApskLayeredScalarRadio} 
and \nedtype{ApskLayeredDimensionalRadio} can not only
model the processing steps missing from their simpler counterparts,
they also feature configurable level of detail: the transmitter
and receiver modules have \fpar{levelOfDetail} parameters that 
control which domains are actually simulated.
These radio models must be used in conjuction with
\nedtype{ApskLayeredScalarRadioMedium} and 
\nedtype{ApskLayeredDimensionalRadioMedium}, respectively.

TODO ApskLayeredScalarRadio and ApskLayeredDimensionalRadio types are missing, actually create them!!! 

TODO limitations for usage in real-world protocol models

TODO example: 1 for flat!

\begin{inifile}
TODO
\end{inifile}


TODO fragment for a layered one!

\begin{inifile}
## Iteration
**.wlan[*].radio.**.levelOfDetail = ${detail="packet", "bit", "symbol"}
**.wlan[*].radio.**.modulation = ${modulation="BPSK", "QPSK", "QAM-16", "QAM-64"}
**.wlan[*].radio.**.fecType = ${fecType="", "ConvolutionalCoder"}
**.bitrate = ${bitrate=$fecType == "" ? 36Mbps : 18Mbps} # we want to have the same 36Mbps gross bitrate (applying 1/2 code rate) 

## Transmitter
**.wlan[*].radio.transmitterType = "ApskLayeredTransmitter"
**.wlan[*].radio.transmitter.encoderType = "ApskEncoder"
**.wlan[*].radio.transmitter.modulatorType = "ApskModulator"

# scrambler
#**.wlan[*].radio.transmitter.scramblerType = "TODO"
**.wlan[*].radio.transmitter.scrambler.seed = "1011101"
**.wlan[*].radio.transmitter.scrambler.generatorPolynomial = "0001001"

# FEC
**.wlan[*].radio.transmitter.encoder.fecEncoder.transferFunctionMatrix = "1 3"
**.wlan[*].radio.transmitter.encoder.fecEncoder.constraintLengthVector = "2"
**.wlan[*].radio.transmitter.encoder.fecEncoder.puncturingMatrix = "1; 1"
**.wlan[*].radio.transmitter.encoder.fecEncoder.punctureK = 1
**.wlan[*].radio.transmitter.encoder.fecEncoder.punctureN = 2

# interleaver
# **.wlan[*].radio.transmitter.encoder.interleaverType = "TODO"

## Receiver
**.wlan[*].radio.receiverType = "ApskLayeredReceiver"
**.wlan[*].radio.receiver.errorModelType = "ApskLayeredErrorModel"
**.wlan[*].radio.receiver.decoderType = "ApskDecoder"
**.wlan[*].radio.receiver.demodulatorType = "ApskDemodulator"

# descrambler
#**.wlan[*].radio.receiver.scramblerType = "TODO"
**.wlan[*].radio.receiver.descrambler.seed = "1011101"
**.wlan[*].radio.receiver.descrambler.generatorPolynomial = "0001001"

# FEC
**.wlan[*].radio.receiver.decoder.fecDecoder.transferFunctionMatrix = "1 3"
**.wlan[*].radio.receiver.decoder.fecDecoder.constraintLengthVector = "2"
**.wlan[*].radio.receiver.decoder.fecDecoder.puncturingMatrix = "1; 1"
**.wlan[*].radio.receiver.decoder.fecDecoder.punctureK = 1
**.wlan[*].radio.receiver.decoder.fecDecoder.punctureN = 2

# Deinterleaver
# **.wlan[*].radio.receiver.decoder.deinterleaverType = "TODO"
\end{inifile}

\subsection{IEEE 802.11 Radios}

TODO why 802.11 needs specialized models 

\subsection{IEEE 802.15.4 Radios}

TODO why 802.15.4 needs specialized models 

\subsection{UWB-IR Radios}

TODO what is it


%%% Local Variables:
%%% mode: latex
%%% TeX-master: "usman"
%%% End:


\cleardoublepage

\chapter{The Transmission Medium}
\label{cha:transmission-medium}

\section{Overview}

For wireless communication, an additional module is required to model the
shared physical medium where the communication takes place. This module
keeps track of transceivers, noise sources, ongoing transmissions,
background noise, and other ongoing noises.

It relies on several models:
 
\begin{enumerate}
  \item signal propagation model
  \item path loss model
  \item obstacle loss model
  \item background noise model
  \item signal analog model
\end{enumerate}

With the help of the above models, the medium module computes 
when, where, and how signals arrive at receivers, including 
the set of interfering signals and noises. In addition, 
the medium module also contains various mechanisms and ways 
to improve the scalability of wireless network simulations.

\section{RadioMedium}

The standard transmission medium model in INET is \nedtype{RadioMedium}. 
\nedtype{RadioMedium} is as an OMNeT++ compound module with
several replaceable submodules. It contains submodules for 
each of the above models (signal propagation, path loss, etc.), 
and various caches for efficiency. 

Note that \nedtype{RadioMedium} is an active compound module, that is, 
it has an associated C++ class that encapsulates the computations.

\nedtype{RadioMedium} contains its components as submodules 
with parametric types:

\begin{ned}
propagation: <propagationType> like IPropagation;
analogModel: <analogModelType> like IAnalogModel;
backgroundNoise: <backgroundNoiseType> like IRadioBackgroundNoise 
    if backgroundNoiseType != "";
pathLoss: <pathLossType> like IPathLoss;
obstacleLoss: <obstacleLossType> like IObstacleLoss 
    if obstacleLossType != "";
mediumLimitCache: <mediumLimitCacheType> like IMediumLimitCache;
communicationCache: <communicationCacheType> like ICommunicationCache;
neighborCache: <neighborCacheType> like INeighborCache 
    if neighborCacheType != "";
\end{ned}

There are many preconfigured versions of \nedtype{RadioMedium}:

\begin{itemize}
  \item For use with \nedtype{UnitDiskRadio}: \nedtype{UnitDiskRadioMedium}
  \item For APSK radios: \nedtype{ApskScalarRadioMedium}, \nedtype{ApskDimensionalRadioMedium},
    \nedtype{ApskLayeredScalarRadioMedium}, \nedtype{ApskLayeredDimensionalRadioMedium},
  \item For IEEE 802.11: \nedtype{Ieee80211ScalarRadioMedium}, \nedtype{Ieee80211DimensionalRadioMedium},
    \nedtype{Ieee80211LayeredScalarRadioMedium}, \nedtype{Ieee80211LayeredDimensionalRadioMedium},
  \item For IEEE 802.15.4: \nedtype{Ieee802154UwbIrRadioMedium}, \nedtype{Ieee802154NarrowbandScalar\-RadioMedium}
\end{itemize}

The following sections describe the parts of the medium model.

\section{Propagation Models}

When a transmitter starts to transmit a signal, the beginning of the signal
propagates through the transmission medium. When the transmitter ends the
transmission, the signal's end propagates similarly. The propagation model
describes how a signal moves through space over time. Its main purpose is
to compute the arrival space-time coordinates at receivers. There are two
built-in models in INET, implemented as simple modules:

\begin{itemize}
        \item \nedtype{ConstantTimePropagation} is a simplistic model where the propagation time is independent of the traveled distance. The propagation time is simply determined by a module parameter.
        \item \nedtype{ConstantSpeedPropagation} is a more realistic model where the propagation time is proportional to the traveled distance. The propagation time is independent of the transmitter and receiver movement during both signal transmission and propagation. The propagation speed is determined by a module parameter.
\end{itemize}

The default propagation model is configured as follows:

\inisnippet{PropagationModelConfigurationExample}{Propagation model configuration example}

A more accurate model could take into consideration the transmitter and
receiver movement. This effect becomes especially important for acoustic
communication, because the propagation speed of the signal is much more
comparable to the speed of the transceivers.

\section{Path Loss Models}

As a signal propagates through space its power density decreases. This is
called path loss and it is the combination of many effects such as
free-space loss, refraction, diffraction, reflection, and absorption. There
are several different path loss models in the literature, which differ in
their parameterization and application area.

In INET, a path loss model is an OMNeT++ simple module implementing a
specific path loss algorithm. Its main purpose is to compute the power loss
for a given signal, but it is also capable of estimating the range for a
given loss. The latter is useful, for example, to allow visualizing
communication range. INET contains a number of built-in path loss
algorithms, each comes with its own set of parameters:

\begin{itemize}
        \item \nedtype{FreeSpacePathLoss} models line of sight path loss for air or vacuum.
        \item \nedtype{BreakpointPathLoss} refines it using dual slope model with two separate path loss exponents.
        \item \nedtype{LogNormalShadowing} models path loss for a wide range of environments (e.g. urban areas, and buildings)
        \item \nedtype{TwoRayGroundReflection} models interference between line of sight and single ground reflection.
        \item \nedtype{TwoRayInterference} refines the above for inter-vechicle communication.
        \item \nedtype{RicianFading} is a stochastical model for the anomaly caused by partial cancellation of a signal by itself.
        \item \nedtype{RayleighFading} is a stochastical model for heavily built-up urban environments when there is no dominant propagation along the line of sight.
        \item \nedtype{NakagamiFading} further refines the above two models for cellular systems.
\end{itemize}

The following example replaces the default free-space path loss model with
log normal shadowing:

\inisnippet{PathLossConfigurationExample}{Path loss configuration example}

\section{Obstacle Loss Models}

When the signal propagates through space it also passes through physical
objects present in that space. As the signal penetrates physical objects,
its power decreases when it reflects from surfaces, and also when it is
absorbed by their material. There are various ways to model this effect,
which differ in the trade-off between accuracy and performance.

In INET, an obstacle loss model is an OMNeT++ simple module. Its main
purpose is to compute the power loss based on the traveled path and the
signal frequency. The obstacle loss models most often use the physical
environment model to determine the set of penetrated physical objects.
INET contains a few built-in obstacle loss models:

\begin{itemize}
        \item \nedtype{IdealObstacleLoss} model determines total or no power loss at all by checking if there is any obstructing physical object along the straight propagation path.
        \item \nedtype{DielectricObstacleLoss} computes the power loss based on the accurate dielectric and reflection loss along the straight path considering the shape, position, orientation, and material of obstructing physical objects.
\end{itemize}

By default, the medium module doesn't contain any obstacle loss model, but
configuring one is very simple:

\inisnippet{ObstacleLossModelConfigurationExample}{Obstacle loss model configuration example}

Statistical obstacle loss models are also possible but currently not provided.

\section{Background Noise Models}

Thermal noise, cosmic background noise, and other random fluctuations of
the electromagnetic field affect the quality of the communication channel.
This kind of noise doesn't come from a particular source, so it doesn't
make sense to model its propagation through space. The background noise
model describes instead how it changes over space and time.

In INET, a background noise model is an OMNeT++ simple module. Its main
purpose is to compute the analog representation of the background noise for
a given space-time interval. For example,
\nedtype{IsotropicScalarBackgroundNoise} computes a background noise that is
independent of space-time coordinates, and its scalar power is determined
by a module parameter.

The simplest background noise model can be configured as follows:

\inisnippet{BackgroundNoiseModelConfigurationExample}{Background noise model configuration example}

\section{Analog Models}

The analog signal is a complex physical phenomenon which can be modeled in
many different ways. Choosing the right analog domain signal representation
is the most important factor in the trade-off between accuracy and
performance. The analog model of the transmission medium determines how
signals are represented while being transmitted, propagated, and received.

In INET, an analog model is an OMNeT++ simple module. Its main purpose is
to compute the received signal from the transmitted signal. The analog
model combines the effect of the antenna, path loss, and obstacle loss
models. Transceivers must be configured transmit and receive signals
according to the representation used by the analog model.

The most commonly used analog model, which uses a scalar signal power
representation over a frequency and time interval, can be configured as
follows:

\inisnippet{AnalogModelConfigurationExample}{Analog model configuration example}

\section{Neighbor Cache}

Transceivers are considered neighbors if successful communication is
possible between them. For wired communication it is easy to determine
which transceivers are neighbors, because they are connected by wires. In
contrast, in wireless communication determining which transceivers are
neighbors isn't obvious at all.

TODO: it's about *notification*, i.e. giving the radio a chance
to react. Signals are counted as interference, regardless of being neighbor or not!

TODO: if no cache, all receivers will be notified

TODO: query always happens with a *radius*!  ``which radios are in your  X-meter proximity''

In INET, a neighbor cache is an OMNeT++ simple module which provides
an efficient way of keeping track of the neighbor relationship between
transceivers. Its main purpose is to compute the set of affected receivers
for a given transmission. All built-in models in INET provide a
conservative approximation only, because they update their state
periodically:

\begin{itemize}
  \item \nedtype{NeighborListNeighborCache} takes a range as parameter,
    and for each transceiver it maintains the list of receivers within
    range (\textit{neighbor list}).
  \item \nedtype{GridNeighborCache} organizes transceivers in a 3D grid with 
    constant cell size.
  \item \nedtype{QuadTreeNeighborCache} organizes transceivers in a 2D quad tree
    (ignoring the Z axis) with constant node size.
\end{itemize}

The following example sets \nedtype{QuadTreeNeighborCache} as neighbor cache:

\inisnippet{NeighborCacheModelConfigurationExample}{Neighbor cache model configuration example}

How should one decide which neighbor cache to choose for a given simulation? 
As the sole purpose of the neighbor cache is to speed up the simulation, 
one should choose the one that leads to the best performance for that particular
network. Which one performs best is best determined by experimentation, as it
depends on many factors: number of nodes, their spatial distribution, their
speed and movement pattern, their communication pattern, and so on. 
Note that not only the choice of neighbor cache but also its parameterization
can affect performance.


\section{Medium Limit Cache}

The medium limit cache (and its default implementation \nedtype{MediumLimitCache})
keeps track of certain thresholds and minimum/maximum values of quantities 
related to layer 1 modeling. Some of these limits can be gathered from other 
modules in the network, but still, all of them can be explicitly specified by the user. 
The quantities include:

\begin{itemize}
    \item maximum speed (can be gathered from mobility models)
    \item maximum transmission power
    \item minimum interference power and reception power
    \item maximum antenna gain (can be computed from antenna models)
    \item minimum time interval to consider two overlapping signals interfering
    \item maximum duration of a transmission
    \item maximum communication range and interference range 
      (can be computed from transmitter and receiver models)
\end{itemize}

These limits allow the transmission medium model to make assumptions about the
locations of nodes (i.e. the maximum distance they can move during some
interval), about the possibility of interference, and about the possibility
of a signal being receivable.


\section{Communication Cache}

The communication cache is used to cache various intermediate computation
results related to the communication on the medium. The main motivation to have
multiple implementations is that different implementations may be the most
efficient in different simulations. Also, a conservative (simple but robust)
implementation may be used for validating new (more efficient but also more
complex) implementations.

Implementations include:

\begin{itemize}
  \item \nedtype{ReferenceCommunicationCache}
  \item \nedtype{MapCommunicationCache}
  \item \nedtype{VectorCommunicationCache}
\end{itemize}


\section{Improving Scalability}

The simulation of wireless networks is inherently less scalable than
that of wired networks. In wired networks, a transmission only affects 
the host's neighbors on the link, which is usually 1 in modern networks
that are dominated by point-to-point links. The wireless medium, however, 
is a broadcast medium. Any transmission is ``heard'' by all nodes 
within interference range, not only the intended recipients.
The signal may be receivable by them (and must be indeeded received 
before the destination address field in it can be examined), 
or may interfere with the reception of other transmissions.
Whichever the case, the transmission must be evaluated or processed
by a much larger number of nodes than in the wired case. 
This makes the computational complexity at least $O(n^2)$ ($n$ being
the number of nodes.) Other effects may further increase the exponent.

The medium module provides a set of parameters that can be used
to alleviate the scalability issue. These \textit{filter} parameters
that can be used to reduce the amount of processing at nodes that are
not the indended recipients of the frame, increasing simulation performance.

There are several filters that can be enabled/disabled individually:

\begin{itemize}
  \item \textit{Range filter}. When this filter is active, the medium module
    does not send signals to a radio if it is outside interference range 
    (or communication range, this option can also be selected.)
  \item \textit{Radio mode filter}. When this filter is active, 
    the medium module does not send signals to a radio if it is neither 
    in \textit{receiver} nor in \textit{transceiver} mode.
  \item \textit{Listening filter}. When this filter is active, the medium module 
    does not send signals to a radio if it listens on the channel in 
    incompatible mode (e.g. different carrier frequency and bandwidth, 
    or different modulation)
  \item \textit{MAC address filter}. When this filter is active, the radio medium 
    does not send signals to a radio if it the destination MAC address
    does not match
\end{itemize}

The corresponding module parameters are called \ttt{rangeFilter},
\ttt{radioModeFilter}, \ttt{listeningFilter} and \ttt{macAddressFilter}. 
By default, all filters are turned off.

TODO when is it safe to use them?

TODO example

\section{Pitfalls}

Why a packet is not received correctly by the radio (PHY)?
- radio mode
- listening mode
- range filter (or out of range)
- attenuation (signal too weak)
- interference too strong
- capture not supported (radio already receiving another frame, and does notswitch)
- sensitivity is too low (threshold is too high) [W]
- SNIR threshold
- error model (random)
- PHY layer checksum



\section{Optimizations}

turn on filters

experiment with caches (neighbor cache, comm cache, limits cache -- this one
may numerically alter results) 

use a more abstract radio model that's still suitable


%%% Local Variables:
%%% mode: latex
%%% TeX-master: "usman"
%%% End:

\cleardoublepage

\include{ch-environment}
\cleardoublepage

\include{ch-mobility}
\cleardoublepage

\include{ch-power}
\cleardoublepage

\include{ch-emulation}
\cleardoublepage

\chapter{Network Autoconfiguration}
\label{cha:network-autoconfiguration}

\section{Overview}

TODO: this describes static autoconfiguration of IPv4 networks

\section{Configuring IPv4 Networks}

An IPv4 network is composed of several nodes like hosts, routers,
switches, hubs, Ethernet buses, or wireless access points.
The nodes having a IPv4 network layer (hosts and routers) should be
configured at the beginning of the simulation. The configuration
assigns IP addresses to the nodes, and fills their routing tables.
If multicast forwarding is simulated, then the multicast routing
tables also must be filled in.

% TODO define nodes, IP nodes, routers, multicast routers

The configuration can be manual (each address and route is fully specified
by the user), or automatic (addresses and routes are generated by
a configurator module at startup).

Before version 1.99.4 INET offered \nedtype{Ipv4FlatNetworkConfigurator}
for automatic and routing files for manual configuration.
Both had serious limitations, so a new configurator has been added
in version 1.99.4: \nedtype{Ipv4NetworkConfigurator}. This configurator
supports both fully manual and fully automatic configuration. It
can also be used with partially specified manual configurations,
the configurator fills in the gaps automatically.

The next section describes the usage of \nedtype{Ipv4NetworkConfigurator}.
The legacy solutions \nedtype{Ipv4FlatNetworkConfigurator} and 
routing files are described in subsequent sections.

\subsection{Ipv4NetworkConfigurator}
\label{subsec:ipv4configurator}

The \nedtype{Ipv4NetworkConfigurator} assigns IP addresses and sets up
static routing for an IPv4 network.

It assigns per-interface IP addresses, strives to take subnets into account,
and can also optimize the generated routing tables by merging routing entries.

Hierarchical routing can be set up by using only a fraction of configuration
entries compared to the number of nodes. The configurator also does
routing table optimization that significantly decreases the size of routing
tables in large networks.

The configuration is performed in stage 2 of the initialization. At this
point interface modules (e.g. PPP) has already registered their interface
in the interface table. If an interface is named \ttt{ppp[0]}, then the
corresponding interface entry is named \ttt{ppp0}. This name can be used
in the config file to refer to the interface.

The configurator goes through the following steps:

\begin{enumerate}
  \item  Builds a graph representing the network topology. The graph
     will have a vertex for every module that has a @node property (this
     includes hosts, routers, and L2 devices like switches, access points,
     Ethernet hubs, etc.) It also assigns weights to vertices and edges that
     will be used by the shortest path algorithm when setting up routes.
     Weights will be infinite for IP nodes that have IP forwarding disabled
     (to prevent routes from transiting them), and zero for all other nodes
     (routers and and L2 devices). Edge weights are chosen to be inversely
     proportional to the bitrate of the link, so that the configurator
     prefers connections with higher bandwidth. For internal purposes,
     the configurator also builds a table of all "links" (the link data
     structure consists of the set of network interfaces that are
     on the same point-to-point link or LAN)

  \item  Assigns IP addresses to all interfaces of all nodes. The
     assignment process takes into consideration the addresses and netmasks
     already present on the interfaces (possibly set in earlier initialize
     stages), and the configuration provided in the XML format (described
     below). The configuration can specify "templates" for the address
     and netmask, with parts that are fixed and parts that can be chosen
     by the configurator (e.g. "10.0.x.x"). In the most general case,
     the configurator is allowed to choose any address and netmask for all
     interfaces (which results in automatic address assignment). In the most
     constrained case, the configurator is forced to use the requested addresses
     and netmasks for all interfaces (which translates to manual address assignment).
     There are many possible configuration options between these two extremums. The
     configurator assigns addresses in a way that maximizes the number of
     nodes per subnet. Once it figures out the nodes that belong to a single
     subnet it, will optimize for allocating the longest possible netmask.
     The configurator might fail to assign netmasks and addresses according
     to the given configuration parameters; if that happens, the assignment
     process stops and an error is signalled.

  \item  Adds the manual routes that are specified in the configuration.

  \item  Adds static routes to all routing tables in the network. The
     configurator uses Dijkstra's weighted shortest path algorithm to find
     the desired routes between all possible node pairs. The resulting
     routing tables will have one entry for all destination interfaces in the
     network. The configurator can be safely instructed to add default routes
     where applicable, significantly reducing the size of the host routing
     tables. It can also add subnet routes instead of interface routes further
     reducing the size of routing tables. Turning on this option requires
     careful design to avoid having IP addresses from the same subnet on
     different links. CAVEAT: Using manual routes and static route generation
     together may have unwanted side effects, because route generation ignores
     manual routes.

  \item  Then it optimizes the routing tables for size. This optimization allows
     configuring larger networks with smaller memory footprint and makes the
     routing table lookup faster. The resulting routing table might be
     different in that it will route packets that the original routing table
     did not. Nevertheless the following invariant holds: any packet routed
     by the original routing table (has matching route) will still be routed
     the same way by the optimized routing table.

  \item  Finally it dumps the requested results of the configuration. It can
     dump network topology, assigned IP addresses, routing tables and its
     own configuration format.
\end{enumerate}

The module can dump the result of the configuration in the XML format
which it can read. This is useful to save the result of a time consuming
configuration (large network with optimized routes), and use it as
the config file of subsequent runs.

\subsubsection*{Network topology graph}

The network topology graph is constructed from the nodes
of the network. The node is a module having a @node property
(this includes hosts, routers, and L2 devices like switches,
 access points, Ethernet hubs, etc.). An IP node is a node
that contains an \nedtype{InterfaceTable} and a \nedtype{Ipv4RoutingTable}.
A router is an IP node that has multiple network interfaces,
and IP forwarding is enabled in its routing table module.
In multicast routers the \fpar{forwardMulticast} parameter
is also set to \fkeyword{true}.

A link is a set of interfaces that can send datagrams to each other
without intervening routers. Each interface belongs to exactly
one link. For example two interface connected
by a point-to-point connection forms a link. Ethernet interfaces
connected via buses, hubs or switches.
The configurator identifies links by discovering
the connections between the IP nodes, buses, hubs, and switches.

Wireless links are identified by the \fpar{ssid} or \fpar{accessPointAddress}
parameter of the 802.11 management module. Wireless interfaces
whose node does not contain a management module are supposed
to be on the same wireless link. Wireless links can also be
configured in the configuration file of \nedtype{Ipv4NetworkConfigurator}:
\begin{verbatim}
<config>
  <wireless hosts="area1.*" interfaces="wlan*">
</config>
\end{verbatim}
puts wlan interfaces of the specified hosts into the same wireless link.

If a link contains only one router, it is marked as the gateway
of the link. Each datagram whose destination is outside the link
must go through the gateway.

\subsubsection*{Address assignment}

Addresses can be set up manually by giving the address and netmask for
each IP node. If some part of the address or netmask is unspecified,
then the configurator can fill them automatically. Unspecified fields
are given as an ``x'' character in the dotted notation of the address.
For example, if the address is specified as 192.168.1.1 and the
netmask is 255.255.255.0, then the node address will be 192.168.1.1
and its subnet is 192.168.1.0. If it is given as 192.168.x.x and
255.255.x.x, then the configurator chooses a subnet address in the range
of 192.168.0.0 - 192.168.255.252, and an IP address within the chosen
subnet. (The maximum subnet mask is 255.255.255.252 allows 2 nodes in the subnet.)

The following configuration generates network addresses below the 10.0.0.0
address for each link, and assign unique IP addresses to each host:

\begin{verbatim}
<config>
  <interface hosts="*" address="10.x.x.x" netmask="255.x.x.x"/>
</config>
\end{verbatim}

The configurator tries to put nodes on the same link into the same subnet,
so its enough to configure the address of only one node on each link.

The following example configures a hierarchical network in a way that keeps
routing tables small.
\begin{verbatim}
<config>
  <interface hosts="area11.lan1.*" address="10.11.1.x" netmask="255.255.255.x"/>
  <interface hosts="area11.lan2.*" address="10.11.2.x" netmask="255.255.255.x"/>
  <interface hosts="area12.lan1.*" address="10.12.1.x" netmask="255.255.255.x"/>
  <interface hosts="area12.lan2.*" address="10.12.2.x" netmask="255.255.255.x"/>
  <interface hosts="area*.router*" address="10.x.x.x" netmask="x.x.x.x"/>
  <interface hosts="*" address="10.x.x.x" netmask="255.x.x.0"/>
</config>
\end{verbatim}

The XML configuration must contain exactly one \verb!<config>! element. Under the
root element there can be multiple of the following elements:

The interface element provides configuration parameters for one or more
interfaces in the network. The selector attributes limit the scope where
the interface element has effects. The parameter attributes limit the
range of assignable addresses and netmasks.
The \verb!<interface>! element may contain the following attributes:
\begin{compactitem}
    \item \ttt{@hosts}
      Optional selector attribute that specifies a list of host name patterns.
      Only interfaces in the specified hosts are affected. The pattern might
      be a full path starting from the network, or a module name anywhere in
      the hierarchy, and other patterns similar to ini file keys. The default
      value is "*" that matches all hosts.
      e.g. "subnet.client*" or "host* router[0..3]" or "area*.*.host[0]"

    \item \ttt{@names}
      Optional selector attribute that specifies a list of interface name
      patterns. Only interfaces with the specified names are affected. The
      default value is "*" that matches all interfaces.
      e.g. "eth* ppp0" or "*"

    \item \ttt{@towards}
      Optional selector attribute that specifies a list of host name patterns.
      Only interfaces connected towards the specified hosts are affected. The
      specified name will be matched against the names of hosts that are on
      the same LAN with the one that is being configured. This works even if
      there's a switch between the configured host and the one specified here.
      For wired networks it might be easier to specify this parameter instead
      of specifying the interface names. The default value is "*".
      e.g. "ap" or "server" or "client*"

    \item \ttt{@among}
      Optional selector attribute that specifies a list of host name patterns.
      Only interfaces in the specified hosts connected towards the specified
      hosts are affected.
      The 'among="X Y Z"' is same as 'hosts="X Y Z" towards="X Y Z"'.

    \item \ttt{@address}
      Optional parameter attribute that limits the range of assignable
      addresses. Wildcards are allowed with using 'x' as part of the address
      in place of a byte. Unspecified parts will be filled automatically by
      the configurator. The default value "" means that the address will not
      be configured. Unconfigured interfaces still have allocated addresses
      in their subnets allowing them to become configured later very easily.
      e.g. "192.168.1.1" or "10.0.x.x"

    \item \ttt{@netmask}
      Optional parameter attribute that limits the range of assignable
      netmasks. Wildcards are allowed with using 'x' as part of the netmask
      in place of a byte. Unspecified parts will be filled automatically be
      the configurator. The default value "" means that any netmask can be
      configured.
      e.g. "255.255.255.0" or "255.255.x.x" or "255.255.x.0"

    \item \ttt{@mtu}                number
      Optional parameter attribute to set the MTU parameter in the interface.
      When unspecified the interface parameter is left unchanged.

    \item \ttt{@metric}                number
      Optional parameter attribute to set the Metric parameter in the interface.
      When unspecified the interface parameter is left unchanged.
\end{compactitem}

Wireless interfaces can similarly be configured by adding
\verb!<wireless>! elements to the configuration. Each \verb!<wireless>!
element with a different id defines a separate subnet.
\begin{compactitem}
    \item \ttt{@id} (optional)
      identifies wireless network, unique value used if missed

    \item \ttt{@hosts}
      Optional selector attribute that specifies a list of host name patterns.
      Only interfaces in the specified hosts are affected. The default value
      is "*" that matches all hosts.

    \item \ttt{@interfaces}
      Optional selector attribute that specifies a list of interface name
      patterns. Only interfaces with the specified names are affected. The
      default value is "*" that matches all interfaces.
\end{compactitem}


\subsubsection{Multicast groups}

Multicast groups can be configured by adding \verb!<multicast-group>!
elements to the configuration file. Interfaces belongs to a multicast
group will join to the group automatically.

For example
\begin{verbatim}
<config>
  <multicast-group hosts="router*" interfaces="eth*" address="224.0.0.5"/>
</config>
\end{verbatim}
adds all Ethernet interfaces of nodes whose name starts with ``router''
to the 224.0.0.5 multicast group.

The \verb!<multicast-group>! element has the following attributes:
\begin{compactitem}
    \item \ttt{@hosts}
      Optional selector attribute that specifies a list of host name patterns.
      Only interfaces in the specified hosts are affected. The default value
      is "*" that matches all hosts.

    \item \ttt{@interfaces}
      Optional selector attribute that specifies a list of interface name
      patterns. Only interfaces with the specified names are affected. The
      default value is "*" that matches all interfaces.

    \item \ttt{@towards}
      Optional selector attribute that specifies a list of host name patterns.
      Only interfaces connected towards the specified hosts are affected.
      The default value is "*".

    \item \ttt{@among}
      Optional selector attribute that specifies a list of host name patterns.
      Only interfaces in the specified hosts connected towards the specified
      hosts are affected.
      The 'among="X Y Z"' is same as 'hosts="X Y Z" towards="X Y Z"'.

    \item \ttt{@address}
      Mandatory parameter attribute that specifies a list of multicast group
      addresses to be assigned. Values must be selected from the valid range
      of multicast addresses.
      e.g. "224.0.0.1 224.0.1.33"
\end{compactitem}


\subsubsection*{Manual route configuration}

The \nedtype{Ipv4NetworkConfigurator} module allows the user
to fully specify the routing tables of IP nodes at the beginning
of the simulation.

The \verb!<route>! elements of the configuration add a route to the
routing tables of selected nodes. The element has the following attributes:
\begin{compactitem}
    \item \ttt{@hosts}
      Optional selector attribute that specifies a list of host name patterns.
      Only routing tables in the specified hosts are affected. The default
      value "" means all hosts will be affected.
      e.g. "host* router[0..3]"

    \item \ttt{@destination}
      Optional parameter attribute that specifies the destination address in
      the route (L3AddressResolver syntax). The default value is "*".
      e.g. "192.168.1.1" or "subnet.client[3]" or "subnet.server(ipv4)" or "*"

    \item \ttt{@netmask}
      Optional parameter attribute that specifies the netmask in the route.
      The default value is "*".
      e.g. "255.255.255.0" or "/29" or "*"

    \item \ttt{@gateway}
      Optional parameter attribute that specifies the gateway (next-hop)
      address in the route (L3AddressResolver syntax). When unspecified
      the interface parameter must be specified. The default value is "*".
      e.g. "192.168.1.254" or "subnet.router" or "*"

    \item \ttt{@interface}
      Optional parameter attribute that specifies the output interface name
      in the route. When unspecified the gateway parameter must be specified.
      This parameter has no default value.
      e.g. "eth0"

    \item \ttt{@metric}
      Optional parameter attribute that specifies the metric in the route.
      The default value is 0.
\end{compactitem}

Multicast routing tables can similarly be configured by adding
\verb!<multicast-route>! elements to the configuration.
\begin{compactitem}
    \item \ttt{@hosts}
      Optional selector attribute that specifies a list of host name patterns.
      Only routing tables in the specified hosts are affected.
      e.g. "host* router[0..3]"

    \item \ttt{@source}
      Optional parameter attribute that specifies the address of the source
      network. The default value is "*" that matches all sources.

    \item \ttt{@netmask}
      Optional parameter attribute that specifies the netmask of the source
      network. The default value is "*" that matches all sources.

    \item \ttt{@groups}
      Optional List of IPv4 multicast addresses specifying the groups this entry
      applies to. The default value is "*" that matches all multicast groups.
      e.g. "225.0.0.1 225.0.1.2".

    \item \ttt{@metric}
      Optional parameter attribute that specifies the metric in the route.

    \item \ttt{@parent}
      Optional parameter attribute that specifies the name of the interface
      the multicast datagrams are expected to arrive. When a datagram arrives
      on the parent interface, it will be forwarded towards the child interfaces;
      otherwise it will be dropped. The default value is the interface on the
      shortest path towards the source of the datagram.

    \item \ttt{@children}
      Mandatory parameter attribute that specifies a list of interface name
      patterns:
      \begin{compactitem}
        \item a name pattern (e.g. "ppp*") matches the name of the interface
        \item a 'towards' pattern (starting with ">", e.g. ">router*") matches the interface
         by naming one of the neighbour nodes on its link.
      \end{compactitem}
      Incoming multicast datagrams are forwarded to each child interface except the
      one they arrived in.
\end{compactitem}

The following example adds an entry to the multicast routing table of \ttt{router1},
that intsructs the routing algorithm to forward multicast datagrams whose source
is in the 10.0.1.0 network and whose destinatation address is 225.0.0.1 to
send on the \ttt{eth1} and \ttt{eth2} interfaces assuming it arrived on the
\ttt{eth0} interface:

\begin{verbatim}
<multicast-route hosts="router1" source="10.0.1.0" netmask="255.255.255.0"
                 groups="225.0.0.1" metric="10"
                 parent="eth0" children="eth1 eth2"/>
\end{verbatim}

\subsubsection*{Automatic route configuration}

If the \fpar{addStaticRoutes} parameter is true, then
the configurator add static routes to all routing tables.

The configurator uses Dijkstra's weighted shortest path algorithm to find
the desired routes between all possible node pairs. The resulting
routing tables will have one entry for all destination interfaces in the
network.

%     Weights will be infinite for IP nodes that have IP forwarding disabled
%     (to prevent routes from transiting them), and zero for all other nodes
%     (routers and and L2 devices). Edge weights are chosen to be inversely
%     proportional to the bitrate of the link, so that the configurator
%     prefers connections with higher bandwidth. For internal purposes,

The configurator can be safely instructed to add default routes
where applicable, significantly reducing the size of the host routing
tables. It can also add subnet routes instead of interface routes further
reducing the size of routing tables. Turning on this option requires
careful design to avoid having IP addresses from the same subnet on
different links.


\begin{caution}
Using manual routes and static route generation
together may have unwanted side effects, because route generation ignores
manual routes. Therefore if the configuration file contains
manual routes, then the \fpar{addStaticRoutes} parameter should be set
to \fkeyword{false}.
\end{caution}

\subsubsection*{Route optimization}

If the \fpar{optimizeRoutes} parameter is \fkeyword{true} then the
configurator tries to optimize the routing table for size.
This optimization allows configuring larger networks with smaller
memory footprint and makes the routing table lookup faster.

The optimization is performed by merging routes whose gateway and
outgoing interface is the same by finding a common prefix that
matches only those routes. The resulting routing table might be
different in that it will route packets that the original routing table
did not. Nevertheless the following invariant holds: any packet routed
by the original routing table (has matching route) will still be routed
the same way by the optimized routing table.

\subsubsection*{Parameters}

This list summarize the parameters of the \nedtype{IPv4NetorkConfigurator}:

\begin{params}
  \param{config}
   {XML configuration parameters for IP address assignment and adding manual routes.}
  \param{assignAddresses}
   {assign IP addresses to all interfaces in the network}
  \param{assignDisjunctSubnetAddresses}
   {avoid using the same address prefix and
    netmask on different links when assigning IP addresses to interfaces}
  \param{addStaticRoutes}
   {add static routes to the routing tables of all nodes
    to route to all destination interfaces (only where applicable; turn off when
    config file contains manual routes)}
  \param{addDefaultRoutes}
    {add default routes if all routes from a source node go
     through the same gateway (used only if addStaticRoutes is true)}
  \param{addSubnetRoutes}
   {add subnet routes instead of destination interface routes
    (only where applicable; used only if addStaticRoutes is true)}
  \param{optimizeRoutes}
   {optimize routing tables by merging routes, the resulting routing table might
    route more packets than the original (used only if addStaticRoutes is true)}
  \param{dumpTopology}
   {if true, then the module prints extracted network topology}
  \param{dumpAddresses}
   {if true, then the module prints assigned IP addresses for all interfaces}
  \param{dumpRoutes}
   {if true, then the module prints configured and optimized routing tables for all nodes to
    the module output}
  \param{dumpConfig}
   {name of the file, write configuration into the given config file that can be fed back
    to speed up subsequent runs (network configurations)}
\end{params}

\subsection{Ipv4FlatNetworkConfigurator (Legacy)}

The \nedtype{Ipv4FlatNetworkConfigurator} module configures
IP addresses and routes of IP nodes of a network.
All assigned addresses share a common subnet prefix,
the network topology will be ignored. Shortest path
routes are also generated from any node to any other
node of the network. The Gateway (next hop) field of the routes
is not filled in by these configurator, so it relies
on proxy ARP if the network spans several LANs.
It does not perform routing table optimization (i.e.
merging similar routes into a single, more general route.)

\begin{warning}
\nedtype{Ipv4FlatNetworkConfigurator} is considered
legacy, use do not use it for new projects.
\end{warning}

The \nedtype{Ipv4FlatNetworkConfigurator} module configures
the network when it is initialized. The configuration
is performed in stage 2, after interface tables are
filled in. Do not use a \nedtype{Ipv4FlatNetworkConfigurator}
module together with static routing files, because they
can iterfere with the configurator.

The \nedtype{Ipv4FlatNetworkConfigurator} searches each IP nodes of the network.
(IP nodes are those modules that have the @node NED property and
has a \nedtype{Ipv4RoutingTable} submodule named ``routingTable'').
The configurator then assigns IP addresses to the IP nodes, controlled
by the following module parameters:
\begin{itemize}
  \item \fpar{netmask} common netmask of the addresses (default is 255.255.0.0)
  \item \fpar{networkAddress} higher bits are the network part of the addresses,
        lower bits should be 0. (default is 192.168.0.0)
\end{itemize}

With the default parameters the assigned addresses are in the range
192.168.0.1 - 192.168.255.254, so there can be maximum 65534 nodes in the
network. The same IP address will be assigned to each interface
of the node, except the loopback interface which always has address 127.0.0.1
(with 255.0.0.0 mask).

After assigning the IP addresses, the configurator fills in the routing tables.
There are two kind of routes:
\begin{itemize}
  \item default routes: for nodes that has only one non-loopback interface
        a route is added that matches with any destination address
        (the entry has 0.0.0.0 \ttt{host} and \ttt{netmask} fields).
        These are remote routes, but the gateway address is left unspecified.
        The delivery of the datagrams rely on the proxy ARP feature of the
        routers.
  \item direct routes following the shortest paths: for nodes that has more
        than one non-loopback interface a separate route is added to each
        IP node of the network. The outgoing interface is chosen by the
        shortest path to the target node. These routes are
        added as direct routes, even if there is no direct link with the
        destination. In this case proxy ARP is needed to deliver the datagrams.
\end{itemize}

\begin{note}
This configurator does not try to optimize the routing tables.
If the network contains $n$ nodes, the size of all routing tables
will be proportional to $n^2$, and the time of the lookup of the
best matching route will be proportional to $n$.
\end{note}

% FIXME weird FlatNetworkConfigurator behaviour.
%       Assigned IP addresses does not mirror the hierachy of networks (e.g. each node in an Ethernet LAN handled as a one-element subnet).
%       No gateway address is set in the routes, delivery relies on proxy ARPing.
%       Direct routes created to each node, even if there is no direct link to it.
%       Different interfaces of a node should have different IP address.
%       Broadcast capable interfaces should have a real netmast (not 255.255.255.255) to support subnet directed IP broadcasts.

\subsection{Routing Files (Legacy)}
\label{subsec:routing_files}

Routing files are files with \ttt{.irt} or \ttt{.mrt} extension,
and their names are passed in the \fpar{routingFile} parameter
to \nedtype{Ipv4RoutingTable} modules.

Routing files may contain network interface configuration and static
routes. Both are optional. Network interface entries in the file
configure existing interfaces; static routes are added to the route table.

\begin{warning}
\nedtype{Routing files} are considered legacy, use do not use them for new
projects. Their contents can be expressed in \nedtype{Ipv4NetworkConfigurator}
config files.
\end{warning}

Interfaces themselves are represented in the simulation by modules
(such as the PPP module). Modules automatically register themselves
with appropriate defaults in the IPv4RoutingTable, and entries in the
routing file refine (overwrite) these settings.
Interfaces are identified by names (e.g. ppp0, ppp1, eth0) which
are normally derived from the module's name: a module called
\ttt{"ppp[2]"} in the NED file registers itself as interface ppp2.

An example routing file (copied here from one of the example simulations):

\begin{verbatim}
ifconfig:

# ethernet card 0 to router
name: eth0   inet_addr: 172.0.0.3   MTU: 1500   Metric: 1  BROADCAST MULTICAST
Groups: 225.0.0.1:225.0.1.2:225.0.2.1

# Point to Point link 1 to Host 1
name: ppp0   inet_addr: 172.0.0.4   MTU: 576   Metric: 1

ifconfigend.

route:
172.0.0.2   *           255.255.255.255  H  0   ppp0
172.0.0.4   *           255.255.255.255  H  0   ppp0
default:    10.0.0.13   0.0.0.0          G  0   eth0

225.0.0.1   *           255.255.255.255  H  0   ppp0
225.0.1.2   *           255.255.255.255  H  0   ppp0
225.0.2.1   *           255.255.255.255  H  0   ppp0

225.0.0.0   10.0.0.13   255.0.0.0        G  0   eth0

routeend.
\end{verbatim}

The \ttt{ifconfig...ifconfigend.} part configures interfaces,
and \ttt{route..routeend.} part contains static routes.
The format of these sections roughly corresponds to the output
of the \ttt{ifconfig} and \ttt{netstat -rn} Unix commands.

An interface entry begins with a \ttt{name:} field, and lasts until
the next \ttt{name:} (or until \ttt{ifconfigend.}). It may
be broken into several lines.

Accepted interface fields are:

\begin{itemize}
  \item \ttt{name:} - arbitrary interface name (e.g. eth0, ppp0)
  \item \ttt{inet\_addr:} - IP address
  \item \ttt{Mask:} - netmask
  \item \ttt{Groups:} Multicast groups. 224.0.0.1 is added automatically,
     and 224.0.0.2 also if the node is a router (IPForward==true).
  \item \ttt{MTU:} - MTU on the link (e.g. Ethernet: 1500)
  \item \ttt{Metric:} - integer route metric
  \item flags: \ttt{BROADCAST}, \ttt{MULTICAST}, \ttt{POINTTOPOINT}
\end{itemize}

The following fields are parsed but ignored: \ttt{Bcast},\ttt{encap},
\ttt{HWaddr}.

Interface modules set a good default for MTU, Metric (as $2*10^9$/bitrate) and
flags, but leave \fvar{inet\_addr} and \fvar{Mask} empty. \fvar{inet\_addr} and
\fvar{mask} should be set either from the routing file or by a dynamic network
configuration module.

The route fields are:

\begin{verbatim}
Destination  Gateway  Netmask  Flags  Metric Interface
\end{verbatim}

\fvar{Destination}, \fvar{Gateway} and \fvar{Netmask} have the usual meaning.
The \fvar{Destination} field should either be an IP address or ``default''
(to designate the default route). For \fvar{Gateway}, \ttt{*} is also
accepted with the meaning \ttt{0.0.0.0}.

\fvar{Flags} denotes route type:

\begin{itemize}
  \item \textit{H} ``host'': direct route (directly attached to the router), and
  \item \textit{G} ``gateway'': remote route (reached through another router)
\end{itemize}

\fvar{Interface} is the interface name, e.g. \ttt{eth0}.

\begin{important}
The meaning of the routes where the destination is a multicast address
has been changed in version 1.99.4. Earlier these entries was used
both to select the outgoing interfaces of multicast datagrams
sent by the higher layer (if multicast interface was otherwise unspecified)
and to select the outgoing interfaces of datagrams that are received from
the network and forwarded by the node.

From version 1.99.4 multicast routing applies reverse path forwarding.
This requires a separate routing table, that can not be populated from
the old routing table entries. Therefore simulations that use multicast
forwarding can not use the old configuration files, they should be
migrated to use an \nedtype{Ipv4NetworkConfigurator} instead.

Some change is needed in models that use link-local multicast too.
Earlier if the IP module received a datagram from the higher layer
and multiple routes was given for the multicast group,
then IP sent a copy of the datagram on each interface of that routes.
From version 1.99.4, only the first matching interface is used (considering
longest match). If the application wants to send the multicast datagram
on each interface, then it must explicitly loop and specify the multicast
interface.
\end{important}

% FIXME 'H' and 'G' flags should be independent. Now they excludes each other, the parser sets route.type to the last one.
%       H = host/network
%       G = indirect/direct

% TODO warn that multicast configuration has changed

\section{Configuring Layer 2}

The \nedtype{L2Configurator} module allows configuring network scenarios at layer 2.
The STP/RTP-related parameters such as link cost, port priority
and the ``is-edge'' flag can be configured with XML files.

This module is similar to \nedtype{Ipv4NetworkConfigurator}. It supports
the selector attributes \ttt{@hosts}, \ttt{@names}, \ttt{@towards}, \ttt{@among},
and they behave similarly to its \nedtype{Ipv4NetworkConfigurator} equivalent.
The \ttt{@ports} selector is also supported, for configuring per-port parameters.

The following example configures port 5 (if it exists) on all switches,
and sets cost=19 and priority=32768:

\begin{XML}
<config>
  <interface hosts='**' ports='5' cost='19' priority='32768'/>
</config>
\end{XML}

For more information about the usage of the selector attributes see
\nedtype{Ipv4NetworkConfigurator}.



%%% Local Variables:
%%% mode: latex
%%% TeX-master: "usman"
%%% End:


\cleardoublepage

\include{ch-lifecycle}
\cleardoublepage

\include{ch-scenario-management}
\cleardoublepage

\include{ch-collecting-results}
\cleardoublepage

\chapter{Visualization}
\label{cha:visualization}

\section{Overview}

The INET Framework is able to visualize a wide range of events and conditions
in the network: packet drops, data link connectivity, wireless signal path loss,
transport connections, routing table routes, and many more. Visualization is
implemented as a collection of configurable INET modules that can be added
to simulations at will.

\section{Visualizing Network Communication}

\subsection{Visualizing Packet Drops}

Several network problems manifest themselves as excessive packet drops, for
example poor connectivity, congestion, or misconfiguration. Visualizing packet
drops helps identifying such problems in simulations, thereby reducing time
spent on debugging and analysis. Poor connectivity in a wireless network can
cause senders to drop unacknowledged packets after the retry limit is exceeded.
Congestion can cause queues to overflow in a bottleneck router, again resulting
in packet drops.

Packet drops can be visualized by including a \nedtype{PacketDropVisualizer}
module in the simulation. The \nedtype{PacketDropVisualizer} module indicates
packet drops by displaying an animation effect at the node where the packet drop
occurs. In the animation, a packet icon gets thrown out from the node icon, and
fades away.

The visualization of packet drops can be enabled with the visualizer's
\fpar{displayPacketDrops} parameter. By default, packet drops at all nodes are
visualized. This selection can be narrowed with the \fpar{nodeFilter},
\fpar{interfaceFilter} and \fpar{packetFilter} parameters.

One can click on the packet drop icon to display information about the packet
drop in the inspector panel.

Packets are dropped for the following reasons:

\begin{itemize}
  \item queue overflow
  \item retry limit exceeded
  \item unroutable packet
  \item network address resolution failed
  \item interface down
\end{itemize}


\subsection{Visualizing Transport Path Activity}

With INET simulations, it is often useful to be able to visualize network
traffic. INET provides several visualizers for this task, operating at various
levels of the network stack.

Transport path activity can be visualized by including a
\nedtype{TransportRouteVisualizer} module in the simulation. \nedtype{TransportRouteVisualizer}
that can provide graphical feedback about transport traffic, i.e. traffic that passes
through the transport layers of two endpoints. Adding an \nedtype{IntegratedVisualizer} is
also an option, because it also contains a \nedtype{TransportRouteVisualizer}. Transport
path activity visualization is disabled by default, it can be enabled by setting
the visualizer's \fpar{displayRoutes} parameter to true.

\nedtype{TransportRouteVisualizer} observes packets that pass through the transport layer,
i.e. carry data from/to higher layers.

The activity between two nodes is represented visually by a polyline arrow which
points from the source node to the destination node. \nedtype{TransportRouteVisualizer}
follows packets throughout their path so that the polyline goes through all
nodes which are the part of the path of packets. The arrow appears after the
first packet has been received, then gradually fades out unless it is reinforced
by further packets. Color, fading time and other graphical properties can be
changed with parameters of the visualizer.

By default, all packets and nodes are considered for the visualization. This
selection can be narrowed with the visualizer's packetFilter and nodeFilter
parameters.

\subsection{Visualizing Network Path Activity}

With INET simulations, it is often useful to be able to visualize network
traffic. INET offers several visualizers for this task, operating at various
levels of the network stack. In this showcase, we examine \nedtype{NetworkRouteVisualizer}
that can provide graphical feedback about network layer level traffic.

Network path activity can be visualized by including a \nedtype{NetworkRouteVisualizer}
module in the simulation. Adding an \nedtype{IntegratedVisualizer} module is also an
option, because it also contains a \nedtype{NetworkRouteVisualizer} module. Network path
activity visualization is disabled by default, it can be enabled by setting the
visualizer's \fpar{displayRoutes} parameter to true.

\nedtype{NetworkRouteVisualizer} currently observes packets that pass through the network
layer (i.e. carry data from/to higher layers), but not those that are internal
to the operation of the network layer protocol. That is, packets such as ARP,
although potentially useful, will not trigger the visualization.

The activity between two nodes is represented visually by a polyline arrow which
points from the source node to the destination node. \nedtype{NetworkRouteVisualizer}
follows packet throughout its path so the polyline goes through all nodes that
are part of the packet's path. The arrow appears after the first packet has been
received, then gradually fades out unless it is reinforced by further packets.
Color, fading time and other graphical properties can be changed with parameters
of the visualizer.

By default, all packets and nodes are considered for the visualization. This
selection can be narrowed with the visualizer's packetFilter and nodeFilter
parameters.


\subsection{Visualizing Data Link Activity}

With INET simulations, it is often useful to be able to visualize network
traffic. INET offers several visualizers for this task, operating at various
levels of the network stack. In this showcase, we examine \nedtype{DataLinkVisualizer}
that can provide graphical feedback about data link level traffic.

Data link activity can be visualized by including a \nedtype{DataLinkVisualizer} module in
the simulation. Adding an \nedtype{IntegratedVisualizer} module is also an option, because
it also contains a \nedtype{DataLinkVisualizer} module. Data link visualization is
disabled by default, it can be enabled by setting the visualizer's displayLinks
parameter to true.

\nedtype{DataLinkVisualizer} currently observes packets that pass through the data link
layer (i.e. carry data from/to higher layers), but not those that are internal
to the operation of the data link layer protocol. That is, frames such as ACK,
RTS/CTS, Beacon or Authentication/Association frames of IEEE 802.11, although
potentially useful, will not trigger the visualization. Visualizing such frames
may be implemented in future INET revisions.

The activity between two nodes is represented visually by an arrow that points
from the sender node to the receiver node. The arrow appears after the first
packet has been received, then gradually fades out unless it is refreshed by
further packets. The style, color, fading time and other graphical properties
can be changed with parameters of the visualizer.

By default, all packets, interfaces and nodes are considered for the
visualization. This selection can be narrowed to certain packets and/or nodes
with the visualizer's \fpar{packetFilter}, \fpar{interfaceFilter}, and
\fpar{nodeFilter} parameters.


\subsection{Visualizing Physical Link Activity}

With INET simulations, it is often useful to be able to visualize network
traffic. For this task, there are several visualizers in INET, operating at
various levels of the network stack. In this showcase, we demonstrate working of
\nedtype{PhysicalLinkVisualizer} that can provide graphical feedback about physical layer
traffic.

Physical link activity can be visualized by including a \nedtype{PhysicalLinkVisualizer}
module in the simulation. Adding an \nedtype{IntegratedVisualizer} module is also an
option, because it also contains a \nedtype{PhysicalLinkVisualizer} module. Physical link
activity visualization is disabled by default, it can be enabled by setting the
visualizer's \fpar{displayLinks} parameter to true.

\nedtype{PhysicalLinkVisualizer} observes frames that pass through the physical layer,
i.e. are received correctly.

The activity between two nodes is represented visually by a dotted arrow which
points from the sender node to the receiver node. The arrow appears after the
first frame has been received, then gradually fades out unless it is refreshed
by further frames. Color, fading time and other graphical properties can be
changed with parameters of the visualizer.

By default, all packets, interfaces and nodes are considered for the
visualization. This selection can be narrowed with the visualizer's
\fpar{packetFilter}, \fpar{interfaceFilter}, and \fpar{nodeFilter} parameters.


\subsection{Visualizing Routing Tables}

In a complex network topology, it is difficult to see how a packet would be
routed because the relevant data is scattered among network nodes and hidden in
their routing tables. INET contains support for visualization of routing tables,
and can display routing information graphically in a concise way. Using
visualization, it is often possible to understand routing in a simulation
without looking into individual routing tables. The visualization currently
supports IPv4.

The \nedtype{RoutingTableVisualizer} module (included in the network as part of
\nedtype{IntegratedVisualizer}) is responsible for visualizing routing table entries.

The visualizer basically annotates network links with labeled arrows that
connect source nodes to next hop nodes. The module visualizes those routing
table entries that participate in the routing of a given set of destination
addresses, by default the addresses of all interfaces of all nodes in the
network. That is, it selects the best (longest prefix) matching routes for all
destination addresses from each routing table, and shows them as arrows that
point to the next hop. Note that one arrow might need to represent several
routing entries, for example when distinct prefixes are routed towards the same
next hop.

Routing table entries are represented visually by solid arrows. An arrow going
from a source node represents a routing table entry in the source node's routing
table. The endpoint node of the arrow is the next hop in the visualized routing
table entry. By default, the routing entry is displayed on the arrows in
following format:

\begin{verbatim}
destination/mask -> gateway (interface)
\end{verbatim}

The format can be changed by setting the visualizer's \fpar{labelFormat} parameter.

Filtering is also possible. The \fpar{nodeFilter} parameter controls which nodes'
routing tables should be visualized (by default, all nodes), and the
\fpar{destinationFilter} parameter selects the set of destination nodes to consider
(again, by default all nodes.)

The visualizer reacts to changes. For example, when a routing protocol changes a
routing entry, or an IP address gets assigned to an interface by DHCP, the
visualizer automatically updates the visualizations according to the specified
filters. This is very useful e.g. for the simulation of mobile ad-hoc networks.

\subsection{Displaying IP Addresses and Other Interface Information}

In the simulation of complex networks, it is often useful to be able to display
node IP addresses, interface names, etc. above the node icons or on the links.
For example, when automatic address assignment is used in a hierarchical network
(e.g. using \nedtype{Ipv4NetworkConfigurator}), visual inspection can help to
verify that the result matches the expectations. While it is true that addresses and other
interface data can also be accessed in the GUI by diving into the interface
tables of each node, that is tedious, and unsuitable for getting an overview.

The \nedtype{InterfaceTableVisualizer} module (included in the network as part of
\nedtype{IntegratedVisualizer}) displays data about network nodes' interfaces.
(Interfaces are contained in interface tables, hence the name.) By default, the
visualization is turned off. When it is enabled using the
\fpar{displayInterfaceTables} parameter, the default is that interface names, IP
addresses and netmask length are displayed, above the nodes (for wireless
interfaces) and on the links (for wired interfaces). By clicking on an interface
label, details are displayed in the inspector panel.

The visualizer has several configuration parameters. The \fpar{format} parameter
specifies what information is displayed about interfaces. It takes a format
string, which can contain the following directives:

\begin{itemize}
  \item \%N: interface name
  \item \%4: IPv4 address
  \item \%6: IPv6 address
  \item \%n: network address. This is either the IPv4 or the IPv6 address
  \item \%l: netmask length
  \item \%M: MAC address
  \item \%\textbackslash: conditional newline for wired interfaces. The '\textbackslash'
  needs to be escaped with another '\textbackslash', i.e. '\%\textbackslash\textbackslash'
  \item \%i and \%s: the info() and str() functions for the interfaceEntry class, respectively
\end{itemize}

The default format string is
\texttt{"\%N \%\textbackslash\textbackslash\%n/\%l"}, i.e. interface name, IP address and
netmask length.

The set of visualized interfaces can be selected with the configurator's
\fpar{nodeFilter} and \fpar{interfaceFilter} parameters. By default, all
interfaces of all nodes are visualized, except for loopback addresses (the default for the
\fpar{interfaceFilter} parameter is \texttt{"not lo\textbackslash*"}.)

It is possible to display the labels for wired interfaces above the node icons,
instead of on the links. This can be done by setting the
\fpar{displayWiredInterfacesAtConnections} parameter to false.

There are also several parameters for styling, such as color and font selection.


\subsection{Visualizing IEEE 802.11 Network Membership}

When simulating wifi networks that overlap in space, it is difficult to see
which node is a member of which network. The membership may even change over
time. It would be useful to be able to display e.g. the SSID above node icons.

IEEE 802.11 network membership can be visualized by including a
\nedtype{Ieee80211Visualizer} module in the simulation. Adding an \nedtype{IntegratedVisualizer} is
also an option, because it also contains a \nedtype{Ieee80211Visualizer}. Displaying
network membership is disabled by default, it can be enabled by setting the
visualizer's \fpar{displayAssociations} parameter to true.

The \nedtype{Ieee80211Visualizer} displays an icon and the SSID above network nodes which
are part of a wifi network. The icons are color-coded according to the SSID. The
icon, colors, and other visual properties can be configured via parameters of
the visualizer.

The visualizer's \fpar{nodeFilter} parameter selects which nodes' memberships are
visualized. The \fpar{interfaceFilter} parameter selects which interfaces are
considered in the visualization. By default, all interfaces of all nodes are
considered.


\subsection{Visualizing Transport Connections}

In a large network with a complex topology, there might be many transport layer
applications and many nodes communicating. In such a case, it might be difficult
to see which nodes communicate with which, or if there is any communication at
all. Transport connection visualization makes it easy to get information about
the active transport connections in the network at a glance. Visualization makes
it easy to identify connections by their two endpoints, and to tell different
connections apart. It also gives a quick overview about the number of
connections in individual nodes and the whole network.

The \nedtype{TransportConnectionVisualizer} module (also part of \nedtype{IntegratedVisualizer})
displays color-coded icons above the two endpoints of an active, established
transport layer level connection. The icons will appear when the connection is
established, and disappear when it is closed. Naturally, there can be multiple
connections open at a node, thus there can be multiple icons. Icons have the
same color at both ends of the connection. In addition to colors, letter codes
(A, B, AA, …) may also be displayed to help in identifying connections. Note
that this visualizer does not display the paths the packets take. If you are
interested in that, take a look at \nedtype{TransportRouteVisualizer}, covered in the
Visualizing Transport Path Activity showcase.

The visualization is turned off by default, it can be turned on by setting the
\fpar{displayTransportConnections} parameter of the visualizer to true.

It is possible to filter the connections being visualized. By default, all
connections are included. Filtering by hosts and port numbers can be achieved by
setting the \fpar{sourcePortFilter}, \fpar{destinationPortFilter},
\fpar{sourceNodeFilter} and \fpar{destinationNodeFilter} parameters.

The icon, colors and other visual properties can be configured by setting the
visualizer's parameters.


\section{Visualizing The Infrastructure}

\subsection{Visualizing the Physical Environment}

The physical environment has a profound effect on the communication of wireless
devices. For example, physical objects like walls inside buildings constraint
mobility. They also obstruct radio signals often resulting in packet loss. It's
difficult to make sense of the simulation without actually seeing where physical
objects are.

The visualization of physical objects present in the physical environment is
essential.

The \nedtype{PhysicalEnvironmentVisualizer} (also part of \nedtype{IntegratedVisualizer}) is
responsible for displaying the physical objects. The objects themselves are
provided by the PhysicalEnvironment module; their geometry, physical and visual
properties are defined in the XML configuration of the PhysicalEnvironment
module.

The two-dimensional projection of physical objects is determined by the
\nedtype{SceneCanvasVisualizer} module. (This is because the projection is also needed by
other visualizers, for example \nedtype{MobilityVisualizer}.) The default view is top view
(z axis), but you can also configure side view (x and y axes), or isometric or
ortographic projection.

The visualizer also supports OpenGL-based 3D rendering using the OpenSceneGraph
(OSG) library. If the OMNeT++ installation has been compiled with OSG
support, you can switch to 3D view using the Qtenv toolbar.

\subsection{Visualizing Node Mobility}

In INET simulations, the movement of mobile nodes is often as important as the
communication among them. However, as mobile nodes roam, it is often difficult
to visually follow their movement. INET provides a visualizer that not only
makes visually tracking mobile nodes easier, but also indicates other properties
like speed and direction.

Node mobility of nodes can be visualized by \nedtype{MobilityVisualizer} module
(included in the network as part of \nedtype{IntegratedVisualizer}). By default,
mobility visualization is enabled, it can be disabled by setting
\fpar{displayMovements} parameter to false.

By default, all mobilities are considered for the visualization. This selection
can be narrowed with the visualizer's \fpar{moduleFilter} parameter.

The visualizer has several important features:

\begin{itemize}
  \item Movement Trail: It displays a line along the recent path of movements.
        The trail gradually fades out as time passes. Color, trail length and
        other graphical properties can be changed with parameters of the
        visualizer.
  \item Velocity Vector: Velocity is represented visually by an arrow. Its
        starting point is the node, and its direction coincides with the
        movement's direction. The arrow's length is proportional to the node's
       speed.
  \item Orientation Arc: Node orientation is represented by an arc whose size
       is specified by the \fpar{orientationArcSize} parameter. This value is the
       relative size of the arc compared to a full circle. The arc's default
       value is 0.25, i.e. a quarter of a circle.
\end{itemize}

These features are disabled by default; they can be enabled by setting the
visualizer's \fpar{displayMovementTrails}, \fpar{displayVelocities} and
\fpar{displayOrientations} parameters to true.


\section{Generic}

The following pages describe generic features that are common to many visualizers.

TODO Styling and Appearance (from the showcase)


%%% Local Variables:
%%% mode: latex
%%% TeX-master: "usman"
%%% End:


\cleardoublepage

\include{ch-authors-guide}
\cleardoublepage

\include{ch-history}
\cleardoublepage

\bibliographystyle{alpha}
\bibliography{inet-users-guide}


%% no need for the following since 'tocbibind' package
%% \phantomsection
%% \addcontentsline{toc}{chapter}{\indexname}
\printindex

\end{document}

%%% Local Variables:
%%% mode: latex
%%% TeX-master: t
%%% End:
